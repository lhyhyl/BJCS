\documentclass[lang=cn,newtx,10pt,scheme=chinese]{../../elegantbook}

\title{基础提高练习题}
\subtitle{北街学长倾力之作}

\author{北街}
% \institute{Elegant\LaTeX{} Program}
\date{2022/12/31}
\version{1.0}
% \bioinfo{自定义}{信息}

% \extrainfo{注意:本模板自 2023 年 1 月 122222 日开始,不再更新和维护!}

\setcounter{tocdepth}{3}

\logo{../../figure/logo-blue.png}
\cover{../../figure/cover.jpg}

% 本文档命令
\usepackage{array}
\newcommand{\ccr}[1]{\makecell{{\color{#1}\rule{1cm}{1cm}}}}

% 修改标题页的橙色带
\definecolor{customcolor}{RGB}{32,178,170}
\colorlet{coverlinecolor}{customcolor}
\usepackage{cprotect}

\addbibresource[location=local]{reference.bib} % 参考文献,不要删除
\usepackage{listings}         % 导入listings宏包
\usepackage{xcolor}           % 支持颜色

% 配置C++代码样式
\lstset{
    language=C++,             % 语言设置为C++
    basicstyle=\ttfamily,      % 基本样式
    keywordstyle=\color{blue}, % 关键词颜色
    commentstyle=\color{green},% 注释颜色
    stringstyle=\color{red},   % 字符串颜色
    numbers=left,              % 显示行号
    numberstyle=\tiny,         % 行号样式
    stepnumber=1,              % 每行显示行号
    breaklines=true,           % 自动换行
    frame=lines                % 代码块边框样式
}
\begin{document}

\maketitle
\frontmatter

\tableofcontents

\mainmatter

\chapter{课堂练习题}
\section{操作系统概述}
\subsection{单项选择题}
\begin{enumerate}
    \item (名校考研真题,单项选择题,2分) 对于一台PC而言,下列各项中 (    ) 对系统必不可少。
    \begin{enumerate}[A.]
        \item OS
        \item BIOS
        \item C语言编辑器
        \item 杀毒软件
    \end{enumerate}

    \item (名校考研真题,单项选择题,2分) 从用户的角度看,OS是 (   )。
    \begin{enumerate}[A.]
        \item 用户与计算机硬件系统之间的接口
        \item 控制和管理计算机系统资源的软件
        \item 合理组织计算机工作流程的软件
        \item 一个大型的工具软件
    \end{enumerate}

    \item (全国统考真题,单项选择题,2分) 计算机开机后,OS最终被加载到 (    )。
    \begin{enumerate}[A.]
        \item BIOS
        \item ROM
        \item EPROM
        \item RAM
    \end{enumerate}

    \item (名校考研真题,单项选择题,2分) 配置了OS的计算机是一台比原来的物理计算机功能更加强大的计算机,这样的计算机只是一台逻辑上的计算机,称为 (    ) 计算机。
    \begin{enumerate}[A.]
        \item 虚拟
        \item 物理
        \item 并行
        \item 共享
    \end{enumerate}

    \item (全国统考真题,单项选择题,2分) 与单道程序系统相比,多道程序系统的优点是 (    )。
    \begin{enumerate}[A.]
        \item 仅 I、III
        \item 仅 I、IV
        \item 仅 II、III
        \item 仅 I、II、IV
    \end{enumerate}

    \item (名校考研真题,单项选择题,2分) 引入多道程序技术的前提条件之一是系统具有 (    )。
    \begin{enumerate}[A.]
        \item 分时功能
        \item 中断功能
        \item 多CPU技术
        \item SPOOLing技术
    \end{enumerate}

    \item (名校考研真题,单项选择题,2分) 下列对OS的叙述中,正确的是 (    )。
    \begin{enumerate}[A.]
        \item OS都在内核态运行
        \item 分时系统中常用的原则是使时间片越小越好
        \item 批处理系统的主要缺点是缺少交互性
        \item DOS是一个单用户多任务的OS
    \end{enumerate}

    \item (名校考研真题,单项选择题,2分) OS的基本类型主要有 (    )。
    \begin{enumerate}[A.]
        \item 批处理系统、分时系统和多任务系统
        \item 批处理系统、分时系统和实时系统
        \item 单用户系统、多用户系统和批处理系统
        \item 实时系统、分时系统和多用户系统
    \end{enumerate}

    \item (全国统考真题,单项选择题,2分) 下列关于批处理系统的叙述中,正确的是 (    )。
    \begin{enumerate}[A.]
        \item 仅 II、III
        \item 仅 I
        \item 仅 I、II
        \item 仅 I、III
    \end{enumerate}

    \item (名校考研真题,单项选择题,2分) (    ) 系统允许一台主机上同时连接多台终端,多个用户可以通过各自的终端同时交互地使用计算机。
    \begin{enumerate}[A.]
        \item 网络
        \item 分布式
        \item 分时
        \item 实时
    \end{enumerate}

    \item (名校考研真题,单项选择题,2分) 分时系统的主要目的是 (    )。
    \begin{enumerate}[A.]
        \item 充分利用IO设备
        \item 比较快速地响应用户
        \item 提高系统吞吐量
        \item 充分利用内存
    \end{enumerate}
    \item (名校考研真题,单项选择题,2分) 下列 (    ) 等的实现最好采用实时系统平台。
    \begin{enumerate}[A.]
        \item 航空订票系统、机床控制系统
        \item 办公自动化系统、机床控制系统、AutoCAD
        \item 机床控制系统、工资管理系统
        \item 航空订票系统、机床控制系统、股票交易系统
    \end{enumerate}

    \item (全国统考真题,单项选择题,2分) 下列关于多任务OS的叙述中,正确的是 (    )。
    \begin{enumerate}[A.]
        \item 仅 I
        \item 仅 II
        \item 仅 I、II
        \item I、II、III
    \end{enumerate}

    \item (名校考研真题,单项选择题,2分) 并发性是指若干事件在 (    ) 发生。
    \begin{enumerate}[A.]
        \item 同一时刻
        \item 不同时刻
        \item 同一时间间隔内
        \item 不同时间间隔内
    \end{enumerate}

    \item (全国统考真题,单项选择题,2分) 单处理机系统中,可并行的是 (    )。
    \begin{enumerate}[A.]
        \item 进程与进程
        \item 处理机与设备
        \item 处理机与通道
        \item 设备与设备
    \end{enumerate}

    \item (全国统考真题,单项选择题,2分) 中断处理和子程序调用都需要压栈以保护现场,中断处理一定会保存而子程序调用不需要保存其内容的是 (    )。
    \begin{enumerate}[A.]
        \item 程序计数器
        \item 程序状态字寄存器
        \item 通用数据寄存器
        \item 通用地址寄存器
    \end{enumerate}

    \item (全国统考真题,单项选择题,2分) 内部异常(内中断)可分为故障(fault)、陷阱(trap)和终止(abort)3类。下列有关内部异常的叙述中,错误的是 (    )。
    \begin{enumerate}[A.]
        \item 内部异常的产生与当前执行的指令相关
        \item 内部异常的检测由CPU的内部逻辑实现
        \item 内部异常的响应发生在指令执行过程中
        \item 内部异常处理后系统会返回到发生异常的指令继续执行
    \end{enumerate}

    \item (全国统考真题,单项选择题,2分) 异常是指令执行过程中在处理机内部发生的特殊事件,中断是来自处理机外部的请求事件。下列关于中断和异常的叙述中,错误的是 (    )。
    \begin{enumerate}[A.]
        \item “访问内存时缺页”属于中断
        \item “整数除以零”属于异常
        \item “DMA传送结束”属于中断
        \item “存储保护错”属于异常
    \end{enumerate}

    \item (全国统考真题,单项选择题,2分) 处理外部中断时,应该由OS保存的是 (    )。
    \begin{enumerate}[A.]
        \item 程序计数器的内容
        \item 通用寄存器的内容
        \item 快表中的内容
        \item Cache中的内容
    \end{enumerate}

    \item (全国统考真题,单项选择题,2分) 本地用户通过键盘登录系统时,首先获得键盘输入信息的程序是 (    )。
    \begin{enumerate}[A.]
        \item 命令解释程序
        \item 中断处理程序
        \item 系统调用服务程序
        \item 用户登录程序
    \end{enumerate}

    \item (全国统考真题,单项选择题,2分) 定时器产生时钟中断后,由时钟中断处理程序更新的部分内容是 (    )。
    \begin{enumerate}[A.]
        \item 仅内核中时钟变量的值
        \item 仅当前进程在时间片内的剩余执行时间
        \item 仅内核中时钟变量的值和当前进程占用CPU的时间
        \item 内核中时钟变量的值、当前进程在时间片内的剩余执行时间和当前进程占用CPU的时间
    \end{enumerate}
    \item (全国统考真题,单项选择题,2分) 下列选项中,会导致用户进程从用户态切换到内核态的操作是 (    )。
    \begin{enumerate}[A.]
        \item 整数除以零
        \item sin()函数调用
        \item read系统调用
        \item I、II、III
    \end{enumerate}

    \item (名校考研真题,单项选择题,2分) OS中有一组特殊的程序,它们不能被系统中断。在OS中它们称为 (    )。
    \begin{enumerate}[A.]
        \item 初始化程序
        \item 原语
        \item 子程序
        \item 控制模块
    \end{enumerate}

    \item (全国统考真题,单项选择题,2分) 下列选项中,OS提供给应用程序的接口是 (    )。
    \begin{enumerate}[A.]
        \item 系统调用
        \item 中断
        \item 库函数
        \item 原语
    \end{enumerate}

    \item (全国统考真题,单项选择题,2分) 若一个用户进程通过read系统调用读取一个磁盘文件中的数据,则下列关于此过程的叙述中,正确的是 (    )。
    \begin{enumerate}[A.]
        \item 若该文件的数据不在内存中,则该进程进入睡眠等待状态
        \item 请求read系统调用会导致CPU从用户态切换到内核态
        \item read系统调用的参数应包含文件名称
        \item I、II、III
    \end{enumerate}

    \item (全国统考真题,单项选择题,2分) 执行系统调用的过程包括如下主要操作:①返回用户态;②执行陷入(trap)指令;③传递系统调用参数;④执行相应的服务程序。正确的执行顺序是 (    )。
    \begin{enumerate}[A.]
        \item ①②③④
        \item ②③④①
        \item ③④①②
        \item ③②④①
    \end{enumerate}

    \item (全国统考真题,单项选择题,2分) 下列关于系统调用的叙述中,正确的是 (    )。
    \begin{enumerate}[A.]
        \item 在执行系统调用服务程序的过程中,CPU处于内核态
        \item OS通过提供系统调用来避免用户程序直接访问外设
        \item 不同的OS为应用程序提供了统一的系统调用接口
        \item 系统调用是OS内核为应用程序提供服务的接口
    \end{enumerate}
\end{enumerate}

\subsection{填空题}

\begin{enumerate}
    \item (名校考研真题,填空题,2分) OS是计算机系统中的一个 (    ),它负责管理和控制计算机系统中的 (    )。

    \item (名校考研真题,填空题,2分) 现代OS的基本特性是并发性、(    )、(    ) 和异步性。

    \item (全国统考真题,填空题,2分) 某设备中断请求的响应时间和处理时间为100ns,每400ns发出一次中断请求,中断响应所允许的最长时延为0ns,则在该设备持续工作的过程中,CPU用于该设备的I/O时间占整个CPU时间的百分比至少是 (    )。

    \item (名校考研真题,填空题,2分) 系统执行原语操作时,要 (    ) (允许/禁止)中断。

    \item (全国统考真题,填空题,2分) UNIX系统是一个 (    )。

    \item (名校考研真题,填空题,2分) (    ) 是指OS仅将应用必需的所有核心功能放入内核,其他功能都放在内核之外,由处在用户态运行的服务进程实现。
\end{enumerate}

\subsection{简答题}
\begin{enumerate}
    \item (名校考研真题,简答题,10分) 方便性和有效性是设计OS的两个主要目标,以两种OS的技术为例,分别说明它们是如何实现这两个目标(一个是实现方便性的例子,一个是实现有效性的例子)。

    \item (名校考研真题,简答题,10分) 在分时系统中,为使多个进程能够及时与系统交互,最关键的问题是能在短时间内使所有就绪进程都能运行。当就绪进程数为100时,为保证响应时间不超过2s,此时的时间片最大是多少?

    \item (全国统考真题,简答题,10分) 某单CPU系统中有输入设备和输出设备各1台,现有3个并发执行的作业,每个作业的输入、计算和输出时间分别为2ms、3ms和4ms,且都按输入、计算和输出的顺序执行,则执行完这3个作业需要的时间最少是多少?

    \item (名校考研真题,简答题,10分) 处理机为什么要区分内核态和用户态?在什么情况下进行两种状态的转换?

    \item (名校考研真题,简答题,10分) 叙述系统调用的概念和OS提供系统调用的原因。
\end{enumerate}
\subsection{综合应用题}

\begin{enumerate}
    \item (名校考研真题,综合应用题,8分) 现有A、B两个程序,程序A按顺序使用CPU 10s,使用设备甲5s,使用CPU 5s,使用设备乙5s,最后使用CPU 10s。程序B按顺序使用设备甲10s,使用CPU 10s,使用设备乙5s,使用CPU 5s,使用设备乙10s,试问:
    \begin{enumerate}
        \item 在顺序执行程序A和程序B的情况下,CPU的利用率是多少?
        \item 在多道程序环境下,CPU的利用率是多少?请画出A、B程序的执行过程。
        \item 在多道批处理系统中,是否并发的进程越多,资源利用率越好?为什么?
    \end{enumerate}

    \item (名校考研真题,综合应用题,8分) 设某计算机系统中有一个CPU、一台输入设备、一台打印机。现有两个进程同时进入就绪状态,且进程A先得到CPU运行,进程B后运行。进程A的运行情况为:计算50ms,打印信息100ms,再计算50ms,再打印信息100ms,结束。进程B的运行情况为:计算50ms,输入数据80ms,再计算100ms,结束。画出它们的运行图,并说明:
    \begin{enumerate}
        \item 开始运行后,CPU有无进行空闲等待?若有,则请说明其在哪段时间内进行了空闲等待,并计算CPU的利用率。
        \item 进程A运行时有无等待现象?若有,在何时发生等待现象?
        \item 进程B运行时有无等待现象?若有,在何时发生等待现象?
    \end{enumerate}
\end{enumerate}

\end{document}
