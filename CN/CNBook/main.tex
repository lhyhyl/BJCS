\documentclass[lang=cn,newtx,10pt,scheme=chinese]{../../elegantbook}

\title{408计算机网络全册讲义}
\subtitle{北街学长倾力之作}

\author{北街}
\date{2025/08/17}
\version{1.0}


\setcounter{tocdepth}{3}

\logo{\detokenize{../../figure/logo-blue.png}}
\cover{\detokenize{../../figure/cover.jpg}}

% 本文档命令
% 可选:快速编译开关(在 main-fast.tex 中开启)。开启后图片以占位框代替、显著加速。
\ifdefined\FAST
    \PassOptionsToPackage{draft}{graphicx}
\fi
% 超快速模式:进一步优化编译速度
\ifdefined\ULTRAFAST
    \PassOptionsToPackage{draft}{graphicx}
    % 跳过复杂的表格包
    \let\longtable\tabular
    \let\endlongtable\endtabular
    % 简化数学环境
    \let\align\equation
    \let\endalign\endequation
\fi
% ———— 以上选项必须出现在 \usepackage{graphicx} 之前 ————
\usepackage{array}
\usepackage{makecell}
\usepackage{longtable}
\usepackage{booktabs}
\usepackage{amsmath}
\newcommand{\ccr}[1]{\makecell{{\color{#1}\rule{1cm}{1cm}}}}
% 收紧 longtable 上下间距,减少分页处垂直胶水
\setlength{\LTpre}{0pt}
\setlength{\LTpost}{0pt}
% 提高断行容错,缓解英文字母/数字混排导致的 Over/Underfull \hbox
\emergencystretch=1em

% 修改标题页的橙色带
\definecolor{customcolor}{RGB}{32,178,170}
\colorlet{coverlinecolor}{customcolor}
\usepackage{cprotect}

\addbibresource[location=local]{reference.bib} % 参考文献,不要删除
\usepackage{listings}         % 导入listings宏包
\usepackage{xcolor}           % 支持颜色
\usepackage{graphicx}        % 支持图形
\graphicspath{{../../figure/}} % 为封面/徽标等图像提供搜索路径
% 配置C++代码样式
\lstset{
    language=C++,             % 语言设置为C++
    basicstyle=\ttfamily,      % 基本样式
    keywordstyle=\color{blue}, % 关键词颜色
    commentstyle=\color[rgb]{0,0.5,0},% 注释颜色  stringstyle=\color{red},   % 字符串颜色
    numbers=left,              % 显示行号
    numberstyle=\tiny,         % 行号样式
    stepnumber=1,              % 每行显示行号
    breaklines=true,           % 自动换行
    frame=lines                % 代码块边框样式
}

\usepackage{tikz}
\usetikzlibrary{chains, positioning, arrows.meta, shapes.geometric, decorations.pathreplacing, external}
% Enable externalization only when explicitly requested (requires -shell-escape)
\ifdefined\EXTTIKZ
\tikzexternalize[prefix=./tikz-cache/]
% 修复TikZ外部化设置
\tikzset{external/system call={xelatex -shell-escape -halt-on-error -interaction=batchmode -jobname "\image" "\texsource"}}
\fi
% 快速模式下简化TikZ图形
\ifdefined\FAST
\tikzset{
    every picture/.style={baseline},
    draft/.style={draw=gray, fill=none}
}
\fi
\begin{document}


% 全局定义 TikZ 样式,供所有图使用
\tikzset{
    computer/.style={rectangle, draw=blue!50, fill=blue!20, thick, minimum width=1.5cm, minimum height=1cm, rounded corners},
    device/.style={rectangle, draw=blue!50, fill=blue!20, thick, minimum width=1cm, minimum height=0.8cm},
    arrow/.style={->, >=Stealth, thick, blue!70},
    nodeText/.style={align=center, font=\small}
}

\maketitle
\frontmatter

\tableofcontents

\mainmatter
% 减少版心撑满的强制要求,缓解 Underfull \vbox 提示
\raggedbottom

% ================== 408计算机网络全册讲义 ==================
% 根据2025年408大纲结构生成章节骨架,逐步填充讲义内容

% ========== 第1章 计算机网络概述 ==========
\chapter{计算机网络概述}

\section{计算机网络的定义、组成与功能}
\subsection{计算机网络的定义}

计算机网络就像是一个巨大的"朋友圈",只不过这个朋友圈里的"朋友"都是计算机设备。想象一下,如果你家里的电脑、手机、平板、智能电视都能互相聊天、分享文件,甚至一起玩游戏,那就是计算机网络的魅力所在!

\subsubsection{官方定义}
**计算机网络是指将地理位置不同的具有独立功能的多台计算机及其外部设备,通过通信线路连接起来,在网络操作系统、网络管理软件及网络通信协议的管理和协调下,实现资源共享和信息传递的计算机系统。**

这个定义听起来很官方对不对?让我们把它"翻译"成人话:

\begin{center}
\begin{tikzpicture}[node distance=2cm, auto]
    
    % 绘制计算机节点
    \node[computer] (pc1) {电脑A};
    \node[computer, right=3cm of pc1] (pc2) {电脑B};
    \node[computer, below=2cm of pc1] (pc3) {电脑C};
    \node[computer, below=2cm of pc2] (pc4) {电脑D};
    
    % 绘制连接线
    \draw[arrow] (pc1) -- (pc2) node[midway, above] {通信线路};
    \draw[arrow] (pc1) -- (pc3);
    \draw[arrow] (pc2) -- (pc4);
    \draw[arrow] (pc3) -- (pc4);
    \draw[arrow] (pc1) -- (pc4);
    \draw[arrow] (pc2) -- (pc3);
    
    % 添加标注
    \node[nodeText, above=0.5cm of pc1] {\textbf{不同地理位置}};
    \node[nodeText, below=1cm of pc3] {\textbf{资源共享}};
    \node[nodeText, below=1cm of pc4] {\textbf{信息传递}};
\end{tikzpicture}
\end{center}

\subsubsection{通俗理解}
把计算机网络想象成一个"智能社区":

\begin{itemize}
    \item \textbf{居民}:各种计算机设备(就像社区里的住户)
    \item \textbf{道路}:通信线路(连接各家各户的道路网络)
    \item \textbf{交通规则}:网络协议(大家都要遵守的"行车规则")
    \item \textbf{社区管理}:网络操作系统(负责维护秩序的物业管理)
    \item \textbf{目标}:资源共享和信息交流(邻里互助,分享资源)
\end{itemize}

\subsubsection{定义中的关键要素分析}

\begin{longtable}{|p{3cm}|p{10cm}|}
\hline
\textbf{关键要素} & \textbf{详细解释} \\
\hline
\textbf{独立功能} & 
每台计算机都有自己的"人格"——独立的处理能力、存储能力和运行程序的能力。它们不是"傀儡",而是有"思想"的个体。 \\
\hline
\textbf{通信线路} & 
就像人与人之间需要语言交流一样,计算机之间需要"说话的渠道"。可以是网线、WiFi、光纤,甚至是卫星信号。 \\
\hline
\textbf{网络协议} & 
这是计算机界的"世界语"。就像我们说中文、英文一样,计算机也需要统一的"语言"来交流,比如TCP/IP协议。 \\
\hline
\textbf{资源共享} & 
这是网络的精髓!就像室友可以共享WiFi、打印机一样,网络让计算机可以共享文件、程序、甚至计算能力。 \\
\hline
\textbf{信息传递} & 
计算机之间可以发送消息、传输文件、视频通话等。就像微信群聊,但参与者是计算机。 \\
\hline
\end{longtable}

\begin{center}
\begin{tikzpicture}[scale=0.8]
    % 绘制高速公路
    \draw[thick, gray!70] (0,0) -- (10,0);
    \draw[thick, gray!70] (0,0.5) -- (10,0.5);
    \draw[dashed, white, thick] (0,0.25) -- (10,0.25);
    
    % 绘制计算机/城市
    \node[computer] at (1,1) {北京};
    \node[computer] at (5,1) {上海};
    \node[computer] at (9,1) {深圳};
    
    % 绘制连接线
    \draw[arrow] (1,0.8) -- (1,0.5);
    \draw[arrow] (5,0.8) -- (5,0.5);
    \draw[arrow] (9,0.8) -- (9,0.5);
    
    % 绘制数据包(汽车)
    \node[draw, fill=red!30, circle, scale=0.7] at (3,0.25) {数据};
    \node[draw, fill=blue!30, circle, scale=0.7] at (7,0.25) {文件};
    
    % 添加标签
    \node[below] at (5,-0.5) {\textbf{数字高速公路(互联网)}};
    \node[above] at (3,1.5) {\textbf{各个"数字城市"(计算机节点)}};
\end{tikzpicture}
\end{center}

在这个比喻中:
\begin{itemize}
    \item \textbf{高速公路} = 网络基础设施(光纤、路由器等)
    \item \textbf{城市} = 网络中的计算机节点
    \item \textbf{汽车} = 传输的数据包
    \item \textbf{交通规则} = 网络协议
    \item \textbf{目的地} = 实现资源共享和信息传递
\end{itemize}

\subsubsection{408考试重点提醒}

\begin{center}
\fcolorbox{red}{yellow!20}{
\begin{minipage}{0.9\textwidth}
\textbf{考试高频考点:}
\begin{enumerate}
    \item 计算机网络定义中的\textbf{关键要素}(地理分布、独立功能、互连、协议、共享)
    \item 网络与分布式系统的\textbf{区别}(网络强调通信,分布式强调协作)
    \item 计算机网络的\textbf{本质目标}:资源共享和信息传递
\end{enumerate}
\end{minipage}
}
\end{center}

\textbf{小贴士}:记住计算机网络就是让"散落各地的计算机朋友们手牵手,一起分享,共同成长"!

\subsection{计算机网络的组成}

计算机网络就像搭积木一样,由各种不同的"积木块"组合而成。每个积木块都有自己独特的作用,缺一不可!让我们来看看这个"网络积木盒"里都有什么宝贝。

\subsubsection{硬件组成——网络的"筋骨"}

\textbf{1. 计算机主机(Host)}

主机就是网络中的"居民",分为两大类:

\begin{center}
\begin{tikzpicture}[node distance=3cm, auto]
    % 定义节点样式
    \tikzset{
        server/.style={rectangle, draw=green!50, fill=green!20, thick, minimum width=2cm, minimum height=1.5cm, rounded corners, align=center},
        client/.style={rectangle, draw=blue!50, fill=blue!20, thick, minimum width=2cm, minimum height=1.5cm, rounded corners, align=center},
        arrow/.style={->, >=stealth, thick},
        nlabel/.style={font=\small\bfseries}
    }
    
    % 服务器端
    \node[server] (server) {服务器\\(Server)};
    \node[nlabel, above=0.3cm of server] {提供服务};
    
    % 客户端
    \node[client, right=4cm of server] (client) {客户机\\(Client)};
    \node[nlabel, above=0.3cm of client] {使用服务};
    
    % 连接箭头
    \draw[arrow, green!70] (server) -- (client) node[midway, above] {提供资源};
    \draw[arrow, blue!70] (client) -- (server) node[midway, below] {请求服务};
    
    % 举例
    \node[below=1cm of server, text width=3cm, align=center] {\small 例如:\\网站服务器\\文件服务器\\邮件服务器};
    \node[below=1cm of client, text width=3cm, align=center] {\small 例如:\\个人电脑\\手机\\平板};
\end{tikzpicture}
\end{center}

\textbf{2. 通信链路(Communication Link)}

通信链路就是网络的"血管系统",负责在设备之间传输数据:

\begin{longtable}{|p{3cm}|p{5cm}|p{5cm}|}
\hline
\textbf{类型} & \textbf{特点} & \textbf{生活例子} \\
\hline
\textbf{有线链路} & 
稳定可靠,速度快,但布线麻烦 & 
就像家里的网线,稳定但不能随便移动 \\
\hline
\textbf{无线链路} & 
灵活方便,但容易受干扰 & 
就像WiFi,方便但信号会被墙壁影响 \\
\hline
\textbf{光纤链路} & 
速度超快,距离超远 & 
就像高铁,又快又远,但造价昂贵 \\
\hline
\end{longtable}

\textbf{3. 交换设备(Switching Equipment)}

交换设备是网络中的"交通枢纽",负责数据的转发和路由:

\begin{center}
\begin{tikzpicture}[scale=0.9]
    % 定义样式
    \tikzset{
        switch/.style={rectangle, draw=orange!70, fill=orange!30, thick, minimum width=1.5cm, minimum height=1cm},
        router/.style={rectangle, draw=red!70, fill=red!30, thick, minimum width=1.5cm, minimum height=1cm},
        hub/.style={circle, draw=gray!70, fill=gray!30, thick, minimum size=1.2cm},
        device/.style={rectangle, draw=blue!50, fill=blue!20, thick, minimum width=1cm, minimum height=0.8cm}
    }
    
    % 集线器层
    \node[hub] (hub) at (2,4) {集线器};
    \node[device] (pc1) at (0,5) {PC1};
    \node[device] (pc2) at (1,3) {PC2};
    \node[device] (pc3) at (3,3) {PC3};
    \node[device] (pc4) at (4,5) {PC4};
    
    % 连接集线器
    \draw[thick] (hub) -- (pc1);
    \draw[thick] (hub) -- (pc2);
    \draw[thick] (hub) -- (pc3);
    \draw[thick] (hub) -- (pc4);
    
    % 交换机层
    \node[switch] (switch) at (8,4) {交换机};
    \node[device] (pc5) at (6,5) {PC5};
    \node[device] (pc6) at (7,3) {PC6};
    \node[device] (pc7) at (9,3) {PC7};
    \node[device] (pc8) at (10,5) {PC8};
    
    % 连接交换机
    \draw[thick] (switch) -- (pc5);
    \draw[thick] (switch) -- (pc6);
    \draw[thick] (switch) -- (pc7);
    \draw[thick] (switch) -- (pc8);
    
    % 路由器
    \node[router] (router) at (5,1.5) {路由器};
    \draw[thick] (hub) -- (router);
    \draw[thick] (switch) -- (router);
    
    % 标签
    \node[above=0.3cm of hub] {\small\textbf{物理层设备}};
    \node[above=0.3cm of switch] {\small\textbf{数据链路层设备}};
    \node[below=0.3cm of router] {\small\textbf{网络层设备}};
\end{tikzpicture}
\end{center}

\subsubsection{软件组成——网络的"大脑"}

网络硬件只是"身体",软件才是"灵魂"!

\textbf{1. 网络操作系统}

就像手机需要iOS或Android系统一样,网络也需要专门的操作系统:

\begin{itemize}
    \item \textbf{功能}:管理网络资源,提供网络服务
    \item \textbf{举例}:Windows Server、Linux、Unix
    \item \textbf{比喻}:就像一个超级管家,负责分配房间、管理钥匙、安排访客
\end{itemize}

\textbf{2. 网络协议}

协议是网络世界的"法律条文",规定了数据传输的规则:

\begin{center}
\begin{tikzpicture}[node distance=1.5cm]
    % 协议栈
    \node[draw, fill=red!20, minimum width=6cm, minimum height=0.8cm] (app) {应用层协议 (HTTP, FTP, SMTP)};
    \node[draw, fill=orange!20, minimum width=6cm, minimum height=0.8cm, below=0.1cm of app] (trans) {传输层协议 (TCP, UDP)};
    \node[draw, fill=yellow!20, minimum width=6cm, minimum height=0.8cm, below=0.1cm of trans] (net) {网络层协议 (IP, ICMP)};
    \node[draw, fill=green!20, minimum width=6cm, minimum height=0.8cm, below=0.1cm of net] (link) {数据链路层协议 (Ethernet, WiFi)};
    \node[draw, fill=blue!20, minimum width=6cm, minimum height=0.8cm, below=0.1cm of link] (phy) {物理层协议 (电信号, 光信号)};
    
    % 标签
    \node[left=1cm of app] {\textbf{协议栈}};
    \node[right=1cm of app] {\small 应用程序间通信};
    \node[right=1cm of trans] {\small 端到端可靠传输};
    \node[right=1cm of net] {\small 路径选择和寻址};
    \node[right=1cm of link] {\small 相邻节点通信};
    \node[right=1cm of phy] {\small 比特流传输};
\end{tikzpicture}
\end{center}

\textbf{3. 应用软件}

这些是用户直接接触的软件,就像手机上的各种APP:

\begin{itemize}
    \item \textbf{浏览器}:Chrome、Firefox(用来上网冲浪的)
    \item \textbf{即时通讯}:微信、QQ(用来聊天吹水的)
    \item \textbf{文件传输}:FTP客户端(用来传输文件的)
    \item \textbf{邮件客户端}:Outlook、Thunderbird(用来收发邮件的)
\end{itemize}

\subsubsection{按功能划分的组成结构}

计算机网络从功能角度可以分为两个子网:

\begin{center}
\begin{tikzpicture}[node distance=2cm]
    % 定义样式
    \tikzset{
        subnet/.style={rectangle, draw=blue!70, fill=blue!20, thick, minimum width=4cm, minimum height=3cm, rounded corners},
        component/.style={rectangle, draw=green!50, fill=green!20, thick, minimum width=2.5cm, minimum height=0.8cm, rounded corners}
    }
    
    % 通信子网
    \node[subnet] (comm) at (0,0) {};
    \node[above=0.1cm of comm.north] {\textbf{通信子网}};
    \node[component] (router1) at (0,0.5) {路由器};
    \node[component] (switch1) at (0,-0.3) {交换机};
    \node[component] (link1) at (0,-1.1) {通信链路};
    
    % 资源子网
    \node[subnet] (resource) at (6,0) {};
    \node[above=0.1cm of resource.north] {\textbf{资源子网}};
    \node[component] (host1) at (6,0.5) {主机};
    \node[component] (terminal1) at (6,-0.3) {终端};
    \node[component] (software1) at (6,-1.1) {应用软件};
    
    % 连接线
    \draw[<->, thick, red] (comm.east) -- (resource.west) node[midway, above] {数据传输};
    
    % 功能说明
    \node[below=1cm of comm, text width=4cm, align=center] {\small\textbf{负责数据传输}\\提供通信服务\\就像快递公司};
    \node[below=1cm of resource, text width=4cm, align=center] {\small\textbf{负责数据处理}\\提供计算资源\\就像办公大楼};
\end{tikzpicture}
\end{center}

\subsubsection{生动比喻——网络就是"数字城市"}

把计算机网络想象成一个完整的现代化城市:

\begin{longtable}{|p{3cm}|p{5cm}|p{5cm}|}
\hline
\textbf{网络组件} & \textbf{城市对应物} & \textbf{功能说明} \\
\hline
\textbf{主机} & 居民楼、办公楼 & 提供居住和工作场所 \\
\hline
\textbf{路由器} & 交通信号灯 & 指挥交通流向 \\
\hline
\textbf{交换机} & 立交桥 & 连接不同道路 \\
\hline
\textbf{通信链路} & 道路系统 & 承载交通流量 \\
\hline
\textbf{协议} & 交通法规 & 规范行车秩序 \\
\hline
\textbf{数据包} & 汽车 & 承载乘客和货物 \\
\hline
\textbf{网络操作系统} & 城市管理部门 & 统一规划和管理 \\
\hline
\end{longtable}

\subsubsection{408考试重点提醒}

\begin{center}
\fcolorbox{red}{yellow!20}{
\begin{minipage}{0.9\textwidth}
\textbf{必考知识点:}
\begin{enumerate}
    \item 网络的\textbf{硬件组成}:主机、通信链路、交换设备
    \item 网络的\textbf{软件组成}:网络操作系统、协议、应用软件
    \item \textbf{通信子网 vs 资源子网}的区别和联系
    \item 各种网络设备的\textbf{工作层次}(物理层、数据链路层、网络层)
\end{enumerate}
\end{minipage}
}
\end{center}

\textbf{记忆口诀}:硬件软件两大类,主机链路加交换,系统协议和应用,通信资源分两边!

\subsection{计算机网络的功能}

如果说网络的组成告诉我们"网络是什么",那么网络的功能就告诉我们"网络能干什么"。计算机网络就像一个超级多功能工具箱,每个功能都能让我们的数字生活更加精彩!

\subsubsection{功能概览——网络的"十八般武艺"}

\begin{center}
\begin{tikzpicture}[node distance=2cm]
    % 定义样式
    \tikzset{
        function/.style={shape=ellipse, draw=blue!70, fill=blue!20, thick, minimum width=2.5cm, minimum height=1.2cm, align=center},
        core/.style={circle, draw=red!70, fill=red!30, thick, minimum size=2cm, align=center, font=\bfseries}
    }
    
    % 中心核心
    \node[core] (center) at (0,0) {计算机\\网络\\功能};
    
    % 六大功能环绕
    \node[function] (share) at (0,3) {资源\\共享};
    \node[function] (comm) at (2.6,1.5) {数据\\通信};
    \node[function] (dist) at (2.6,-1.5) {分布式\\处理};
    \node[function] (balance) at (0,-3) {负载\\均衡};
    \node[function] (reliability) at (-2.6,-1.5) {提高\\可靠性};
    \node[function] (cost) at (-2.6,1.5) {降低\\成本};
    
    % 连接线
    \draw[thick, blue!70] (center) -- (share);
    \draw[thick, blue!70] (center) -- (comm);
    \draw[thick, blue!70] (center) -- (dist);
    \draw[thick, blue!70] (center) -- (balance);
    \draw[thick, blue!70] (center) -- (reliability);
    \draw[thick, blue!70] (center) -- (cost);
\end{tikzpicture}
\end{center}

\subsubsection{1. 资源共享——网络的"共产主义理想"}

资源共享是计算机网络最重要的功能,就像室友之间共享WiFi密码一样自然!

\textbf{硬件资源共享}

\begin{center}
\begin{tikzpicture}[scale=0.9]
    % 定义样式
    \tikzset{
        computer/.style={rectangle, draw=blue!50, fill=blue!20, thick, minimum width=1.5cm, minimum height=1cm},
        device/.style={rectangle, draw=green!50, fill=green!20, thick, minimum width=1.5cm, minimum height=1cm},
        arrow/.style={->, >=stealth, thick}
    }
    
    % 计算机
    \node[computer] (pc1) at (0,2) {电脑A};
    \node[computer] (pc2) at (0,0) {电脑B};
    \node[computer] (pc3) at (0,-2) {电脑C};
    
    % 共享设备
    \node[device] (printer) at (4,1) {打印机};
    \node[device] (scanner) at (4,-1) {扫描仪};
    \node[device] (storage) at (6,0) {存储服务器};
    
    % 连接线
    \draw[arrow, red] (pc1) -- (printer);
    \draw[arrow, red] (pc2) -- (printer);
    \draw[arrow, red] (pc3) -- (printer);
    
    \draw[arrow, blue] (pc1) -- (scanner);
    \draw[arrow, blue] (pc2) -- (scanner);
    \draw[arrow, blue] (pc3) -- (scanner);
    
    \draw[arrow, green] (pc1) -- (storage);
    \draw[arrow, green] (pc2) -- (storage);
    \draw[arrow, green] (pc3) -- (storage);
    
    % 标签
    \node[above=0.3cm of printer] {\small\textbf{共享打印机}};
    \node[below=0.3cm of scanner] {\small\textbf{共享扫描仪}};
    \node[right=0.3cm of storage] {\small\textbf{共享存储}};
\end{tikzpicture}
\end{center}

\textbf{软件资源共享}

想象一下,如果每个人都要买正版Adobe全家桶,那得花多少钱?网络让我们可以共享软件许可:

\begin{itemize}
    \item \textbf{应用程序共享}:多用户共享同一个软件许可
    \item \textbf{数据库共享}:大家都能访问同一个数据库
    \item \textbf{计算资源共享}:借用别人的CPU和内存来加速计算
\end{itemize}

\textbf{信息资源共享}

这是最常见的共享方式,就像微信群里分享链接一样:

\begin{longtable}{|p{3cm}|p{5cm}|p{5cm}|}
\hline
\textbf{共享类型} & \textbf{实际例子} & \textbf{生活比喻} \\
\hline
\textbf{文件共享} & 
百度网盘、OneDrive & 
就像公共图书馆,大家都能借书 \\
\hline
\textbf{网页内容} & 
各种网站、博客 & 
就像公告板,大家都能看信息 \\
\hline
\textbf{数据库} & 
在线百科、论文库 & 
就像超大型档案室 \\
\hline
\textbf{多媒体} & 
YouTube、B站 & 
就像电影院,但是免费的 \\
\hline
\end{longtable}

\subsubsection{2. 数据通信——网络的"千里传音"}

数据通信让相距千里的计算机能够"聊天",就像古代的飞鸽传书,但速度是光速!

\begin{center}
\begin{tikzpicture}[node distance=3cm]
    % 发送方
    \node[computer] (sender) {发送方};
    \node[above=0.3cm of sender] {\textbf{小明的电脑}};
    
    % 接收方
    \node[computer, right=6cm of sender] (receiver) {接收方};
    \node[above=0.3cm of receiver] {\textbf{小红的电脑}};
    
    % 网络云
    \node[draw, ellipse, fill=gray!20, minimum width=3cm, minimum height=1.5cm] (network) at (3,0) {互联网};
    
    % 数据流
    \draw[arrow, red, thick] (sender) -- (network) node[midway, above] {\small 发送数据};
    \draw[arrow, blue, thick] (network) -- (receiver) node[midway, above] {\small 接收数据};
    
    % 数据包
    \node[draw, fill=yellow!30, rounded corners] at (1.5,-1) {\small 电子邮件};
    \node[draw, fill=yellow!30, rounded corners] at (3,-1.5) {\small 视频通话};
    \node[draw, fill=yellow!30, rounded corners] at (4.5,-1) {\small 文件传输};
\end{tikzpicture}
\end{center}

\textbf{数据通信的特点}:
\begin{itemize}
    \item \textbf{高速}:光纤网络速度可达Gbps级别(比火箭还快!)
    \item \textbf{可靠}:有错误检测和纠正机制(不会搞丢你的作业)
    \item \textbf{实时}:支持实时通信(视频聊天不会有时差)
    \item \textbf{多样}:支持文字、图片、音频、视频(什么都能传)
\end{itemize}

\subsubsection{3. 分布式处理——网络的"众人拾柴火焰高"}

分布式处理就像大家一起做大作业,每个人负责一部分,最后合成完整的作品。

\begin{center}
\begin{tikzpicture}[scale=0.8]
    % 定义样式
    \tikzset{
        task/.style={rectangle, draw=orange!70, fill=orange!20, thick, minimum width=2cm, minimum height=1cm, align=center},
        computer/.style={rectangle, draw=blue!50, fill=blue!20, thick, minimum width=1.5cm, minimum height=0.8cm, align=center}
    }
    
    % 主任务
    \node[task] (main) at (0,3) {大任务\\(渲染电影)};
    
    % 子任务
    \node[task] (sub1) at (-3,1) {子任务1\\(第1-100帧)};
    \node[task] (sub2) at (0,1) {子任务2\\(第101-200帧)};
    \node[task] (sub3) at (3,1) {子任务3\\(第201-300帧)};
    
    % 计算机
    \node[computer] (pc1) at (-3,-1) {电脑A};
    \node[computer] (pc2) at (0,-1) {电脑B};
    \node[computer] (pc3) at (3,-1) {电脑C};
    
    % 结果
    \node[task] (result) at (0,-3) {最终结果\\(完整电影)};
    
    % 连接线
    \draw[arrow] (main) -- (sub1);
    \draw[arrow] (main) -- (sub2);
    \draw[arrow] (main) -- (sub3);
    
    \draw[arrow] (sub1) -- (pc1);
    \draw[arrow] (sub2) -- (pc2);
    \draw[arrow] (sub3) -- (pc3);
    
    \draw[arrow] (pc1) -- (result);
    \draw[arrow] (pc2) -- (result);
    \draw[arrow] (pc3) -- (result);
    
    % 标签
    \node[left=0.5cm of main] {\textbf{任务分解}};
    \node[left=0.5cm of pc1] {\textbf{并行计算}};
    \node[left=0.5cm of result] {\textbf{结果合并}};
\end{tikzpicture}
\end{center}

\textbf{分布式处理的优势}:
\begin{itemize}
    \item \textbf{提高效率}:多台计算机同时工作,就像多人搬家比一人搬家快
    \item \textbf{容错能力}:一台机器坏了,其他机器继续工作
    \item \textbf{资源利用}:充分利用网络中闲置的计算资源
\end{itemize}

\subsubsection{4. 负载均衡——网络的"交通疏导"}

负载均衡就像交警指挥交通,让每条路都不会太拥堵。

\begin{center}
\begin{tikzpicture}[scale=0.9]
    % 用户请求
    \node[computer] (users) at (0,2) {大量用户};
    
    % 负载均衡器
    \node[draw, circle, fill=red!30, minimum size=1.5cm, align=center] (balancer) at (3,2) {负载\\均衡器};
    
    % 服务器
    \node[computer, align=center] (server1) at (6,3) {服务器1\\(30\%负载)};
    \node[computer, align=center] (server2) at (6,2) {服务器2\\(35\%负载)};
    \node[computer, align=center] (server3) at (6,1) {服务器3\\(35\%负载)};
    
    % 连接线
    \draw[arrow, thick] (users) -- (balancer);
    \draw[arrow] (balancer) -- (server1) node[midway, above] {\tiny 30\%};
    \draw[arrow] (balancer) -- (server2) node[midway, above] {\tiny 35\%};
    \draw[arrow] (balancer) -- (server3) node[midway, above] {\tiny 35\%};
    
    % 标签
    \node[below=0.3cm of balancer] {\small\textbf{智能分配}};
\end{tikzpicture}
\end{center}

\subsubsection{5. 提高可靠性——网络的"备胎机制"}

网络通过冗余设计提高可靠性,就像重要文件要备份一样。

\textbf{可靠性措施}:
\begin{itemize}
    \item \textbf{数据备份}:重要数据存多份(鸡蛋不放一个篮子里)
    \item \textbf{路径冗余}:多条路径传输数据(一条路堵了走另一条)
    \item \textbf{设备冗余}:关键设备有备用(主力坏了上替补)
    \item \textbf{容错机制}:自动检测和修复错误(自动纠错)
\end{itemize}

\subsubsection{6. 降低成本——网络的"省钱妙招"}

网络让我们能够以更低的成本获得更好的服务:

\begin{longtable}{|p{4cm}|p{4cm}|p{5cm}|}
\hline
\textbf{成本类型} & \textbf{传统方式} & \textbf{网络方式} \\
\hline
\textbf{硬件成本} & 每人买一台打印机 & 大家共享一台高级打印机 \\
\hline
\textbf{软件成本} & 每人买一份软件 & 网络版软件多人共享 \\
\hline
\textbf{维护成本} & 每台机器单独维护 & 集中管理,统一维护 \\
\hline
\textbf{通信成本} & 长途电话、传真 & 网络通信几乎免费 \\
\hline
\end{longtable}

\subsubsection{408考试重点提醒}

\begin{center}
\fcolorbox{red}{yellow!20}{
\begin{minipage}{0.9\textwidth}
\textbf{必考功能(按重要性排序):}
\begin{enumerate}
    \item \textbf{资源共享}:最基本、最重要的功能
    \item \textbf{数据通信}:网络存在的基础
    \item \textbf{分布式处理}:现代网络的重要应用
    \item \textbf{负载均衡}:提高系统性能
    \item \textbf{可靠性}:系统稳定运行的保证
    \item \textbf{成本效益}:网络应用的经济优势
\end{enumerate}
\end{minipage}
}
\end{center}

\textbf{记忆口诀}:共享通信分布式,均衡可靠降成本,六大功能要记牢,考试必定能过关!

\textbf{温馨提示}:网络功能虽然很多,但\textbf{资源共享}永远是最核心的!就像朋友的意义不在于数量,而在于愿意分享。

\section{计算机网络的分类}
计算机网络就像动物王国一样丰富多彩,有各种各样的"物种"。不同的分类方法就像不同的观察角度,让我们能够更好地理解网络这个大家族!
\subsection{按覆盖范围分类}

这是最常见的分类方法,就像按照活动范围给动物分类:家养的、野生的、海洋的...

\subsubsection{个人区域网络(PAN - Personal Area Network)}

\textbf{覆盖范围}:10米以内,就是你伸手够得着的范围

\begin{center}
\begin{tikzpicture}[scale=0.8]
    % 定义样式
    \tikzset{
        device/.style={rectangle, draw=blue!50, fill=blue!20, thick, minimum width=1.2cm, minimum height=0.8cm, rounded corners},
        person/.style={circle, draw=green!50, fill=green!20, thick, minimum size=1.5cm}
    }
    
    % 人物
    \node[person] (person) at (0,0) {小明};
    
    % PAN设备
    \node[device] (phone) at (1.5,1) {手机};
    \node[device] (watch) at (-1.5,1) {智能手表};
    \node[device] (earphone) at (1.5,-1) {蓝牙耳机};
    \node[device] (laptop) at (-1.5,-1) {笔记本};
    
    % 连接线
    \draw[dashed, blue] (person) circle (2.5cm);
    \draw[dotted] (person) -- (phone);
    \draw[dotted] (person) -- (watch);
    \draw[dotted] (person) -- (earphone);
    \draw[dotted] (person) -- (laptop);
    
    % 标签
    \node[below=3cm of person] {\textbf{PAN范围:10米内}};
\end{tikzpicture}
\end{center}

\textbf{典型技术}:蓝牙、红外线
\textbf{生活例子}:手机连蓝牙耳机、智能手表同步手机数据

\subsubsection{局域网(LAN - Local Area Network)}

\textbf{覆盖范围}:几百米到几公里,比如一栋楼、一个校园

\begin{center}
\begin{tikzpicture}[scale=0.9]
    % 建筑物
    \draw[thick] (0,0) rectangle (6,3);
    \node[above] at (3,3) {\textbf{学校/公司大楼}};
    
    % 楼层和设备
    \node[device] (pc1) at (1,2.5) {PC1};
    \node[device] (pc2) at (2,2.5) {PC2};
    \node[device] (pc3) at (4,2.5) {PC3};
    \node[device] (server) at (5,2.5) {服务器};
    
    \node[device] (pc4) at (1,1.5) {PC4};
    \node[device] (pc5) at (2,1.5) {PC5};
    \node[device] (printer) at (4,1.5) {打印机};
    \node[device] (pc6) at (5,1.5) {PC6};
    
    % 交换机
    \node[draw, circle, fill=red!30] (switch) at (3,0.5) {交换机};
    
    % 连接线
    \draw (pc1) -- (switch);
    \draw (pc2) -- (switch);
    \draw (pc3) -- (switch);
    \draw (server) -- (switch);
    \draw (pc4) -- (switch);
    \draw (pc5) -- (switch);
    \draw (printer) -- (switch);
    \draw (pc6) -- (switch);
    
    % 范围标示
    \node[below=0.5cm of switch] {\textbf{LAN范围:几百米到几公里}};
\end{tikzpicture}
\end{center}

\textbf{特点}:
\begin{itemize}
    \item \textbf{高速}:通常100Mbps-10Gbps
    \item \textbf{低延迟}:响应快,适合实时应用
    \item \textbf{私有}:通常由单一组织管理
    \item \textbf{可靠}:错误率低,性能稳定
\end{itemize}

\textbf{典型技术}:以太网、WiFi

\subsubsection{城域网(MAN - Metropolitan Area Network)}

\textbf{覆盖范围}:几十公里,比如一个城市

\begin{center}
\begin{tikzpicture}[scale=0.7]
    % 城市轮廓
    \draw[thick] (0,0) circle (4cm);
    \node at (0,4.2) {\textbf{城市范围}};
    
    % 不同区域
    \node[draw, fill=blue!20, minimum width=1.5cm, minimum height=1cm] (area1) at (-2,2) {商业区};
    \node[draw, fill=green!20, minimum width=1.5cm, minimum height=1cm] (area2) at (2,2) {住宅区};
    \node[draw, fill=yellow!20, minimum width=1.5cm, minimum height=1cm] (area3) at (-2,-2) {工业区};
    \node[draw, fill=red!20, minimum width=1.5cm, minimum height=1cm] (area4) at (2,-2) {政府区};
    
    % 中心节点
    \node[draw, circle, fill=orange!30, minimum size=1cm] (center) at (0,0) {中心节点};
    
    % 连接线
    \draw[thick] (area1) -- (center);
    \draw[thick] (area2) -- (center);
    \draw[thick] (area4) -- (center);
    
    \node at (0,-4.5) {\textbf{MAN范围:几十公里(一个城市)}};
\end{tikzpicture}
\end{center}

\textbf{典型应用}:城市光纤网络、有线电视网络
\textbf{生活例子}:城市的宽带网络、数字电视网络

\subsubsection{广域网(WAN - Wide Area Network)}

\textbf{覆盖范围}:几百公里到全球,跨越国家和大洲

\begin{center}
\begin{tikzpicture}[scale=0.8]
    % 地球
    \draw[thick] (0,0) circle (3cm);
    \node at (0,3.3) {\textbf{全球范围}};
    
    % 大洲
    \node[draw, fill=blue!20, ellipse] (asia) at (-1,1) {亚洲};
    \node[draw, fill=green!20, ellipse] (europe) at (1,1.5) {欧洲};
    \node[draw, fill=yellow!20, ellipse] (america) at (-1.5,-1) {美洲};
    \node[draw, fill=red!20, ellipse] (africa) at (1,-1.5) {非洲};
    
    % 连接线(海底光缆)
    \draw[thick, blue, dashed] (asia) to[bend left=20] (europe);
    \draw[thick, blue, dashed] (asia) to[bend right=30] (america);
    \draw[thick, blue, dashed] (europe) to[bend left=20] (america);
    \draw[thick, blue, dashed] (europe) to[bend right=20] (africa);
    
    % 卫星
    \node[draw, star, fill=orange!30] (satellite) at (0,4) {卫星};
    \draw[dotted] (satellite) -- (europe);
    
    \node at (0,-3.5) {\textbf{WAN范围:全球(互联网就是最大的WAN)}};
\end{tikzpicture}
\end{center}
\textbf{特点}:
\begin{itemize}
    \item \textbf{覆盖面广}:跨越地理边界
    \item \textbf{传输延迟大}:距离远导致延迟增加
    \item \textbf{带宽有限}:相对于LAN带宽较小
    \item \textbf{复杂性高}:涉及多个运营商和技术
\end{itemize}

\textbf{典型技术}:光纤、卫星、微波

\subsubsection{网络范围对比表}

\begin{longtable}{|p{2cm}|p{2cm}|p{3cm}|p{3cm}|p{3cm}|}
\hline
\textbf{网络类型} & \textbf{覆盖范围} & \textbf{典型技术} & \textbf{传输速度} & \textbf{生活例子} \\
\hline
\textbf{PAN} & 10米内 & 蓝牙、红外 & 几Mbps & 手机连耳机 \\
\hline
\textbf{LAN} & 几公里 & 以太网、WiFi & 100Mbps-10Gbps & 校园网、家庭网 \\
\hline
\textbf{MAN} & 几十公里 & 光纤、有线电视 & 几十Mbps-几Gbps & 城市宽带网 \\
\hline
\textbf{WAN} & 全球 & 光纤、卫星 & 几Mbps-几Gbps & 互联网 \\
\hline
\end{longtable}

\subsection{按拓扑结构分类}

网络拓扑就像房子的建筑结构,决定了网络的"长相"和性能特点。

\subsubsection{总线型拓扑}

就像公交车站,大家都在同一条线上排队:

\begin{center}
\begin{tikzpicture}[scale=0.9]
    % 总线
    \draw[thick, blue] (0,0) -- (8,0);
    \node[below=0.3cm] at (4,0) {\textbf{总线(公共传输介质)}};
    
    % 节点
    \node[computer] (pc1) at (1,1) {PC1};
    \node[computer] (pc2) at (3,1) {PC2};
    \node[computer] (pc3) at (5,1) {PC3};
    \node[computer] (pc4) at (7,1) {PC4};
    
    % 连接线
    \draw (pc1) -- (1,0);
    \draw (pc2) -- (3,0);
    \draw (pc3) -- (5,0);
    \draw (pc4) -- (7,0);
    
    % 终端电阻
    \node[draw, circle, fill=red!30, scale=0.7] at (0,0) {T};
    \node[draw, circle, fill=red!30, scale=0.7] at (8,0) {T};
    \node[below=0.5cm] at (0,0) {\tiny 终端器};
    \node[below=0.5cm] at (8,0) {\tiny 终端器};
\end{tikzpicture}
\end{center}

\textbf{优点}:
\begin{itemize}
    \item \textbf{简单经济}:布线简单,成本低(就像坐公交便宜)
    \item \textbf{易于扩展}:添加节点容易
\end{itemize}

\textbf{缺点}:
\begin{itemize}
    \item \textbf{可靠性差}:总线断了全网瘫痪(公交车坏了大家都走不了)
    \item \textbf{性能下降}:节点多了会冲突增加
    \item \textbf{故障定位难}:不知道哪里出了问题
\end{itemize}

\subsubsection{星型拓扑}

就像太阳系,所有行星围绕太阳转:

\begin{center}
\begin{tikzpicture}[scale=0.8]
    % 中心节点
    \node[draw, circle, fill=yellow!30, minimum size=1.5cm] (center) at (0,0) {交换机(中心)};
    
    % 周围节点
    \node[computer] (pc1) at (0,2.5) {PC1};
    \node[computer] (pc2) at (2.2,1.5) {PC2};
    \node[computer] (pc3) at (2.2,-1.5) {PC3};
    \node[computer] (pc4) at (0,-2.5) {PC4};
    \node[computer] (pc5) at (-2.2,-1.5) {PC5};
    \node[computer] (pc6) at (-2.2,1.5) {PC6};
    
    % 连接线
    \draw[thick] (center) -- (pc1);
    \draw[thick] (center) -- (pc2);
    \draw[thick] (center) -- (pc3);
    \draw[thick] (center) -- (pc4);
    \draw[thick] (center) -- (pc5);
    \draw[thick] (center) -- (pc6);
\end{tikzpicture}
\end{center}

\textbf{优点}:
\begin{itemize}
    \item \textbf{可靠性高}:单个节点故障不影响其他节点
    \item \textbf{易于管理}:集中控制,故障定位容易
    \item \textbf{性能好}:每个节点独享带宽
\end{itemize}

\textbf{缺点}:
\begin{itemize}
    \item \textbf{中心依赖}:中心节点坏了全网瘫痪(太阳没了行星就乱套了)
    \item \textbf{布线成本高}:需要大量线缆
\end{itemize}

\subsubsection{环型拓扑}

就像接龙游戏,大家手拉手围成圈:

\begin{center}
\begin{tikzpicture}[scale=0.8]
    % 环形连接
    \node[computer] (pc1) at (0,2) {PC1};
    \node[computer] (pc2) at (2,1) {PC2};
    \node[computer] (pc3) at (2,-1) {PC3};
    \node[computer] (pc4) at (0,-2) {PC4};
    \node[computer] (pc5) at (-2,-1) {PC5};
    \node[computer] (pc6) at (-2,1) {PC6};
    
    % 连接线(环形)
    \draw[thick, ->] (pc1) to[bend left=15] (pc2);
    \draw[thick, ->] (pc2) to[bend left=15] (pc3);
    \draw[thick, ->] (pc3) to[bend left=15] (pc4);
    \draw[thick, ->] (pc4) to[bend left=15] (pc5);
    \draw[thick, ->] (pc5) to[bend left=15] (pc6);
    \draw[thick, ->] (pc6) to[bend left=15] (pc1);
    
    % 数据流向标示
    \node[draw, fill=green!30, circle, scale=0.7] at (1,1.5) {数据};
    \node[above=0.3cm] at (1,1.5) {\tiny 令牌};
\end{tikzpicture}
\end{center}

\textbf{优点}:
\begin{itemize}
    \item \textbf{无冲突}:令牌传递机制避免冲突
    \item \textbf{公平访问}:每个节点都有平等机会
\end{itemize}

\textbf{缺点}:
\begin{itemize}
    \item \textbf{可靠性差}:一个节点坏了整个环就断了
    \item \textbf{延迟大}:数据要绕一圈才能到达
\end{itemize}

\subsubsection{网状拓扑}

就像蜘蛛网,节点之间互相连接:

\begin{center}
\begin{tikzpicture}[scale=0.8]
    % 节点
    \node[computer] (pc1) at (0,2) {PC1};
    \node[computer] (pc2) at (2,1) {PC2};
    \node[computer] (pc3) at (2,-1) {PC3};
    \node[computer] (pc4) at (0,-2) {PC4};
    \node[computer] (pc5) at (-2,-1) {PC5};
    \node[computer] (pc6) at (-2,1) {PC6};
    
    % 网状连接
    \draw[thick] (pc1) -- (pc2);
    \draw[thick] (pc1) -- (pc6);
    \draw[thick] (pc1) -- (pc4);
    \draw[thick] (pc2) -- (pc3);
    \draw[thick] (pc2) -- (pc6);
    \draw[thick] (pc3) -- (pc4);
    \draw[thick] (pc3) -- (pc5);
    \draw[thick] (pc4) -- (pc5);
    \draw[thick] (pc5) -- (pc6);
    \draw[thick] (pc1) -- (pc3);
    \draw[thick] (pc2) -- (pc4);
    \draw[thick] (pc6) -- (pc3);
\end{tikzpicture}
\end{center}

\textbf{优点}:
\begin{itemize}
    \item \textbf{可靠性极高}:多条路径,一条断了还有其他路
    \item \textbf{负载分散}:流量可以分散到多条路径
\end{itemize}

\textbf{缺点}:
\begin{itemize}
    \item \textbf{成本极高}:需要大量连接线路(就像每家都要修路到每家)
    \item \textbf{管理复杂}:路径选择和管理复杂
\end{itemize}

\subsection{按传输介质分类}

\subsubsection{有线网络}

就像自来水管,通过物理线缆传输数据:

\begin{itemize}
    \item \textbf{双绞线网络}:最常见,像电话线的升级版
    \item \textbf{同轴电缆网络}:像有线电视线
    \item \textbf{光纤网络}:用光传输,速度最快
\end{itemize}

\subsubsection{无线网络}

就像广播电台,通过空气传播信号:

\begin{itemize}
    \item \textbf{WiFi网络}:家庭和办公室常用
    \item \textbf{移动网络}:手机上网用的3G/4G/5G
    \item \textbf{卫星网络}:偏远地区的救星
\end{itemize}

\subsection{按通信方式分类}

\subsubsection{点对点网络(Point-to-Point)}

就像打电话,两个人直接对话:

\begin{center}
\begin{tikzpicture}
    \node[computer] (pc1) at (0,0) {电脑A};
    \node[computer] (pc2) at (4,0) {电脑B};
    \draw[thick, <->] (pc1) -- (pc2) node[midway, above] {专用连接};
\end{tikzpicture}
\end{center}

\subsubsection{多点网络(Multi-point)}

就像开会,多个人在一个房间里讨论:

\begin{center}
\begin{tikzpicture}
    % 共享介质
    \draw[thick, blue] (0,0) -- (6,0);
    
    % 多个节点
    \node[computer] (pc1) at (1,1) {PC1};
    \node[computer] (pc2) at (3,1) {PC2};
    \node[computer] (pc3) at (5,1) {PC3};
    
    \draw (pc1) -- (1,0);
    \draw (pc2) -- (3,0);
    \draw (pc3) -- (5,0);
    
    \node[below] at (3,0) {共享传输介质};
\end{tikzpicture}
\end{center}

\subsubsection{408考试重点提醒}
\begin{center}
\fcolorbox{red}{yellow!20}{
\begin{minipage}{0.9\textwidth}
\textbf{必考分类重点:}
\begin{enumerate}
    \item \textbf{按覆盖范围}:PAN < LAN < MAN < WAN(记住范围大小)
    \item \textbf{按拓扑结构}:星型最常用,总线型最经济,网状型最可靠
    \item \textbf{LAN的特点}:高速、低延迟、私有管理
    \item \textbf{WAN的特点}:覆盖广、延迟大、带宽有限
\end{enumerate}
\end{minipage}
}
\end{center}
% \end{minipage}
\textbf{记忆口诀}:个局城广按范围,总星环网看拓扑,有线无线看介质,点对多点看通信!

\section{计算机网络的标准化工作及相关组织}

网络世界要想“大家说同一种话”,必须靠标准来统一。\textbf{标准化的目标}是:互联互通、兼容性、可扩展、可维护,避免“你说你的,我说我的”。本节聚焦四大组织:ISO、ITU、IEEE、IETF——它们就像四位“网络立法者”,各管一摊,又相互协同。

\subsection{ISO国际标准化组织}
	\textbf{定位与职责}:非政府国际组织,覆盖工业与技术的广泛标准。\textbf{在网络领域的核心贡献是OSI参考模型与相关服务接口标准}。

	\textbf{代表性标准}
\begin{longtable}{|p{4cm}|p{8cm}|}
\hline
	\textbf{标准/系列} & \textbf{作用说明} \\
\hline
ISO 10646(Unicode相关) & 字符集标准,为跨语言信息交换打基础。 \\
\hline
\end{longtable}

	\textbf{一眼记忆}:\textbf{OSI模型之父},强调“\textbf{分层与接口}”。

\subsection{ITU国际电信联盟}
	\textbf{定位与职责}:联合国专门机构,负责全球电信/无线电频谱/卫星轨道协调与\textbf{公用电信网}相关标准。内部主要分为\textbf{ITU-T}(电信标准化)与\textbf{ITU-R}(无线电通信)。

	\textbf{代表性标准(多以字母系列命名)}
\begin{longtable}{|p{3.5cm}|p{8.5cm}|}
\hline
	\textbf{系列/建议} & \textbf{作用说明} \\
\hline
X系列(数据通信/网络) & 如\textbf{X.25}分组交换、X.500目录服务;为早期广域网、目录系统提供规范。 \\
\hline
G系列(传输系统) & 光纤传输与编码,如\textbf{G.709 OTN}承载网框架,定义光传送网层次与帧结构。 \\
\hline
H系列(多媒体) & \textbf{H.264/AVC、H.265/HEVC}等视频编码标准,影响流媒体与视频会议。 \\
\hline
T系列(传真/终端) & 早期传真等终端通信协议。 \\
\hline
\end{longtable}

	\textbf{一眼记忆}:\textbf{管“电信与线路承载”},从频谱到光纤再到编解码。

\subsection{IEEE电气电子工程师协会}
	\textbf{定位与职责}:学术性专业组织,制定大量\textbf{局域网与链路层}相关标准,最有影响力的是\textbf{IEEE 802系列}。

	\textbf{代表性标准}
\begin{longtable}{|p{4cm}|p{8cm}|}
\hline
	\textbf{标准} & \textbf{作用说明} \\
\hline
IEEE 802.3(以太网) & 规定有线以太网的\textbf{物理层与MAC子层},从10Mb/s到\textbf{100GbE/400GbE}不断演进(\textbf{考试高频})。 \\
\hline
IEEE 802.11(WLAN) & 无线局域网标准(Wi-Fi),含a/b/g/n/ac/ax演进,定义信道、调制、MAC。 \\
\hline
IEEE 802.1(桥接与VLAN) & 包括\textbf{802.1Q VLAN}、生成树协议STP/RSTP/MSTP等二层交换关键技术。 \\
\hline
IEEE 802.15(WPAN) & 个人区域网,如蓝牙(与SIG协作)、ZigBee等低功耗短距网络。 \\
\hline
\end{longtable}

	\textbf{一眼记忆}:\textbf{管“局域网二层技术”},以太网与Wi-Fi是其“门面担当”。

\subsection{IETF互联网工程任务组}
	\textbf{定位与职责}:开放性、以工程实践为导向的\textbf{互联网协议标准组织},标准文档称为\textbf{RFC}。核心流程:Internet-Draft → RFC(分信息性、实验性、最佳现行实践BCP、标准轨道STD等)。

	\textbf{代表性协议(RFC)}
\begin{longtable}{|p{4cm}|p{8cm}|}
\hline
	\textbf{协议/系列} & \textbf{作用说明} \\
\hline
	\textbf{TCP/IP(IPv4/IPv6、TCP、UDP)} & 互联网的基础协议族,涵盖寻址、路由、传输(\textbf{考试顶流})。 \\
\hline
DNS(RFC 1034/1035 及后续) & 分布式命名系统,定义名称解析与消息格式。 \\
\hline
HTTP/1.1、HTTP/2、HTTP/3 & 万维网应用协议,HTTP/3基于QUIC(UDP之上)改善时延与拥塞表现。 \\
\hline
TLS(原SSL演进) & 传输层安全协议,保障Web/邮件等的加密与认证。 \\
\hline
\end{longtable}

	\textbf{一眼记忆}:\textbf{管“互联网与上层协议”},RFC是“活文档”,强调可部署性与互操作。

\subsection{四大组织“谁管什么”——一图速记}

\begin{center}
\begin{tikzpicture}[node distance=2.2cm]
        % 样式
            \tikzset{ org/.style={rectangle, rounded corners, draw=blue!60, fill=blue!15, thick, minimum width=3.5cm, minimum height=1cm, align=center},
                             area/.style={rectangle, draw=gray!60, dashed, minimum width=4.2cm, minimum height=3.6cm, align=center},
                             arrow/.style={->, >=stealth, thick} }

        % 四象限区域
        \node[area, label=above:{\small 物理/承载层面}] (phy) at (-4,0) {};

        \node[area, label=above:{\small 链路/局域网}] (link) at (0,0) {};

        \node[area, label=above:{\small 网络/传输/应用}] (net)  at (4,0) {};

        \node at (-4,1.5) {\small 频谱/光纤/编解码};
        \node at (0,1.5) {\small 以太网/Wi-Fi/VLAN};
        \node at (4,1.5) {\small IP/TCP/DNS/HTTP};

        % 组织节点
        \node[org] (itu) at (-4,0) {ITU-T/ITU-R\\电信与承载};

        \node[org] (ieee) at (0,0) {IEEE 802\\LAN/WLAN/VLAN};

        \node[org] (ietf) at (4,0) {IETF\\互联网RFC};
        
        \node[org] (iso) at (0,-3) {ISO\\OSI模型/通用标准};

        % 关系箭头
        \draw[arrow] (iso) -- (ieee);
        \draw[arrow] (iso) -- (itu);
        \draw[arrow] (iso) -- (ietf);
\end{tikzpicture}
\end{center}

\subsection{408考试高频对比与易混点}

\begin{longtable}{|p{3cm}|p{4cm}|p{7cm}|}
\hline
	\textbf{组织} & \textbf{关键词} & \textbf{易考点/易混淆} \\
\hline
ISO & OSI七层、接口 & \textbf{OSI是框架/模型,不是具体协议};考“层与层的功能界限”。 \\
\hline
ITU-T & X/G/H系列、公网承载 & 常把\textbf{X.25}与IP混淆;G.709与以太网的关系是承载与被承载。 \\
\hline
IEEE & 802.3/802.11/802.1Q & \textbf{以太网=802.3、Wi-Fi=802.11、VLAN=802.1Q}要对号入座。 \\
\hline
IETF & RFC、互联网协议 & RFC编号繁多,记住\textbf{TCP/IP、DNS、HTTP、TLS}是高频。 \\
\hline
\end{longtable}

\begin{center}
\fcolorbox{red}{yellow!20}{
\begin{minipage}{0.92\textwidth}
	\textbf{必背三句话:}
\begin{enumerate}
    \item \textbf{ISO——提模型(OSI)};\textbf{IEEE——定二层(802.x)};\textbf{IETF——管互联网(RFC)};\textbf{ITU——管电信承载(X/G/H)}。
    \item \textbf{OSI是理论框架,TCP/IP是事实标准}(考试常考对比题)。
    \item \textbf{802.3=以太网,802.11=Wi‑Fi,802.1Q=VLAN}(名称与编号要精准匹配)。
\end{enumerate}
\end{minipage}}
\end{center}

	\textbf{记忆口诀}:\textbf{国(际)标找ISO,电信承载看ITU,局域二层问IEEE,互联网协议找IETF!}

\section{计算机网络体系结构与参考模型}

\subsection{网络体系结构的基本概念}

	\textbf{网络体系结构}是对网络功能进行分层、定义层间\textbf{服务}与\textbf{接口}、规定同层\textbf{协议}的一套蓝图。它回答三个问题:\textbf{分几层?每层干啥?层与层怎么合作?}

	\textbf{分层思想的好处}
\begin{itemize}
    \item \textbf{模块化}:各层相对独立,便于开发、测试与替换。
    \item \textbf{标准化}:清晰的接口(SAP/服务访问点),降低系统耦合。
    \item \textbf{互操作}:不同厂商实现可“对上对下”对接。
    \item \textbf{演进性}:单层升级不牵一发动全身。
\end{itemize}

	\textbf{术语小结}(考试常考):

\begin{itemize}
    \item \textbf{服务}:下层向上层提供“能做什么”。
    \item \textbf{接口}:上层调用下层服务的入口(SAP)。
    \item \textbf{协议}:同层实体间“怎么做”的约定(语法/语义/时序)。
    \item \textbf{PDU命名}:物理层比特、链路层\textbf{帧}、网络层\textbf{分组/数据报}、传输层\textbf{段/报文段}、应用层\textbf{报文}。
\end{itemize}

\begin{center}
\begin{tikzpicture}[node distance=0.1cm]
        % 左右两端主机分层示意
			\tikzset{layer/.style={rectangle, draw=blue!50, fill=blue!15, rounded corners, minimum width=2.8cm, minimum height=0.55cm, align=center, font=\scriptsize}}
    % 左侧
    \node[layer] (la) {应用层};
    \node[layer, below=of la] (lt) {传输层};
    \node[layer, below=of lt] (ln) {网络层};
    \node[layer, below=of ln] (lld) {数据链路层};
    \node[layer, below=of lld] (lp) {物理层};
    % 右侧
    \node[layer, right=5.2cm of la] (ra) {应用层};
    \node[layer, below=of ra] (rt) {传输层};
    \node[layer, below=of rt] (rn) {网络层};
    \node[layer, below=of rn] (rld) {数据链路层};
    \node[layer, below=of rld] (rp) {物理层};
    % 同层“虚通道”
    \draw[->, >=Stealth] (la.east) -- (ra.west);
    \draw[->, >=Stealth] (lt.east) -- (rt.west);
    \draw[->, >=Stealth] (ln.east) -- (rn.west);
    \draw[->, >=Stealth] (lld.east) -- (rld.west);
    \draw[->, >=Stealth] (lp.east) -- (rp.west);
    % 标注
    \node[below=0.1cm of lp] {发送方};
    \node[below=0.1cm of rp] {接收方};
\end{tikzpicture}
\end{center}

\subsection{OSI七层参考模型}

	\textbf{七层职责速记}:应(应用)表(表示)会(会话)传(传输)网(网络)数(数据链路)物(物理)。

\begin{longtable}{|p{2.6cm}|p{6.2cm}|p{2.3cm}|p{3.2cm}|}
\hline
	\textbf{层次} & \textbf{核心功能} & \textbf{PDU} & \textbf{典型要点} \\
\hline
应用层 & 应用服务(HTTP/FTP/SMTP/…) & 报文 & Web/邮件/文件传输 \\
\hline
表示层 & 数据表示、加密压缩、语法语义转换 & 报文 & SSL/TLS常被视作跨层(历史争议) \\
\hline
会话层 & 建立/管理/终止会话、同步、检查点 & 报文 & 检查点/恢复、全双工/半双工 \\
\hline
传输层 & 端到端可靠/不可靠传输、复用分用、流量/差错控制 & 段 & TCP、UDP \\
\hline
网络层 & 逻辑寻址与路由、分片与转发 & 分组/数据报 & IPv4/IPv6、路由协议作用在此层次之上 \\
\hline
数据链路层 & 成帧、差错检测、介质访问控制 & 帧 & 以太网、PPP、802.11、VLAN \\
\hline
物理层 & 比特传输、接口与电气特性 & 比特 & 电压/光信号、带宽、编码 \\
\hline
\end{longtable}

\subsection{TCP/IP四层模型}

\begin{itemize}
    \item \textbf{应用层}:囊括OSI的应用/表示/会话(三层合一)。
    \item \textbf{传输层}:TCP、UDP。
    \item \textbf{网际层}:IP、ICMP、ARP/ND、路由。
    \item \textbf{网络接口层}:对应OSI的数据链路+物理。
\end{itemize}

\begin{center}
\begin{tikzpicture}
        \tikzset{blk/.style={rectangle, draw=green!60, fill=green!15, minimum width=3.2cm, minimum height=0.6cm, align=center, font=\scriptsize}}
    \node[blk] (a) at (0,2) {应用层\newline(HTTP/DNS/SMTP…)};
    \node[blk] (t) at (0,1.2) {传输层\newline(TCP/UDP)};
    \node[blk] (n) at (0,0.4) {网际层\newline(IP/ICMP/ARP/ND)};
    \node[blk] (l) at (0,-0.4) {网络接口层\newline(以太网/802.11)};
    \draw[->] (a) -- (t);
    \draw[->] (t) -- (n);
    \draw[->] (n) -- (l);
\end{tikzpicture}
\end{center}

	\textbf{对比要点}:OSI重\textbf{理论分层与接口};TCP/IP重\textbf{实用可部署},\textbf{事实标准}地位明确。

\subsection{五层参考模型}

教学与考试常用的“\textbf{五层模型}”介于两者之间:\textbf{应用/传输/网络/数据链路/物理}。它保留了链路与物理的区分,便于分别讨论成帧与比特传输。

\begin{longtable}{|p{3cm}|p{10.5cm}|}
\hline
	\textbf{层次} & \textbf{典型协议/功能} \\
\hline
应用层 & HTTP、DNS、SMTP/POP3/IMAP、FTP、DHCP、SSH、NTP \\
\hline
传输层 & TCP(可靠、面向连接)、UDP(简单、开销低) \\
\hline
网络层 & IPv4/IPv6、ICMP、路由协议运行其上(RIP/OSPF/BGP) \\
\hline
数据链路层 & 以太网(802.3)、无线LAN(802.11)、PPP、VLAN(802.1Q) \\
\hline
物理层 & 双绞线、同轴、光纤;编码/调制;速率/带宽/误码率 \\
\hline
\end{longtable}

\subsection{各层协议与功能}

	\textbf{封装/解封装示意(考试常画)}

\begin{center}
\begin{tikzpicture}
        % 封装/解封装示意(从上到下逐层封装)
        \tikzset{
            box/.style={rectangle, rounded corners=1pt, minimum height=0.62cm, font=\scriptsize, align=center, inner xsep=4pt, draw},
            pay/.style={box, draw=gray!60, fill=gray!10},
            thd/.style={box, draw=blue!60, fill=blue!15},
            nhd/.style={box, draw=orange!70, fill=orange!20},
            lhd/.style={box, draw=purple!60, fill=purple!15},
            captext/.style={font=\scriptsize, text=gray!70}
        }

        % 应用层:仅有应用数据
        \node[pay] (app) at (0,0) {应用数据(报文)};
        \node[captext, right=8pt of app] {应用层数据};

        % 传输层:TCP/UDP首部 + 应用数据
        \node[thd, below=1.2cm of app] (t_hdr) {TCP/UDP首部};
        \node[pay, right=0pt of t_hdr] (t_pay) {应用数据};
        \node[captext, right=8pt of t_pay] {传输层PDU:段};

        % 网络层:IP首部 + TCP/UDP首部 + 应用数据
        \node[nhd, below=1.2cm of t_hdr] (n_hdr) {IP首部};
        \node[thd, right=0pt of n_hdr] (n_th) {TCP/UDP首部};
        \node[pay, right=0pt of n_th] (n_pay) {应用数据};
        \node[captext, right=8pt of n_pay] {网络层PDU:数据报};

        % 数据链路层:以太网首部 + IP首部 + TCP/UDP首部 + 应用数据 + FCS
        \node[lhd, below=1.2cm of n_hdr] (l_hdr) {以太网首部};
        \node[nhd, right=0pt of l_hdr] (l_ip) {IP首部};
        \node[thd, right=0pt of l_ip] (l_th) {TCP/UDP首部};
        \node[pay, right=0pt of l_th] (l_pay) {应用数据};
        \node[lhd, right=0pt of l_pay] (l_fcs) {FCS/帧校验序列};
        \node[captext, right=8pt of l_fcs] {数据链路层PDU:帧};

        % 封装箭头(自上而下)
        \draw[->, >=Stealth, thick, gray!70] (app.south) -- node[midway, right, font=\scriptsize] {封装} (t_hdr.north);
        \draw[->, >=Stealth, thick, gray!70] ([xshift=0.5cm]t_pay.south) -- node[midway, right, font=\scriptsize] {封装} ([xshift=0.5cm]n_hdr.north);
        \draw[->, >=Stealth, thick, gray!70] ([xshift=0.8cm]n_pay.south) -- node[midway, right, font=\scriptsize] {封装} ([xshift=0.8cm]l_hdr.north);

        % 物理层说明
        \node[captext, below=1.0cm of l_th] {物理层:比特流传输};
\end{tikzpicture}
\end{center}

	\textbf{地址/标识与设备分层(必背)}

\begin{longtable}{|p{3.2cm}|p{5.6cm}|p{4.7cm}|}
\hline
	\textbf{层次} & \textbf{主要地址/标识} & \textbf{典型设备} \\
\hline
应用层 & URL/域名、应用端口(80/443/53…) & 服务器软件/代理/负载均衡(逻辑) \\
\hline
传输层 & 端口号(TCP/UDP) & 四层负载均衡/防火墙策略 \\
\hline
网络层 & IP地址(IPv4/IPv6) & \textbf{路由器}、三层交换机 \\
\hline
数据链路层 & \textbf{MAC地址}、VLAN ID & \textbf{交换机}、网桥、AP \\
\hline
物理层 & 物理接口/光模块/速率 & 中继器、集线器、线缆/光纤 \\
\hline
\end{longtable}
\begin{center}
\fcolorbox{red}{yellow!20}{\begin{minipage}{0.92\textwidth}
	\textbf{408必考要点与易错警示:}
\begin{enumerate}
    \item \textbf{OSI vs TCP/IP 对应关系}:应/表/会≈应用;链路+物理≈网络接口。
    \item \textbf{PDU命名与层次设备}:段-包/数据报-帧;\textbf{集线器L1,交换机L2,路由器L3}。
    \item \textbf{地址三件套}:MAC在二层,IP在三层,端口在四层(\textbf{三层三地址要对齐})。
    \item \textbf{封装方向}:发送端自上而下逐层加首部;接收端自下而上剥离首部。
    \item \textbf{事实标准}:\textbf{TCP/IP}为事实标准;\textbf{OSI}是理论模型(常考辨析)。
\end{enumerate}
\end{minipage}}
\end{center}

	\textbf{记忆口诀}:\textbf{上三为应表会,中层传网做转发;下二链物管介质,三址对应不混搭。}

% ========== 第2章 物理层 ==========
\chapter{物理层}

\section{通信基础}
\subsection{数据通信的基本概念}
数据通信就是把“信息”可靠地从甲地搬到乙地。为此,我们需要把数据变成可在物理介质上传输的“信号”,穿过有噪声、有衰减的“信道”,再在对端还原出来。

\begin{center}
\resizebox{\linewidth}{!}{%
\begin{tikzpicture}[node distance=0.9cm]
    	\tikzset{
        proc/.style={rectangle, draw=blue!60, fill=blue!10, rounded corners, minimum width=2.3cm, minimum height=0.8cm, align=center},
        chan/.style={rectangle, draw=gray!60, fill=gray!10, minimum width=3.4cm, minimum height=0.8cm, align=center},
        n/.style={align=center, font=\small},
        arr/.style={->, >=Stealth, thick}
    }
    % blocks
    \node[proc] (src) {源/数据源\\(Data)};
    \node[proc, right=1.4cm of src] (enc) {发送端\\编码/调制};
    \node[chan, right=1.4cm of enc] (chl) {信道/介质\\(噪声/衰减/失真)};
    \node[proc, right=1.4cm of chl] (dec) {接收端\\解调/译码};
    \node[proc, right=1.4cm of dec] (dst) {宿/接收方\\(Data)};
    % arrows
    \draw[arr] (src) -- (enc);
    \draw[arr] (enc) -- node[n, above]{信号} (chl);
    \draw[arr] (chl) -- (dec);
    \draw[arr] (dec) -- (dst);
\end{tikzpicture}}%
\end{center}

\paragraph{核心名词一次搞定}
{\small
\begin{longtable}{|p{0.25\linewidth}|p{0.70\linewidth}|}
\hline
	\textbf{术语} & \textbf{要点解释(配考试高频点)} \\
\hline
数据 (Data) & 有意义的符号集合(文本、数值、图片、音频等)本身不等于可传输的电磁波。\\
\hline
信号 (Signal) & 用物理量表示数据。\textbf{模拟信号}连续取值,\textbf{数字信号}离散取值。考试常问“模拟/数字”的差别与转换位置。\\
\hline
码元 (Symbol) & 在单位时间内传送的\textbf{基本离散信号波形},每个码元可承载多比特:$M$ 元码元可承载 $\log_2 M$ 比特。\\
\hline
比特率 (bps) & 每秒传送的\textbf{比特数},记作 $R_b$,单位 bit/s。\\
\hline
波特率 (Baud) & 每秒传送的\textbf{码元数},记作 $R_s$,单位 baud。\textbf{关系式}:$R_b = R_s\cdot \log_2 M$。常考与比特率的区别。\\
\hline
频带/带宽 (Hz) & 物理意义的“带宽”= 可用频率范围宽度,单位 Hz。注意与网络口语里的“带宽=可用速率(bps)”\textbf{不是一个量}(易混)。\\
\hline
基带/带通信 & \textbf{基带}:数字基带信号直接在介质上传输(如以太网双绞线)。\textbf{带通}:把基带调制到载波后在\textbf{通带}上传输(如无线/光纤长距)。\\
\hline
信道 & 信号通过的物理路径,受\textbf{噪声/衰减/失真/串扰}影响。相关指标:SNR、BER、带宽、时延。\\
\hline
误码率 (BER) & 错误比特数/总比特数。受调制方式、信道噪声、编码增益影响。\\
\hline
信噪比 (SNR) & $\mathrm{SNR}=\dfrac{S}{N}$;$\mathrm{SNR_{dB}}=10\log_{10}\!\left(\dfrac{S}{N}\right)$。\textbf{3 dB≈功率×2,10 dB≈×10}(速算常考)。\\
\hline
\end{longtable}
}

\paragraph{三组高频易混淆}
{\small
\begin{longtable}{|p{0.30\linewidth}|p{0.36\linewidth}|p{0.30\linewidth}|}
\hline
	\textbf{名词} & \textbf{定义} & \textbf{记忆/考点} \\
\hline
比特率 $R_b$ vs 波特率 $R_s$ & $R_b$ 是\textbf{比特/秒};$R_s$ 是\textbf{码元/秒} & $R_b = R_s\cdot \log_2 M$;当 $M=2$ 时二元码元,$R_b=R_s$。\\
\hline
带宽(Hz) vs “带宽”(bps) & 物理带宽=频率范围;网络口语带宽=传输速率 & 考题常设陷阱,问“带宽增加一定能否提升速率?”需结合奈奎斯特/香农条件。\\
\hline
基带传输 vs 带通传输 & 基带:不调载波;带通:经调制在通带上传输 & 有线短距多用基带(以太网),无线/长距多用带通。\\
\hline
\end{longtable}
}

\paragraph{传输方式与工作方式}
\begin{itemize}
    \item \textbf{按方向性}:\textbf{单工}(A→B)、\textbf{半双工}(A↔B但同一时刻单向)、\textbf{全双工}(A↔B同时双向)。
    \item \textbf{按并发度}:\textbf{串行}(逐位)vs \textbf{并行}(多位同时,短距如主板总线)。
    \item \textbf{按定时}:\textbf{异步}(字符为单位,起止位,简单/开销大/容错强)vs \textbf{同步}(帧为单位,共享时钟,效率高)。
\end{itemize}

\begin{center}
\resizebox{\linewidth}{!}{%
\begin{tikzpicture}[node distance=0.9cm]
    	\tikzset{end/.style={rectangle, draw=blue!60, fill=blue!10, rounded corners, minimum width=1.8cm, minimum height=0.7cm, align=center}, arr/.style={->, >=Stealth, thick}}
    % simplex
    \node[end] (sa) {A};
    \node[end, right=1.4cm of sa] (sb) {B};
    \draw[arr] (sa) -- (sb);
    \node[below=0.1cm of sb] {单工};
    % half-duplex
    \node[end, right=2.4cm of sb] (ha) {A};
    \node[end, right=1.4cm of ha] (hb) {B};
    \draw[arr] (ha) -- (hb);
    \draw[arr, bend right=30] (hb.west) to (ha.east);
    \node[below=0.1cm of hb] {半双工};
    % full-duplex
    \node[end, right=2.4cm of hb] (fa) {A};
    \node[end, right=1.4cm of fa] (fb) {B};
    \draw[arr] (fa) -- (fb);
    \draw[arr] (fb) -- (fa);
    \node[below=0.1cm of fb] {全双工};
\end{tikzpicture}}%
\end{center}

\paragraph{信道损伤的三大来源}
\begin{itemize}
    \item \textbf{衰减 Attenuation}:随距离/频率增大而增大,常用 dB 表示。$A_{\mathrm{dB}}=10\log_{10}(P_\text{入}/P_\text{出})$,\textbf{可叠加}(方便链路预算)。
    \item \textbf{噪声 Noise}:热噪声、串扰、脉冲噪声等,直接降低 SNR,\textbf{提高 BER}。
    \item \textbf{失真 Distortion}:多径、频率选择性衰落导致波形展宽、码间串扰(ISI)。
\end{itemize}

\paragraph{小结与前导}
\begin{itemize}
    \item 通信性能受\textbf{频带(Hz)}、\textbf{SNR}、\textbf{码间串扰}共同制约。下一小节将分开讨论信号表示与传输特性;随后用\textbf{奈奎斯特定理}(无噪声限速)与\textbf{香农定理}(有噪声容量)给出\textbf{理论上限}。
    \item \textbf{牢记三个公式}:$R_b = R_s\log_2 M$;$\mathrm{SNR_{dB}}=10\log_{10}(S/N)$;dB 可\textbf{直接相加}(功率比相乘)。
\end{itemize}

\begin{center}
{\setlength{\fboxsep}{6pt}%
\fcolorbox{red}{yellow!15}{\begin{minipage}{0.90\textwidth}
	\textbf{408考试提示}
\begin{itemize}
    \item 概念选择题高频:\textbf{比特率/波特率/码元/带宽(Hz) vs 带宽(bps)/基带与带通/异步与同步}。
    \item 速算题:$M$ 元码、已知波特率求比特率;dB 级联增益/损耗;给定 SNR 估容量(香农)或无噪声限速(奈奎斯特)。
    \item 易错点:把“带宽(Hz)”当“速率(bps)”;把“波特=比特/秒”混淆在 $M\neq2$ 的情形。
\end{itemize}
\end{minipage}}}%
\end{center}

\subsection{信号的传输}
信号在信道中传输会受到带宽限制、噪声、频率选择性衰减与时延失真的共同影响。理解\textbf{时域—频域}两个视角,有助于掌握“为什么方波变圆了、为什么会码间串扰(ISI)”。

\paragraph{时域/频域双视角}
\begin{itemize}
    \item \textbf{时域}:观测 $s(t)$ 的波形。带宽受限会“抹平”陡峭边沿,脉冲展宽导致相邻码元互相影响(ISI)。
    \item \textbf{频域}:用\textbf{傅里叶展开}把信号分解为不同频率的正弦叠加;信道像一个\textbf{滤波器}(低通LPF/带通BPF),削减通带外的频率分量。
\end{itemize}

\begin{center}
\resizebox{\linewidth}{!}{%
\begin{tikzpicture}[node distance=1.2cm]
    	\tikzset{
        blk/.style={rectangle, draw=blue!60, fill=blue!10, rounded corners, minimum width=2.2cm, minimum height=0.8cm, align=center},
        arr/.style={->, >=Stealth, thick}, lab/.style={font=\small, align=center}
    }
    % blocks
    \node[blk] (bits) {比特序列};
    \node[blk, right=1.6cm of bits] (wave) {基带波形 $s(t)$};
    \node[blk, right=1.8cm of wave] (chn) {信道\\(LPF/BPF, 噪声)};
    \node[blk, right=1.8cm of chn] (recv) {接收波形 $r(t)$};
    % arrows
    \draw[arr] (bits) -- (wave);
    \draw[arr] (wave) -- node[lab, above]{带宽受限} (chn);
    \draw[arr] (chn) -- node[lab, above]{失真+噪声} (recv);
\end{tikzpicture}}%
\end{center}

\paragraph{三类主要传输损伤}
{\small\begin{longtable}{|p{0.22\linewidth}|p{0.36\linewidth}|p{0.36\linewidth}|}
\hline
    \textbf{类型} & \textbf{现象/原因} & \textbf{影响与缓解} \\
\hline
衰减 (Attenuation) & 功率随距离/频率下降;介质损耗 & 降低幅度与SNR;用中继/放大、选更优介质;用 dB 做链路预算。\\
\hline
噪声 (Noise) & 热噪声、串扰、脉冲噪声 & 提高误码率BER;采用\textbf{编码增益}、滤波、屏蔽与合理布线。\\
\hline
失真 (Distortion) & 频率选择性衰减、群时延不平坦 & 波形展宽、\textbf{码间串扰ISI};均衡、成形滤波、提高采样判决裕度(眼图张开)。\\
\hline
\end{longtable}}

\paragraph{衰减与分贝(dB)快记}
\begin{itemize}
    \item 功率比与分贝:$G_{\mathrm{dB}}=10\log_{10}\!\left(\dfrac{P_\text{出}}{P_\text{入}}\right)$;损耗为负值。\textbf{可直接相加}表示级联段总增益/损耗。
    \item 近似换算:3 dB≈功率×2,10 dB≈×10;-3 dB≈×1/2。
\end{itemize}

\paragraph{低通信道与带通信道}
\begin{itemize}
    \item \textbf{低通}:包含直流分量,适合\textbf{基带传输}(如以太网双绞线的短距场景)。
    \item \textbf{带通}:只允许一段频带通过,需\textbf{调制到载波}后传输(无线、电缆长距、光纤等)。
\end{itemize}

\begin{center}
\resizebox{0.9\linewidth}{!}{%
\begin{tikzpicture}[x=1cm,y=1cm]
    	\tikzset{axis/.style={->,>=Stealth,thick}, lab/.style={font=\small}}
    % Low-pass schematic
    \draw[axis] (0,0) -- (5.2,0) node[right]{f};
    \draw[axis] (0,0) -- (0,2.0) node[above]{|H(f)|};
    \draw[blue,thick,fill=blue!10] (0,0) -- (0,1.6) -- (2.2,1.6) -- (2.6,0) -- cycle;
    \node[lab] at (1.5,1.85){低通(基带)};
    % Band-pass schematic (shifted)
    \begin{scope}[xshift=7cm]
        \draw[axis] (0,0) -- (5.2,0) node[right]{f};
        \draw[axis] (0,0) -- (0,2.0) node[above]{|H(f)|};
        \draw[blue,thick,fill=blue!10] (1.4,0) -- (2.0,1.5) -- (3.2,1.5) -- (3.8,0) -- cycle;
        \node[lab] at (2.6,1.85){带通(调制)};
    \end{scope}
\end{tikzpicture}}%
\end{center}

\paragraph{ISI 直观理解}
理想矩形码元含有高频分量,受限带宽后边沿变缓、脉冲展宽,超出一个码元时间进入相邻时隙,造成\textbf{码间串扰}。解决思路:\textbf{成形滤波}(受控带宽内成形,例如升余弦)、均衡补偿信道、提高采样判决裕度。

\begin{center}
\resizebox{\linewidth}{!}{%
\begin{tikzpicture}[x=0.9cm,y=0.9cm]
    	\tikzset{axis/.style={->,>=Stealth,thin}, lab/.style={font=\small}}
    % Ideal pulses
    \draw[axis] (0,0) -- (9,0) node[right]{t};
    \draw[axis] (0,0) -- (0,2.2) node[above]{幅度};
    \draw[red,thick] (0,0) -- (0,2) -- (2,2) -- (2,0) -- (4,0) -- (4,2) -- (6,2) -- (6,0) -- (8,0);
    \node[lab,red] at (4,2.3){理想矩形码元};
    % Received (smeared) pulses
    \begin{scope}[yshift=-3.2cm]
        \draw[axis] (0,0) -- (9,0) node[right]{t};
        \draw[axis] (0,0) -- (0,2.2) node[above]{幅度};
        \draw[blue,thick,smooth] plot coordinates {(0,0) (0.4,0.8) (0.8,1.5) (1.6,1.9) (2.2,1.4) (2.6,0.7) (3.2,0.2) (3.8,0.1) (4.2,0.8) (4.6,1.5) (5.4,1.9) (6.0,1.5) (6.6,0.7) (7.4,0.2) (8,0.1)};
        \node[lab,blue] at (4,2.3){受限带宽后(产生ISI)};
    \end{scope}
\end{tikzpicture}}%
\end{center}

\begin{center}
{\setlength{\fboxsep}{6pt}%
\fcolorbox{red}{yellow!15}{\begin{minipage}{0.90\textwidth}
    \textbf{408考试提示}
\begin{itemize}
    \item 概念辨析:\textbf{低通=可基带;带通=需调制};\textbf{时域失真}$\leftrightarrow$\textbf{频域带宽受限}。
    \item 计算速记:dB 可\textbf{相加};3 dB≈2 倍功率,-3 dB≈1/2;利用 SNR 与下一节容量/限速公式联动。
    \item 现象判断:边沿被抹平、多径展宽→\textbf{ISI};对策:成形、均衡、判决时钟优化。
\end{itemize}
\end{minipage}}}%
\end{center}
\subsection{信道容量}
    \textbf{信道容量(Channel Capacity)}指在给定\textbf{带宽 $B$} 与\textbf{噪声条件(SNR)}下,理论上\textbf{可无差错传输}的\textbf{最高信息速率}(bit/s)。容量是\textbf{上限}:实际可达速率$\le$容量。

\paragraph{两个核心结论(先记结论,下一小节推导)}

{\small\begin{longtable}{|p{0.28\linewidth}|p{0.66\linewidth}|}
\hline
	\textbf{情形} & \textbf{容量/限速表达式与要点} \\
\hline
无噪声、理想低通 & 奈奎斯特限速:$R_b\le R_s\log_2 M$ 且 $R_s\le 2B$,故 $\boxed{R_b\le 2B\,\log_2 M}$;二元($M=2$)时 $R_b\le 2B$。\\
\hline
有噪声(高斯白噪声) & 香农容量:$\boxed{C=B\,\log_2\!\bigl(1+\mathrm{SNR}\bigr)}$。对数增长:提高 SNR 收益\textbf{递减},而增大带宽是\textbf{线性}增长。\\
\hline
\end{longtable}}

\paragraph{单位与换算}
\begin{itemize}
    \item 带宽 $B$ 用 Hz;容量/速率用 bit/s(kbps=$10^3$,Mbps=$10^6$)。
    \item $\mathrm{SNR}_{\mathrm{linear}}=\dfrac{S}{N}$;$\mathrm{SNR}_{\mathrm{dB}}=10\log_{10}(S/N)$;二者换算:$\mathrm{SNR}_{\mathrm{linear}}=10^{\mathrm{SNR}_{\mathrm{dB}}/10}$。
    \item 3 dB≈功率×2(SNR 乘 2),10 dB≈×10。香农公式中代入\textbf{线性} SNR。
\end{itemize}

\begin{center}
\resizebox{0.9\linewidth}{!}{%
\begin{tikzpicture}[x=1cm,y=1cm]
    	\tikzset{axis/.style={->,>=Stealth,thick}, lab/.style={font=\small}}
    % C vs B (SNR fixed)
    \draw[axis] (0,0) -- (6.2,0) node[right]{B};
    \draw[axis] (0,0) -- (0,3.2) node[above]{C};
    \draw[blue,thick] (0,0) -- (5.8,2.6);
    \node[lab,blue] at (4.8,2.9){SNR固定: $C\propto B$};
    % C vs SNR (B fixed)
    \begin{scope}[xshift=8cm]
        \draw[axis] (0,0) -- (6.2,0) node[right]{SNR(dB)};
        \draw[axis] (0,0) -- (0,3.2) node[above]{C};
        \draw[red,thick,smooth] plot coordinates {(0,0.4) (1,0.9) (2,1.35) (3,1.7) (4,1.95) (5,2.15) (6,2.3)};
        \node[lab,red] at (3.8,2.9){B固定: C 对数增长};
    \end{scope}
\end{tikzpicture}}%
\end{center}

\paragraph{速算例题}
\begin{enumerate}[label=例\arabic*.]
    \item 已知 $B=1\,\mathrm{MHz}$,$\mathrm{SNR}=20\,\mathrm{dB}$。求香农容量。
  
    $\mathrm{SNR}_{\mathrm{linear}}=10^{20/10}=100$,$C=B\log_2(1+100)\approx 10^6\times 6.658=6.66\,\mathrm{Mbps}$。
  
    \item 已知电话信道 $B=3\,\mathrm{kHz}$,二元传输($M=2$)。若无噪声,奈奎斯特限速 $R_b\le 2B=6\,\mathrm{kbps}$。若 $\mathrm{SNR}=30\,\mathrm{dB}$,香农容量 $C=3000\log_2(1+1000)\approx 29.9\,\mathrm{kbps}$。
  
    \item 若希望 $C\ge 10\,\mathrm{Mbps}$,在 $B=2\,\mathrm{MHz}$ 下需多大 SNR?
  
    $10\times10^6=2\times10^6\log_2(1+\mathrm{SNR})\Rightarrow \log_2(1+\mathrm{SNR})=5$,$\mathrm{SNR}=31\Rightarrow \mathrm{SNR}_{\mathrm{dB}}\approx 10\log_{10}31\approx 14.9\,\mathrm{dB}$。
\end{enumerate}

\begin{center}
{\setlength{\fboxsep}{6pt}%
\fcolorbox{red}{yellow!15}{\begin{minipage}{0.90\textwidth}
	\textbf{408考试提示}
\begin{itemize}
    \item \textbf{先分情形}:无噪声看\textbf{奈奎斯特}($R_s\le 2B$,$R_b=R_s\log_2M$);有噪声看\textbf{香农容量}($C=B\log_2(1+\mathrm{SNR})$)。
    \item \textbf{换算谨慎}:SNR 若给 dB,进香农公式前先转\textbf{线性值};结果单位留意 bit/s→kbps/Mbps。
    \item \textbf{增长对比}:增大 B 线性提速;提高 SNR 为\textbf{对数收益},题目常让你判断“该加带宽还是加功率”。
    \item 本小节给出结论与速算,详细推导放到“奈奎斯特定理和香农定理”。
\end{itemize}
\end{minipage}}}%
\end{center}
\subsection{奈奎斯特定理和香农定理}
本小节给出两个“上限”的来龙去脉:无噪声条件下的\textbf{奈奎斯特限速}与有噪声下的\textbf{香农容量},并指出常见混淆点。

\paragraph{奈奎斯特(无噪声、零ISI 条件)}
\begin{itemize}
    \item 对\textbf{理想低通信道}(带宽 $B$ Hz),若采用满足\textbf{奈奎斯特零ISI准则}的成形(理想 sinc 或升余弦族),可实现\textbf{符号率} $R_s\le 2B$(采样定理同源:带限 $B$ 的信号以 $2B$ 采样可复原)。
    \item 若每个码元承载 $\log_2 M$ bit,则\textbf{比特率} $\boxed{R_b\le 2B\,\log_2 M}$;$M=2$ 时 $R_b\le 2B$。
    \item 实际常用\textbf{升余弦成形},带宽为 $B=\dfrac{1+\beta}{2}\,R_s$(滚降系数 $\beta\in[0,1]$),于是 $R_s\le \dfrac{2B}{1+\beta}$(拓展了解,考试以理想式为主)。
\end{itemize}
\begin{center}
\resizebox{0.86\linewidth}{!}{%
\begin{tikzpicture}[x=1cm,y=1cm]
        \tikzset{axis/.style={->,>=Stealth,thick}, lab/.style={font=\small}}
    % bandwidth vs symbol rate schematic
    \draw[axis] (0,0) -- (6.2,0) node[right]{B(Hz)};
    \draw[axis] (0,0) -- (0,3.0) node[above]{$R_s(\mathrm{Symbol/s})$};
    \draw[blue,thick] (0,0) -- (5.6,2.8);
    \node[lab,blue] at (4.6,2.6){理想:$R_s=2B$};
    % beta effect
    \draw[red,thick] (0,0) -- (5.6,1.9);
    \node[lab,red] at (4.5,1.7){升余弦:$R_s=\dfrac{2B}{1+\beta}$($\beta>0$)};
\end{tikzpicture}}%
\end{center}

\paragraph{香农(有噪声、高斯白噪声信道)}
\begin{itemize}
    \item \textbf{香农容量}:$\boxed{C=B\,\log_2\!\bigl(1+\mathrm{SNR}\bigr)}$,为\textbf{无差错通信可达的极限}(允许使用任意长的纠错码)。
    \item \textbf{谱效率} $\eta=\dfrac{C}{B}=\log_2(1+\mathrm{SNR})$(bit/s/Hz)。低 SNR 下 $\eta\approx \dfrac{\mathrm{SNR}}{\ln 2}$;高 SNR 下近似 $\eta\approx \log_2(\mathrm{SNR})$(对数增长)。
    \item 启示:堆功率(提 SNR)收益递减;\textbf{加带宽}是线性提升,但受物理/法规限制。
\end{itemize}
\paragraph{二者关系与易混点}
{\small\begin{longtable}{|p{0.25\linewidth}|p{0.32\linewidth}|p{0.32\linewidth}|}
\hline
    \textbf{维度} & \textbf{奈奎斯特(无噪声)} & \textbf{香农(有噪声)} \\
\hline
适用 & 理想低通、零ISI可达 & 高斯白噪声信道容量上限 \\
适用 & 理想低通、零ISI可达 & 高斯白噪声信道容量上限 \\
\hline
表达 & $R_b\le 2B\,\log_2M$ & $C=B\log_2(1+\mathrm{SNR})$ \\
\hline
含噪声? & 不考虑噪声/误码概率 & 显式考虑噪声(线性 SNR) \\
\hline
设计含义 & 决定\textbf{码元率与成形带宽}的下限关系 & 决定\textbf{可达速率}与\textbf{编码增益}的上限 \\
\hline
常见混淆 & 把“奈奎斯特”当采样公式或把 $2B$ 当容量 & 用 dB 值直接代 SNR(漏了线性化) \\
\hline
\end{longtable}}

\paragraph{小例题}
\begin{enumerate}[label=例\arabic*.]
    \item $B=200\,\mathrm{kHz}$,理想无噪声,$M=8$(每码元 3 bit)。求最大比特率。
  
    $R_b\le 2B\,\log_2M=2\times 2\times 10^5\times 3=1.2\,\mathrm{Mbps}$。
  
    \item 目标谱效率 $\eta=4\,\mathrm{bit/s/Hz}$,问最低 SNR(dB)?
  
    $\eta=\log_2(1+\mathrm{SNR})=4\Rightarrow \mathrm{SNR}=2^4-1=15$,$\mathrm{SNR}_{\mathrm{dB}}\approx 10\log_{10}15\approx 11.76\,\mathrm{dB}$。
  
    \item $B=3\,\mathrm{MHz}$、$\mathrm{SNR}=18\,\mathrm{dB}$。香农容量?若理想无噪声且 $M=16$ 时奈奎斯特限速?
  
    $\mathrm{SNR}=10^{1.8}\approx 63.1$,$C=3\times10^6\log_2(1+63.1)\approx 3\times10^6\times 6.01=18.0\,\mathrm{Mbps}$;奈奎斯特:$R_b\le 2B\log_2M=2\times 3\times10^6\times 4=24\,\mathrm{Mbps}$(但实际有噪声时达不到)。
\end{enumerate}
\begin{center}
{\setlength{\fboxsep}{6pt}%
\fcolorbox{red}{yellow!15}{\begin{minipage}{0.90\textwidth}
    \textbf{408考试提示}
\begin{itemize}
    \item \textbf{三步走}:先分情形(无/有噪声)→ 统一单位(Hz、bit/s)→ dB 先转线性再代公式。
    \item \textbf{谱效率快算}:$\eta=\log_2(1+\mathrm{SNR})$;给 $\eta$ 求 SNR 用 $\mathrm{SNR}=2^{\eta}-1$,再转 dB。
    \item \textbf{常设陷阱}:把 $2B$ 当容量;把 dB 直接代入香农;忘记 $M$ 元导致 $\log_2M$。
    \item \textbf{工程启示}:若容量受限且 SNR 提升代价大,优先\textbf{拓展带宽/提高频谱利用}或\textbf{采用信道编码}提升靠近香农极限的效率。
\end{itemize}
\end{minipage}}}%
\end{center}

\section{传输介质}
\subsection{有线传输介质}
\subsubsection{双绞线}
双绞线由两根绝缘铜线按一定\textbf{绞距}缠绕构成,利用\textbf{自互相抵消}降低电磁辐射与串扰。分为\textbf{非屏蔽}(UTP)与\textbf{屏蔽}(STP/FTP)。以太网常见 RJ-45 接头,支持 PoE 供电。
\\
{\small\begin{longtable}{|p{0.22\linewidth}|p{0.22\linewidth}|p{0.22\linewidth}|p{0.22\linewidth}|}
\hline
    \textbf{类别} & \textbf{额定带宽(信号)} & \textbf{典型速率/以太网} & \textbf{典型距离} \\
\hline
Cat5e & \~100 MHz & 1GBASE-T(1 Gbps) & 100 m(水平链路) \\
\hline
Cat6 & 250 MHz & 10GBASE-T 可达\~55 m;1 Gbps 100 m & 55/100 m \\
\hline
Cat6a & 500 MHz & 10GBASE-T(10 Gbps) & 100 m \\
\hline
Cat7/7a & $\geq 600/1000$ MHz(屏蔽) & 10G/万兆以上(视标准) & 100 m \\
\hline
Cat8 & $\leq 2000$ MHz(屏蔽) & 25G/40GBASE-T(数据中心) & 30 m \\
\hline
\end{longtable}}

要点小结:\textbf{成本低、施工方便},短距接入场景通用;高频衰减与串扰限制了距离与速率;\textbf{屏蔽}可改善抗干扰但施工与接地要求更高。
\subsubsection{同轴电缆}
同轴由\textbf{内导体—绝缘层—屏蔽层—外护套}构成,\textbf{50\,Ω}多用于数据/无线电,\textbf{75\,Ω}多用于 CATV。屏蔽效果好,频段高、距离较 UTP 更长。
\\
{\small\begin{longtable}{|p{0.28\linewidth}|p{0.28\linewidth}|p{0.36\linewidth}|}
\hline
    \textbf{类型/阻抗} & \textbf{典型应用} & \textbf{特点} \\
\hline
50\,Ω(RG-58/213 等) & 旧以太网 10BASE2/5、射频馈线 & 驻波小、功率承载较高,历史以太网 185 m/500 m;现多用于射频连接。\\
\hline
75\,Ω(RG-6 等) & 有线电视、Cable Modem(DOCSIS) & 衰减低、视频/宽带接入;分支/连接器质量影响回传与上行。\\
\hline
\end{longtable}}

要点小结:\textbf{抗干扰强、带宽较高},但较粗硬、施工成本高;在 LAN 中已被双绞线/光纤替代,接入网/射频仍常见。
\subsubsection{光纤}
光纤以光脉冲在玻璃/塑料纤芯中反射/折射传输,\textbf{带宽极高、损耗极低、抗电磁干扰强}。分\textbf{单模(SMF)}与\textbf{多模(MMF)}。
\\
{\small\begin{longtable}{|p{0.20\linewidth}|p{0.28\linewidth}|p{0.22\linewidth}|p{0.22\linewidth}|}
\hline
    \textbf{类型} & \textbf{参数} & \textbf{典型距离} & \textbf{典型应用} \\
\hline
多模 MMF(OM3/OM4) & 芯径 50/62.5\,µm;850/1300\,nm & 百米到数百米(10G 常见 300\textasciitilde 550 m) & 机房/园区短距互联 \\
\hline
单模 SMF & 芯径 \~9\,µm;1310/1550\,nm & 十公里到百公里(中继/放大可更远) & 城域/骨干/长距 \\
\hline
\end{longtable}}

要点小结:速率从千兆到 100G/400G 乃至更高;\textbf{WDM}(CWDM/DWDM)提升频谱利用;\textbf{缺点}是成本、熔接/端接工艺要求与部署难度较高。
\subsection{无线传输介质}
\subsubsection{无线电波}
无线电波常用于\textbf{Wi-Fi(2.4/5/6\,GHz)}、蜂窝、蓝牙等,\textbf{绕射/反射}能力较强,穿透墙体能力优于更高频的微波/毫米波,但易受同频干扰。
\subsubsection{微波}
微波(SHF/毫米波段)频率高、带宽大,适合\textbf{点对点回传/卫星通信}。\textbf{视距(LOS)}要求强,对遮挡/雨衰敏感;定向天线增益高、链路预算严格。
\subsubsection{红外线}
红外线波长更短,\textbf{无法穿墙}、多用于室内\textbf{短距直视}通信或遥控,抗电磁干扰强,但受遮挡/阳光影响大。

\begin{center}
\resizebox{0.95\linewidth}{!}{%
\begin{tikzpicture}[x=1cm,y=1cm]
    	\tikzset{axis/.style={->,>=Stealth,thick}, band/.style={draw,rounded corners,fill=blue!10}, lab/.style={font=\small}}
    % frequency axis (rough, illustrative)
    \draw[axis] (0,0) -- (14,0) node[right]{频率(对数示意)};
    \node[lab] at (1,0.5){300 MHz};
    \node[lab] at (4,0.5){2.4 GHz};
    \node[lab] at (6,0.5){5 GHz};
    \node[lab] at (7.2,0.5){6 GHz};
    \node[lab] at (10,0.5){60 GHz};
    \node[lab] at (13,0.5){300 GHz};
    % bands
    \draw[band] (0.5,0.9) rectangle (5.0,1.6); \node[lab] at (2.8,1.25){无线电波(ISM/蜂窝)};
    \draw[band,fill=green!10] (5.0,1.9) rectangle (11.5,2.6); \node[lab] at (8.4,2.35){微波/毫米波(定向, LOS)};
    \draw[band,fill=red!10] (11.5,2.9) rectangle (13.8,3.6); \node[lab] at (12.7,3.35){红外(短距)};
    % notes
    \node[lab] at (4.0,-0.6){2.4/5/6 GHz Wi-Fi;蓝牙/物联多在 2.4 GHz};
    \node[lab] at (9.8,-0.6){60 GHz 高频率—带宽大但穿透弱};
\end{tikzpicture}}%
\end{center}

{\small\begin{longtable}{|p{0.22\linewidth}|p{0.30\linewidth}|p{0.38\linewidth}|}
\hline
    \textbf{介质} & \textbf{优势} & \textbf{局限/应用} \\
\hline
无线电波 & 穿透较好、部署便捷、终端普及 & 易受同频干扰、共享介质碰撞;常见于 Wi-Fi/蜂窝/蓝牙 \\
\hline
微波/毫米波 & 频谱宽、容量大、定向强、保密性好 & 视距要求、对遮挡/雨衰敏感;用于微波回传、固定无线接入、卫星 \\
\hline
红外 & 抗电磁干扰、短距直视安全性高 & 无法穿墙、易受强光影响;用于遥控/室内短距链路 \\
\hline
\end{longtable}}

\paragraph{综合对比(选型速查)}
{\small\begin{longtable}{|p{0.14\linewidth}|p{0.18\linewidth}|p{0.18\linewidth}|p{0.14\linewidth}|p{0.14\linewidth}|p{0.14\linewidth}|}
\hline
    \textbf{介质} & \textbf{带宽级别} & \textbf{典型速率} & \textbf{典型距离} & \textbf{抗干扰} & \textbf{成本} \\
\hline
双绞线 & 中 & 1G/10G & 100 m & 中 & 低 \\
\hline
同轴 & 中-高 & 10M/百兆/接入宽带 & 100 m-公里 & 高 & 中 \\
\hline
光纤 & 极高 & 10G/40G/100G+ & 公里-百公里 & 极高 & 中-高 \\
\hline
无线电波 & 中 & 十兆-千兆(视制式) & 室内/小区级 & 低-中 & 低 \\
\hline
微波/毫米波 & 高 & 百兆-多千兆 & 公里级(视LOS) & 中-高 & 中 \\
\hline
红外 & 低-中 & 兆级-十兆 & 室内短距 & 高(EMI) & 低 \\
\hline
\end{longtable}}

\begin{center}
{\setlength{\fboxsep}{6pt}%
\fcolorbox{red}{yellow!15}{\begin{minipage}{0.90\textwidth}
    \textbf{408考试提示}
\begin{itemize}
    \item 识记型:\textbf{UTP vs STP}、\textbf{Cat 类别与速率/距离}、\textbf{SMF vs MMF}(芯径/波长/距离)。
    \item 判断型:场景选型——\textbf{长距/抗干扰}优先光纤;\textbf{短距/低成本}优先双绞线;\textbf{无布线}用无线。
    \item 易错点:把“同轴/光纤带宽高就一定低成本”混淆;忽视\textbf{屏蔽接地}对 STP 的要求;SNR 与容量关系需回连香农公式。
\end{itemize}
\end{minipage}}}%
\end{center}

\section{传输方式}
\subsection{串行传输和并行传输}
串行与并行是按位并发度对数据在物理链路上的组织方式划分:
\begin{itemize}
    \item \textbf{串行传输}:在\textbf{一对导线/一条链路}上\textbf{逐位}依次发送,比特沿时间轴排队。典型实现配合\textbf{编码+时钟恢复}(如8b/10b、64b/66b等)与\textbf{差分传输},以获得\textbf{高抗干扰、远距离、超高速}能力(Gbps+)。
    \item \textbf{并行传输}:通过\textbf{多条数据线}同时并发发送\textbf{多个比特}(如8位/16位/32位总线),常配合一根或多根\textbf{时钟线}。受\textbf{时钟偏斜(clock skew)}和\textbf{串扰}影响,\textbf{距离短、频率受限},适合板级/背板/机内短距高并发。
\end{itemize}

\paragraph{直观对照图}
\begin{center}
\resizebox{0.95\linewidth}{!}{%
\begin{tikzpicture}[node distance=1.8cm]
    % styles
    	\tikzset{dev/.style={rectangle,draw=blue!60,fill=blue!10,rounded corners,minimum width=2.6cm,minimum height=0.9cm,align=center},
             blk/.style={rectangle,draw=gray!70,fill=gray!10,rounded corners,minimum width=2.3cm,minimum height=0.9cm,align=center},
             arr/.style={->,>=Stealth,thick}, lab/.style={font=\small}, bus/.style={-Latex,thick,blue!70} }

    % Parallel block (left)
    \node[dev] (cpu) {主机/控制器};
    \node[dev, right=5.5cm of cpu] (dev) {设备/存储器};
    % draw 8-bit bus lines
    \foreach \i [count=\k from 0] in {0,...,7} {
        \draw[bus] ([yshift=\i*0.18cm]cpu.east) -- ([yshift=\i*0.18cm]dev.west);
    }
    % clock line
    \draw[bus,red!70,dashed] ([yshift=-0.5cm]cpu.east) -- node[lab,fill=white,inner sep=1pt] {CLK} ([yshift=-0.5cm]dev.west);
    % labels
    \node[lab,above=0.75cm of cpu] {\textbf{并行总线:}多根数据线+D/C线};
    \node[lab,below=0.55cm of cpu,align=left] {优点:每拍多位\;\;\;缺点:时钟偏斜/串扰\\\textbf{距离短、速率受限}};

    % divider
    \node[lab] (sep) at ($(cpu)!0.5!(dev) + (0,-2.2)$) {\Large vs};

    % Serial block (below)
    \node[blk, below=3.0cm of cpu] (ser) {串行器 SER};
    \node[blk, below=3.0cm of dev] (des) {解串器 DES};
    \draw[arr,blue!70,thick] (ser) -- node[lab,fill=white,inner sep=1pt] {\small 差分对/链路} (des);
    \node[dev, left=2.2cm of ser] (host) {主机/控制器};
    \node[dev, right=2.2cm of des] (peer) {设备/外设};
    \draw[arr] (host) -- (ser);
    \draw[arr] (des) -- (peer);
    \node[lab,below=0.35cm of ser,align=left] {\textbf{串行链路:}编码+时钟恢复(CDR)\\\textbf{线缆少、抗干扰强、可远距离/超高速}};
\end{tikzpicture}}%
\end{center}

\paragraph{优缺点与应用对比}
{\small\begin{longtable}{|p{0.20\linewidth}|p{0.36\linewidth}|p{0.36\linewidth}|}
\hline
    	\textbf{维度} & \textbf{串行} & \textbf{并行} \\
\hline
线路与连接件 & 1对/少数对(常用\textbf{差分对}),接头小、布线简 & 多根数据线+时钟/控制线,接头宽、布线复杂 \\
\hline
时钟与同步 & 依赖\textbf{编码自同步}+\textbf{时钟恢复}(CDR),抖动容限设计重要 & 需\textbf{共享时钟},受\textbf{时钟偏斜}影响,需等长/校准 \\
\hline
速率与距离 & 易达\textbf{高频/长距}(Gbps、米-百米/公里,视介质) & 适合\textbf{短距/机内/板级},频率受限,长度随位宽增加更受限 \\
\hline
抗干扰/EMI & 差分+编码,\textbf{抗干扰强、EMI低} & 多线并行,\textbf{串扰}与EMI较明显 \\
\hline
成本与扩展 & 线缆/针脚少,\textbf{成本低、可多lane聚合}(x4/x8/x16) & 引脚/线缆多,扩展代价高 \\
\hline
典型接口 & \makecell[l]{USB 2/3/4, PCIe, SATA, \newline HDMI/DP, Ethernet,\newline \\串口(UART/RS-232/485)} 
& \makecell[l]{内存总线(DDR并非传统并口但属多线并发)
\newline \\早期IDE/并口打印机、\newline FPGA/背板并行总线} \\
\hline
\end{longtable}}

\paragraph{考试提示}
\begin{itemize}
    \item \textbf{记忆法}:\textbf{并行靠“多线同拍”},受\textbf{偏斜/串扰}困扰;\textbf{串行靠“编码+CDR”},\textbf{远距/高速}更强。
    \item \textbf{易混点}:并行\textbf{不是}一定更快——当频率提升受限时,\textbf{高速串行+多Lane}往往胜出(如PCI→PCIe)。
    \item \textbf{场景选择}:\textbf{机内短距高并发}(如CPU-内存)偏并行;\textbf{板间/机间/外设长距}首选串行。
\end{itemize}

\subsection{同步传输和异步传输}
同步与异步描述的是\textbf{时钟配合与定界方式}:
\begin{itemize}
    \item \textbf{异步传输}(Asynchronous):发送端与接收端\textbf{各自独立时钟},以\textbf{起始位(Start)/停止位(Stop)}为\textbf{字符级}定界,必要时附\textbf{奇偶校验位}。\textbf{简单、成本低},适合\textbf{低速、间歇}通信,如 UART/RS-232/485。
    \item \textbf{同步传输}(Synchronous):双方\textbf{共享时钟}或通过\textbf{自同步编码+时钟恢复(CDR)}保持同步,\textbf{帧/块级}定界,常用\textbf{前导码/标志/位填充}和\textbf{FCS/CRC}。\textbf{效率高、吞吐大},用于\textbf{高速、连续}链路,如以太网、HDLC/PPP、SATA、PCIe。
\end{itemize}

\paragraph{示意图(字段级对比)}
\begin{center}
\resizebox{0.95\linewidth}{!}{%
\begin{tikzpicture}[node distance=0pt]
    	\tikzset{seg/.style={rectangle,draw=blue!60,fill=blue!10,minimum width=1.35cm,minimum height=0.7cm,align=center,font=\small},
             seg2/.style={rectangle,draw=gray!70,fill=gray!15,minimum width=2.0cm,minimum height=0.7cm,align=center,font=\small},
             note/.style={font=\small}}
    % Asynchronous line (character oriented)
    \node[note,anchor=west] at (0,0.95) {\textbf{异步(字节定界,起始/停止)}};
    \node[seg] (a0) at (0,0) {Start};
    \node[seg, right=0pt of a0] (a1) {D0};
    \node[seg, right=0pt of a1] (a2) {D1};
    \node[seg, right=0pt of a2] (a3) {D2};
    \node[seg, right=0pt of a3] (a4) {D3};
    \node[seg, right=0pt of a4] (a5) {D4};
    \node[seg, right=0pt of a5] (a6) {D5};
    \node[seg, right=0pt of a6] (a7) {D6};
    \node[seg, right=0pt of a7] (a8) {D7};
    \node[seg, right=0pt of a8,fill=orange!20] (ap) {Parity?};
    \node[seg, right=0pt of ap,fill=green!20] (as) {Stop};
    \node[note,below=0.05cm of a0,align=left] {8N1效率示例:$\;\,\frac{8}{8+1+1}=80\%$};

    % Synchronous line (frame oriented)
    \begin{scope}[yshift=-2.3cm]
        \node[note,anchor=west] at (0,0.95) {\textbf{同步(帧定界,前导/标志+FCS)}};
        \node[seg2,fill=yellow!25] (s0) at (0,0) {Sync/Preamble};
        \node[seg2, right=0pt of s0, minimum width=5.2cm, fill=blue!10] (s1) {Payload/Frame Data};
        \node[seg2, right=0pt of s1, fill=red!15] (s2) {FCS/CRC};
        \node[note,below=0.05cm of s0,align=left] {以太网:7B 0x55 前导 + 1B 0xD5 SFD;HDLC:0x7E 标志+位填充};
    \end{scope}
\end{tikzpicture}}%
\end{center}

\paragraph{对比表}
{\small\begin{longtable}{|p{0.20\linewidth}|p{0.36\linewidth}|p{0.36\linewidth}|}
\hline
    	\textbf{维度} & \textbf{异步传输} & \textbf{同步传输} \\
\hline
定界与粒度 & \textbf{字符/字节}为单位,起始/停止位定界,\textbf{可选奇偶校验} & \textbf{帧/块}为单位,前导/标志+\textbf{位填充},帧尾常有\textbf{FCS/CRC} \\
\hline
时钟与同步 & 无共享时钟,靠\textbf{起始位边沿}临时校准采样点 & \textbf{共享时钟}或\textbf{自同步编码+CDR}保持全程同步 \\
\hline
开销与效率 & 每字节\textbf{2-3 位}冗余(起停/校验),速率越低影响越明显 & 帧级开销,相对\textbf{更高效率},适合\textbf{高速/长帧} \\
\hline
性能与场景 & 低速、间歇、点对点、实现简单,如\textbf{UART/RS-232/485} & 高速、连续、链路层协议丰富,如\textbf{以太网、HDLC/PPP、SAS/SATA、PCIe} \\
\hline
鲁棒与实现 & 容忍时钟漂移较大但吞吐受限;实现简单 & 对抖动/抖动容限要求高,需\textbf{编码、时钟恢复、缓冲}支持 \\
\hline
\end{longtable}}

\paragraph{408考试提示}
\begin{itemize}
    \item \textbf{口令}:\textbf{异步“起停定字节”},\textbf{同步“前导定帧+FCS”};二者对应的\textbf{时钟获取}方式不同。
    \item \textbf{速算}:8N1 有效负载效率约 \textasciitilde80\%;若含奇偶校验(8E1)则为 $\tfrac{8}{11}\approx72.7\%$。
    \item \textbf{知名实例}:以太网 7B 0x55 + 1B 0xD5;HDLC/PPP 使用 0x7E 标志与\textbf{位填充}。
\end{itemize}

\subsection{单向通信 双向交替通信和双向同时通信}
根据发送与接收是否可以同时进行,物理层链路可分为:
\begin{itemize}
    \item \textbf{单向通信(Simplex)}:仅\textbf{单方向}传输,反向无信道能力。例:\textbf{广播/电视下行}、\textbf{键盘→主机}早期模型。
    \item \textbf{双向交替(半双工,Half-duplex)}:\textbf{双向可达但不可同时},需\textbf{轮流}占用信道。例:对讲机、早期\textbf{共享式以太网}(需CSMA/CD)。
    \item \textbf{双向同时(全双工,Full-duplex)}:\textbf{两端可同时}发送与接收。例:\textbf{交换式以太网全双工}、\textbf{光纤双纤上下行}、\textbf{xDSL回声抵消}。
\end{itemize}

\paragraph{示意图}
\begin{center}
\resizebox{0.9\linewidth}{!}{%
\begin{tikzpicture}[node distance=3.6cm]
    		\tikzset{end/.style={rectangle,draw=blue!60,fill=blue!10,rounded corners,minimum width=2.6cm,minimum height=0.9cm,align=center},
             arr/.style={->,>=Stealth,thick}, both/.style={<->,>=Stealth,thick},
             lab/.style={font=\small}}
    % Simplex
    \node[end] (sA) {节点A};
    \node[end, right=3.6cm of sA] (sB) {节点B};
    \draw[arr,blue!70] (sA) -- (sB);
    \node[lab,above=0.2cm of sA] {\textbf{单向通信(Simplex)}};

    % Half-duplex
    \begin{scope}[yshift=-2.1cm]
        \node[end] (hA) {节点A};
        \node[end, right=3.6cm of hA] (hB) {节点B};
        \draw[arr,gray!70,dashed] (hA) -- (hB);
        \draw[arr,gray!70,dashed] (hB) -- (hA);
        \node[lab,above=0.2cm of hA] {\textbf{半双工(交替)}};
        \node[lab] at ($(hA)!0.5!(hB)+(0,-0.6)$) {不能同时,需仲裁/轮转(如CSMA/CD)};
    \end{scope}

    % Full-duplex
    \begin{scope}[yshift=-4.2cm]
        \node[end] (fA) {节点A};
        \node[end, right=3.6cm of fA] (fB) {节点B};
        \draw[both,blue!70] (fA) -- (fB);
        \node[lab,above=0.2cm of fA] {\textbf{全双工(同时)}};
        \node[lab] at ($(fA)!0.5!(fB)+(0,-0.6)$) {同时双向,\textbf{无冲突域}(交换式以太网)};
    \end{scope}
\end{tikzpicture}}%
\end{center}

\paragraph{对比表}
{\small\begin{longtable}{|p{0.20\linewidth}|p{0.26\linewidth}|p{0.26\linewidth}|p{0.22\linewidth}|}
\hline
    	\textbf{维度} & \textbf{单向通信} & \textbf{半双工} & \textbf{全双工} \\
\hline
方向/是否同时 & 单向/不可同时 & 双向/不可同时(交替) & 双向/\textbf{可同时} \\
\hline
冲突域与仲裁 & 无冲突 & 可能冲突,需\textbf{仲裁/退避}(如CSMA/CD) & \textbf{无冲突域}(点到点交换式) \\
\hline
带宽利用 & 仅一向 & 两向共享,时分占用 & 两向各得一份(或\textbf{回声抵消}/双纤) \\
\hline
典型实现 & 广播/电视下行、传感上报单向 & 共享介质、早期集线器以太网、对讲机 & 交换式以太网、光纤双纤、xDSL、PCIe 物理链路 \\
\hline
备注 & 简单,交互差 & 机制简单、效率受制于轮转/冲突 & 需要\textbf{交换/回声抵消/隔离上下行} \\
\hline
\end{longtable}}

\paragraph{408考试提示}
\begin{itemize}
    \item \textbf{以太网考点}:\textbf{共享介质+半双工}才需要\textbf{CSMA/CD};\textbf{交换式全双工\,不再使用 CSMA/CD}。
    \item \textbf{实现方式}:全双工可通过\textbf{物理分离上下行}(双绞线不同线对、光纤双纤)或\textbf{回声抵消}实现;无线蜂窝可\textbf{FDD/TDD}区分上下行,Wi‑Fi通常半双工。
    \item \textbf{易错点}:把“全双工=两倍带宽”一概而论不严谨,取决于\textbf{物理层/设备能力与瓶颈}。
\end{itemize}
\section{物理层设备}
\subsection{中继器}
中继器(Repeater)工作在\textbf{物理层(L1)},对\textbf{比特流}进行\textbf{再生整形、放大},以\textbf{延长传输距离}、弥补衰减与失真;其不识别帧或地址,不改变上层协议内容。

\paragraph{工作机理}
\begin{itemize}
    \item 将输入端退化的电/光信号\textbf{再生整形}为标准波形,并适当\textbf{放大}后转发到输出端。
    \item 双端\textbf{同速率、同编码}下工作,不能跨\textbf{不同物理层参数}(如速率、编码规则、介质类型)随意互通。
    \item 只在\textbf{比特级}处理,\textbf{不缓存、不过滤、不判断地址},也\textbf{不隔离冲突域}。
\end{itemize}

\begin{center}
\resizebox{0.9\linewidth}{!}{%
\begin{tikzpicture}[node distance=3.6cm]
        \tikzset{end/.style={rectangle,draw=blue!60,fill=blue!10,rounded corners,minimum width=2.6cm,minimum height=0.9cm,align=center},
                     rep/.style={rectangle,draw=gray!70,fill=gray!10,rounded corners,minimum width=2.4cm,minimum height=0.9cm,align=center},
                     arr/.style={->,>=Stealth,thick}, lab/.style={font=\small}}
    \node[end] (a) {段1 电缆};
    \node[rep, right=2.2cm of a] (r) {中继器\\再生/放大/整形};
    \node[end, right=2.2cm of r] (b) {段2 电缆};
    \draw[arr] (a) -- (r);
    \draw[arr] (r) -- (b);
    \node[lab,above=0.15cm of a]{衰减/失真};
    \node[lab,above=0.15cm of b]{标准波形};
\end{tikzpicture}}%
\end{center}

\paragraph{应用边界与规范}
\begin{itemize}
    \item 以太网早期\textbf{10BASE5/10BASE2/10BASE-T}等有级联规则(“5-4-3”规则等,考试了解层面)。
    \item 中继器\textbf{不改变}冲突域规模(仍是同一冲突域),现代以太网更多使用\textbf{交换机}替代。
    \item 可用于\textbf{光—电—光}的再生中继链路;注意不同物理介质/速率需\textbf{媒体转换器}等专用设备,不是一般中继器的职责。
\end{itemize}

{\small\begin{longtable}{|p{0.22\linewidth}|p{0.30\linewidth}|p{0.38\linewidth}|}
\hline
	\textbf{设备} & \textbf{工作层次/作用} & \textbf{对冲突域/广播域} \\
\hline
中继器 Repeater & 物理层:再生/整形/放大 & \textbf{不}划分冲突域;\textbf{不}影响广播域 \\
\hline
集线器 Hub & 物理层:多端口中继/共享总线 & 同上;所有端口共享,半双工,冲突概率高 \\
\hline
交换机 Switch & 数据链路层:基于 MAC 转发 & \textbf{划分冲突域}(端口为单位);\textbf{不}划分广播域 \\
\hline
\end{longtable}}

\begin{center}
{\setlength{\fboxsep}{6pt}%
\fcolorbox{red}{yellow!15}{\begin{minipage}{0.90\textwidth}
	\textbf{408考试提示}
\begin{itemize}
    \item 识记:中继器/集线器\textbf{均为 L1},\textbf{不}隔离冲突域;交换机 L2,可\textbf{划分冲突域}。
    \item 易混:中继器\textbf{不理解帧/地址},不可替代交换机;不同速率/编码的互联需要\textbf{媒体转换器}或\textbf{网桥/交换机}层面的适配。
    \item 正确场景:延长物理段、光电再生;错误场景:指望其\textbf{减少冲突/提高吞吐}(应使用交换机)。
\end{itemize}
\end{minipage}}}%
\end{center}
\subsection{集线器}
集线器(Hub)本质是\textbf{多端口中继器},工作在\textbf{物理层(L1)}。其把一个端口接收到的比特流\textbf{电气再生并广播}到所有其他端口,\textbf{共享同一冲突域},通常\textbf{半双工}工作。

\paragraph{工作特点}
\begin{itemize}
    \item 同一集线器上的所有端口共享同一物理总线语义:\textbf{一个发,其他都能“听到”}。
    \item \textbf{冲突域不被划分},需配合 CSMA/CD;端口越多、越繁忙,\textbf{冲突概率越高、有效吞吐越低}。
    \item 不识别 MAC,不做转发表或过滤;不隔离广播域。
\end{itemize}

\begin{center}
\resizebox{0.9\linewidth}{!}{%
\begin{tikzpicture}[node distance=1.6cm]
    	\tikzset{host/.style={rectangle,draw=blue!60,fill=blue!10,rounded corners,minimum width=1.8cm,minimum height=0.7cm,align=center},
                     hub/.style={rectangle,draw=gray!70,fill=gray!10,rounded corners,minimum width=2.4cm,minimum height=0.9cm,align=center},
                     arr/.style={-,thick}, lab/.style={font=\small}}
    \node[host] (h1) {Host1};
    \node[host, below=0.8cm of h1] (h2) {Host2};
    \node[host, below=0.8cm of h2] (h3) {Host3};
    \node[hub, right=3.2cm of h2] (hb) {集线器(Hub)};
    \draw[arr] (h1) -- (hb.west);
    \draw[arr] (h2) -- (hb.west);
    \draw[arr] (h3) -- (hb.west);
    \node[lab, above=0.1cm of hb]{共享冲突域,半双工};
\end{tikzpicture}}%
\end{center}

\paragraph{与交换机对比(选择题高频)}
{\small\begin{longtable}{|p{0.22\linewidth}|p{0.30\linewidth}|p{0.38\linewidth}|}
\hline
	\textbf{属性} & \textbf{集线器 Hub(L1)} & \textbf{交换机 Switch(L2)} \\
\hline
转发方式 & 比特再生+广播到所有端口 & 按 MAC 学习/转发表选择性转发 \\
\hline
冲突域 & \textbf{不}划分(整机一个) & \textbf{端口级划分}(每端口独立) \\
\hline
双工/速率 & 多为半双工/低速 & 支持全双工/更高速率 \\
\hline
性能/吞吐 & 随端口数/负载上升而下降 & 线速转发、吞吐高、可扩展 \\
\hline
应用现状 & 已基本被交换机替代 & LAN 标配接入/汇聚 \\
\hline
\end{longtable}}

\begin{center}
{\setlength{\fboxsep}{6pt}%
\fcolorbox{red}{yellow!15}{\begin{minipage}{0.90\textwidth}
	\textbf{408考试提示}
\begin{itemize}
    \item 选项辨析:看到\textbf{广播到所有端口/半双工/共享冲突域}基本指向“集线器”。
    \item 场景判断:想\textbf{隔离冲突/提升吞吐}应选“交换机”,不是集线器/中继器。
    \item 历史回顾:集线器在现代以太网已罕用,题目多在\textbf{概念辨析/对比}上设问。
\end{itemize}
\end{minipage}}}%
\end{center}

\section{编码与调制}
\subsection{编码与调制的基本概念}
本小节聚焦物理层中的“\textbf{编码}”与“\textbf{调制}”两件事:前者多指\textbf{基带数字线码}的表示方式,后者是将信息\textbf{映射到模拟载波}(幅度/频率/相位/矢量)上以适应\textbf{带通信道}。

\paragraph{核心概念速览}
{\small\begin{longtable}{|p{0.22\linewidth}|p{0.72\linewidth}|}
\hline
    	\textbf{基带编码(线码)} & 将比特流直接映射为\textbf{基带波形}(电平/极性/过零等),典型:NRZ-L/NRZ-I、RZ、AMI、曼彻斯特、差分曼彻斯特等。\\
\hline
    	\textbf{带通调制} & 将信息映射到\textbf{载波}的幅度/频率/相位,或其联合(ASK/FSK/PSK/QAM),便于在\textbf{带通信道}上传输(无线、长距、电缆/光纤)。\\
\hline
    	\textbf{码元/符号} & 一次调制/编码输出的最小单位,可能承载\textbf{k 个比特}。M进制调制:$k=\log_2 M$;\textbf{符号率}=$R_s$(波特)。\\
\hline
    	\textbf{比特率与波特率} & 比特率 $R_b$(bps),符号率 $R_s$(Baud)。二进制时 $R_b=R_s$;M进制时 $R_b=k\,R_s$,其中 $k=\log_2 M$。\\
\hline
    	\textbf{频谱与带宽} & 线码频谱与直流/高频成分相关;带通调制占用载波附近频带。带宽(Hz)受\textbf{脉冲成形}与滚降系数影响(如升余弦 $(1+\alpha)R_s$ 近似)。\\
\hline
    	\textbf{自同步与直流分量} & 自同步编码(如曼彻斯特/差分曼彻斯特)便于时钟恢复;有/无直流分量影响低通信道可否直接传(以太网常需抑制直流)。\\
\hline
\end{longtable}}

\paragraph{星座图一瞥(PSK/QAM,含判决边界与Gray标注)}
\begin{center}
\resizebox{0.95\linewidth}{!}{%
\begin{tikzpicture}[>=Stealth]
    % styles and helpers
    \tikzset{
        axis/.style={-},
        pt/.style={circle,fill=blue!70,inner sep=1.2pt},
        ptR/.style={circle,fill=red!70,inner sep=1.2pt},
        dashline/.style={gray!70,dashed},
        lab/.style={font=\small},
        bit/.style={font=\scriptsize, inner sep=1pt, fill=white, text=black}
    }
    % ====================== Panel A: PSK ======================
    \begin{scope}[xshift=0cm]
        % Axes for PSK
        \draw[axis,->] (-2.2,0) -- (2.3,0) node[below right]{I(同相)};
        \draw[axis,->] (0,-2.2) -- (0,2.3) node[above left]{Q(正交)};
        \node[lab] at (0,2.1) {PSK 星座与判决边界};

        % Parameters
        \def\R{1.6}

        % BPSK on real axis
        \node[lab] at (-1.3,1.8) {BPSK};
        \draw[ptR] (-\R,0) circle (0pt);
        \draw[ptR] (\R,0) circle (0pt);
        \node[bit] at (-\R,0.3) {1};
        \node[bit] at (\R,0.3) {0};
        % Decision boundary for BPSK
        \draw[dashline] (0,-0.4) -- (0,0.4);

        % QPSK (Gray) at 45°,135°,-135°,-45°
        \node[lab] at (0.9,1.8) {QPSK(Gray)};
        \foreach \a/\b in {45/00,135/01,-135/11,-45/10}{
            \draw[pt] ({0.95*\R*cos(\a)},{0.95*\R*sin(\a)}) circle (0pt);
            \node[bit] at ({0.95*\R*cos(\a)+0.22},{0.95*\R*sin(\a)+0.20}) {\b};
        }
        % Decision boundaries for QPSK (I=0 and Q=0)
        \draw[dashline] (-2.2,0) -- (2.2,0);
        \draw[dashline] (0,-2.2) -- (0,2.2);

        % 8-PSK ring + a few Gray labels
        \node[lab] at (-1.45,-1.8) {8-PSK};
        \draw[gray!60] (0,0) circle (1.15*\R);
        \foreach \a/\b in {0/000,45/001,90/011,135/010,180/110,225/111,270/101,315/100}{
            \draw[pt] ({1.15*\R*cos(\a)},{1.15*\R*sin(\a)}) circle (0pt);
        }
        % Show several decision rays (every 45° bisector)
        \foreach \t in {22.5,67.5,112.5,157.5,202.5,247.5,292.5,337.5}{
            \draw[dashline] (0,0) -- ({2.1*cos(\t)},{2.1*sin(\t)});
        }
        % Label a few example Gray codes (sparse to avoid clutter)
        \node[bit] at ({1.15*\R*cos(0)+0.18},{1.15*\R*sin(0)+0.18}) {000};
        \node[bit] at ({1.15*\R*cos(90)+0.18},{1.15*\R*sin(90)+0.18}) {011};
        \node[bit] at ({1.15*\R*cos(180)+0.18},{1.15*\R*sin(180)+0.18}) {110};
        \node[bit] at ({1.15*\R*cos(270)+0.18},{1.15*\R*sin(270)+0.18}) {101};
    \end{scope}

    % ====================== Panel B: 16-QAM ======================
    \begin{scope}[xshift=7.0cm]
        % Axes
        \draw[axis,->] (-2.6,0) -- (2.7,0) node[below right]{I};
        \draw[axis,->] (0,-2.6) -- (0,2.7) node[above left]{Q};
        \node[lab] at (0,2.45) {16-QAM(方阵,判决边界与最小距离)};

        % Grid parameters: levels {-3,-1,1,3} * d
        \def\d{0.55}
        \foreach \ix in {-3,-1,1,3}{
            \foreach \qy in {-3,-1,1,3}{
                \draw[pt] (\ix*\d,\qy*\d) circle (0pt);
            }
        }

        % Decision boundaries (vertical / horizontal midlines at ±2d, 0)
        \foreach \x in {-2*\d,0,2*\d}{
            \draw[dashline] (\x,-2.6) -- (\x,2.6);
        }
        \foreach \y in {-2*\d,0,2*\d}{
            \draw[dashline] (-2.6,\y) -- (2.6,\y);
        }

        % Minimum distance annotation (2d) between nearest neighbors
        \draw[<->,red!70,thick] (-3*\d,3*\d+0.45) -- (-1*\d,3*\d+0.45);
        \node[bit, text=red!80] at (-2*\d,3*\d+0.65) {$2d$(最小距离)};

        % Example Gray labels for I and Q axes (show on top/right edges)
        % I bits (from left to right): 00,01,11,10
        \node[bit] at (-3*\d,-2.35) {I:00};
        \node[bit] at (-1*\d,-2.35) {01};
        \node[bit] at ( 1*\d,-2.35) {11};
        \node[bit] at ( 3*\d,-2.35) {10};
        % Q bits (from bottom to top): 00,01,11,10
        \node[bit] at (2.35,-3*\d) {Q:00};
        \node[bit] at (2.35,-1*\d) {01};
        \node[bit] at (2.35, 1*\d) {11};
        \node[bit] at (2.35, 3*\d) {10};

        % Show full symbol bits on one example point (top-right)
        \node[bit] at (3*\d+0.35,3*\d+0.35) {I:10,Q:10};
    \end{scope}

    % ====================== Panel C: 64-QAM ======================
    \begin{scope}[xshift=13.6cm]
        % Axes
        \draw[axis,->] (-2.6,0) -- (2.7,0) node[below right]{I};
        \draw[axis,->] (0,-2.6) -- (0,2.7) node[above left]{Q};
        \node[lab] at (0,2.45) {64-QAM(8×8,稠密方阵)};

        % Grid parameters: levels {-7,-5,-3,-1,1,3,5,7} * s
        \def\s{0.32}
        \foreach \ix in {-7,-5,-3,-1,1,3,5,7}{
            \foreach \qy in {-7,-5,-3,-1,1,3,5,7}{
                \draw[pt] (\ix*\s,\qy*\s) circle (0pt);
            }
        }

        % Draw a light bounding box to emphasize span
        \draw[gray!60,rounded corners] (-7*\s-0.15,-7*\s-0.15) rectangle (7*\s+0.15,7*\s+0.15);

        % Sparse decision boundaries (only the main ones to avoid clutter)
        \foreach \x in {-6*\s,-4*\s,-2*\s,0,2*\s,4*\s,6*\s}{
            \draw[dashline] (\x,-2.55) -- (\x,2.55);
        }
        \foreach \y in {-6*\s,-4*\s,-2*\s,0,2*\s,4*\s,6*\s}{
            \draw[dashline] (-2.55,\y) -- (2.55,\y);
        }

        % Note about谱效率 / 抗噪取舍
        \node[bit, text=black] at (0,-2.2) {阶数↑→谱效率↑,但对SNR更敏感};
    \end{scope}
\end{tikzpicture}}%
\end{center}

\paragraph{关键关系式}
\begin{itemize}
    \item \textbf{M进制调制}:每符号携带 $k=\log_2 M$ bit,$R_b = k\,R_s$;在固定带宽与SNR下,增大 $M$ 提升谱效率但\textbf{误码率对噪声更敏感}。
    \item \textbf{基带线码与带宽}:自同步编码常提升时钟恢复鲁棒性,但\textbf{频谱更宽};抑制直流的码型利于\textbf{变压器/低通}链路传输。
    \item \textbf{基带 vs 带通}:\textbf{低通信道}可直接基带;\textbf{带通信道}需经调制到载波后传输。
\end{itemize}

\paragraph{两者区别与场景对比}
{\small\begin{longtable}{|p{0.20\linewidth}|p{0.36\linewidth}|p{0.36\linewidth}|}
\hline
    	\textbf{维度} & \textbf{基带编码(线码)} & \textbf{带通调制} \\
\hline
对象与输出 & 数字基带波形(电平/极性/过零) & 模拟载波的幅/频/相位或矢量(I/Q) \\
\hline
代表方式 & NRZ、RZ、AMI、曼彻斯特、差分曼彻斯特 & ASK、FSK、BPSK/QPSK、$M$-QAM \\
\hline
时钟同步 & 依赖码型自同步与接收端时钟恢复 & 载波同步+符号定时恢复(CDR/PLL) \\
\hline
频谱占用 & 与码型相关(有无直流、高频成分) & 以 $f_c$ 为中心的通带;受成形滤波影响 \\
\hline
常见场景 & 双绞线短距以太网、电气底层互连 & 无线/有线长距、光纤、同轴宽带接入 \\
\hline
\end{longtable}}

\paragraph{408考试提示}
\begin{itemize}
    \item 辨析:\textbf{比特率(bps)}、\textbf{波特率(Baud)}、\textbf{码元/符号}、\textbf{M进制与谱效率}之间的关系。
    \item 记忆:\textbf{基带=线码}(NRZ/曼彻斯特…);\textbf{带通=调制}(ASK/FSK/PSK/QAM)。\textbf{低通}$\Rightarrow$可基带;\textbf{带通}$\Rightarrow$需调制。
    \item 场景:\textbf{以太网双绞线}多用基带线码;\textbf{Wi‑Fi/蜂窝/长距链路}使用带通调制与更高阶$M$‑QAM。
\end{itemize}
\subsection{数字信号的编码}
数字线码是把\textbf{比特流}映射为\textbf{基带电信号}的规则。关注三点:\textbf{自同步能力}(便于时钟恢复)、\textbf{直流分量}(影响低通信道)、\textbf{频谱/带宽占用}(影响速率与距离)。

\paragraph{常见线码小示意}
\begin{center}
\resizebox{0.9\linewidth}{!}{%
\begin{tikzpicture}[x=0.9cm,y=0.6cm]

    % 基本参数
    \def\N{8} % 比特数量
    % 统一的比特边界(虚线)
    \foreach \x in {0,...,8} {\draw[draw=gray!60,dashed] (\x,7) -- (\x,-7.5);}    

    % 顶部“比特流”与位标注
    \node[font=\small,left=0.2cm] at (0,6.8) {\textbf{比特流}};
    \draw[draw=gray!60] (0,6) -- (8,6);
    % 例子比特序列:1 0 1 1 0 0 1 0
    \foreach [count=\i] \b in {1,0,1,1,0,0,1,0} {
    \node[font=\small] at (\i-0.5,6.35) {\b};
    }

    % 行标题与基线
    \draw[draw=gray!60] (0,3) -- (8,3);    \node[font=\small,left=0.2cm] at (0,3.8) {\textbf{双极性NRZ(AMI)}};
    \draw[draw=gray!60] (0,0) -- (8,0);    \node[font=\small,left=0.2cm] at (0,0.8) {\textbf{双极性RZ}};
    \draw[draw=gray!60] (0,-3) -- (8,-3);  \node[font=\small,left=0.2cm] at (0,-2.2) {\textbf{曼彻斯特}};
    \draw[draw=gray!60] (0,-6) -- (8,-6);  \node[font=\small,left=0.2cm] at (0,-5.2) {\textbf{差分曼彻斯特}};

    % ===== 双极性NRZ(AMI):1 交替正/负,0 为 0 电平 =====
    % 电平:+1 -> y=4, 0 -> y=3, -1 -> y=2
    \draw[thick,blue!70]
        (0,4) -- (1,4) -- (1,3) -- (2,3) % 1 -> 0
        -- (2,2) -- (3,2)                 % 1(-)
        -- (3,4) -- (4,4)                 % 1(+)
        -- (4,3) -- (5,3) -- (6,3)        % 0,0
        -- (6,2) -- (7,2)                 % 1(-)
        -- (7,3) -- (8,3);                % 0

    % ===== 双极性RZ:1 半比特取正/负交替,后半回零;0 维持零 =====
    % 电平:+1 -> y=1, 0 -> y=0, -1 -> y=-1
    \draw[thick,blue!70]
        (0,0) -- (0,1) -- (0.5,1) -- (0.5,0) -- (1,0)      % 1(+)
        -- (2,0)                                            % 0
        -- (2,-1) -- (2.5,-1) -- (2.5,0) -- (3,0)           % 1(-)
        -- (3,1) -- (3.5,1) -- (3.5,0) -- (4,0)             % 1(+)
        -- (5,0) -- (6,0)                                   % 0,0
        -- (6,-1) -- (6.5,-1) -- (6.5,0) -- (7,0)           % 1(-)
        -- (8,0);                                           % 0

    % ===== 曼彻斯特:1=高->低,0=低->高;每比特中点必跳变 =====
    % 高 -> y=-2, 低 -> y=-4
    \draw[thick,blue!70]
        % bit1=1
        (0,-2) -- (0.5,-2) -- (0.5,-4) -- (1,-4)
        % bit2=0
        -- (1,-4) -- (1.5,-4) -- (1.5,-2) -- (2,-2)
        % bit3=1
        -- (2,-2) -- (2.5,-2) -- (2.5,-4) -- (3,-4)
        % bit4=1(与前一位相同,边界处上跳)
        -- (3,-2) -- (3.5,-2) -- (3.5,-4) -- (4,-4)
        % bit5=0
        -- (4,-4) -- (4.5,-4) -- (4.5,-2) -- (5,-2)
        % bit6=0(与前一位相同,边界处下跳)
        -- (5,-4) -- (5.5,-4) -- (5.5,-2) -- (6,-2)
        % bit7=1
        -- (6,-2) -- (6.5,-2) -- (6.5,-4) -- (7,-4)
        % bit8=0
        -- (7,-4) -- (7.5,-4) -- (7.5,-2) -- (8,-2);

    % ===== 差分曼彻斯特:中点必跳变;1=起始跳变,0=起始不跳变 =====
    % 高 -> y=-5, 低 -> y=-7;初始取高
    \draw[thick,blue!70]
        % bit1=1:起始跳变到低 -> 中点跳变到高
        (0,-5) -- (0,-7) -- (0.5,-7) -- (0.5,-5) -- (1,-5)
        % bit2=0:起始不跳变,高->中点->低
        -- (1,-5) -- (1.5,-5) -- (1.5,-7) -- (2,-7)
        % bit3=1:起始跳变到高 -> 中点跳变到低
        -- (2,-5) -- (2.5,-5) -- (2.5,-7) -- (3,-7)
        % bit4=1:起始跳变到高 -> 中点跳变到低
        -- (3,-5) -- (3.5,-5) -- (3.5,-7) -- (4,-7)
        % bit5=0:起始不跳变,低->中点->高
        -- (4,-7) -- (4.5,-7) -- (4.5,-5) -- (5,-5)
        % bit6=0:起始不跳变,高->中点->低
        -- (5,-5) -- (5.5,-5) -- (5.5,-7) -- (6,-7)
        % bit7=1:起始跳变到高 -> 中点跳变到低
        -- (6,-5) -- (6.5,-5) -- (6.5,-7) -- (7,-7)
        % bit8=0:起始不跳变,低->中点->高
        -- (7,-7) -- (7.5,-7) -- (7.5,-5) -- (8,-5);
\end{tikzpicture}}%
\end{center}

\paragraph{常见线码对比}
{\small\begin{longtable}{|p{0.18\linewidth}|p{0.18\linewidth}|p{0.18\linewidth}|p{0.22\linewidth}|p{0.18\linewidth}|}
\hline
    	\textbf{线码} & \textbf{自同步} & \textbf{直流分量} & \textbf{带宽/频谱特性} & \textbf{常见应用} \\
\hline
NRZ-L/NRZ-I & 弱(长0/1易失同步) & 有(取决于数据偏置) & 带宽较窄,频谱含直流 & 早期链路、与4B/5B/8B/10B配合 \\
\hline
RZ & 中 & 较少直流 & 频谱比NRZ更宽(回到零) & 早期系统 \\
\hline
曼彻斯特 & 强(每比特中点必跳变) & 无直流(均衡) & 带宽约为NRZ的~2倍 & 10BASE-T、RFID等 \\
\hline
差分曼彻斯特 & 强(中点必跳变) & 无直流 & 带宽与曼彻斯特相近 & 令牌环等 \\
\hline
AMI(双极性) & 中(需避免长零) & 直流抑制较好 & 频谱集中,需违规/填充避免长零 & E1/T1 体系(HDB3/B8ZS变体) \\
\hline
\end{longtable}}

\paragraph{408考试提示}
\begin{itemize}
    \item \textbf{自同步排序}:曼彻斯特/差分曼彻斯特 \textgreater{} AMI \textgreater{} NRZ。
    \item \textbf{直流分量}:曼彻斯特/差分曼彻斯特/双极性(AMI)\textbf{无直流或被抑制},利于变压器耦合与低通链路。
    \item \textbf{带宽}:曼彻斯特\textbf{带宽约翻倍};NRZ较窄但需码组映射(如4B/5B、8B/10B)来保证过零密度。
\end{itemize}
\subsubsection{曼彻斯特编码}
每比特间隔\textbf{中点必有一次跳变}用于\textbf{时钟恢复}:
\begin{itemize}
    \item 约定1种常见映射:\textbf{1=高到低,0=低到高}(IEEE 802.3 10BASE-T)。
    \item 优点:\textbf{自同步强、无直流分量、实现简单};缺点:\textbf{带宽约为NRZ的2倍}。
\end{itemize}
\subsubsection{差分曼彻斯特编码}
规则:\textbf{每个比特中点必跳变};数据位由\textbf{起始边界是否跳变}来表示(\textbf{1=起始跳变,0=起始不跳变}的一种常见约定)。
\begin{itemize}
    \item 优点:\textbf{抗极性反接}(差分定义)、\textbf{自同步强}、\textbf{无直流};带宽略高于NRZ。
\end{itemize}

\subsubsection{双极性不归零(AMI,Bipolar NRZ)}
\textbf{编码规则}:比特1用\textbf{交替极性}的全比特宽脉冲(+V、-V 轮流出现),比特0为\textbf{零电平}(0V)。相邻“1”极性交替,称为\textbf{交替标记反转}(AMI)。
\begin{itemize}
    \item \textbf{直流分量}:正负对称,\textbf{DC抑制好};频谱较NRZ更集中,低频分量较少。
    \item \textbf{自同步性}:较\textbf{一般},\textbf{长零序列}会导致过零密度不足,需配合\textbf{零替代/违规编码}(如\textbf{HDB3(E1)}、\textbf{B8ZS(T1)})。
    \item \textbf{带宽占用}:与NRZ相近,\textbf{明显小于曼彻斯特}。
    \item \textbf{工程特性}:出现\textbf{同极性连续“1”}可作为\textbf{双极性违规}进行错误/控制标记识别(也用于零替代方案)。
    \item \textbf{典型应用}:\textbf{E1/T1 干线}基础线码(结合HDB3/B8ZS);专线/承载场景常见。
\end{itemize}

\subsubsection{双极性归零(Bipolar RZ)}
\textbf{编码规则}:比特1在\textbf{半个比特时间}输出+V或−V(\textbf{极性交替}),\textbf{后半比特回零};比特0为\textbf{全比特零电平}。
\begin{itemize}
    \item \textbf{自同步性}:\textbf{较强},因“1”位\textbf{中点必回零},过零密度高,\textbf{时钟恢复更稳};长零序列仍可能影响同步。
    \item \textbf{直流分量}:正负对称且回零,\textbf{DC分量很小},适合\textbf{变压器耦合/低通链路}。
    \item \textbf{带宽占用}:脉冲变窄、回零使\textbf{高频分量增加},\textbf{带宽大于AMI(NRZ)}(约为其1.5–2倍,视成形而定)。
    \item \textbf{典型应用}:早期载波系统、部分\textbf{磁记录/专用链路};现代以太网中少用,更多用于\textbf{教学与特定工程}。
    \item \textbf{工程注意}:为避免\textbf{长零}导致的定时退化,常配合\textbf{扰码/零替代}或上层帧结构保障过零密度。
\end{itemize}

\subsection{模拟信号的调制}
带通调制是把离散信息映射到模拟载波上(幅度/频率/相位或I/Q矢量),以适应只能通过特定频带传输的信道(无线、同轴、光纤等)。

\paragraph{带通信号的一般形式与I/Q正交分解}
带通信号可写为
\[
    s(t)=A(t)\cos\bigl(2\pi f_c t+\varphi(t)\bigr),\quad \text{或}\quad s(t)=I(t)\cos(2\pi f_c t)-Q(t)\sin(2\pi f_c t),
\]
其中 $f_c$ 为载波频率;$A(t)$、$\varphi(t)$ 或 $I/Q$ 随符号变化。调制的本质是选择\textbf{哪一维}承载信息:幅度(ASK)、频率(FSK)、相位(PSK),或同时在I/Q二维承载(QAM)。

\paragraph{ASK/FSK/PSK/QAM 的定义与特点}
\begin{itemize}
    \item \textbf{二进制ASK(开关键控,OOK)}:比特1用较大幅度,0用小幅度/零幅度;实现简单,但\textbf{抗噪声(功率效率)差}。
    \item \textbf{二进制FSK}:用不同载波\textbf{频率}表示0/1($f_0\neq f_1$)。\textbf{非相干}解调易做、抗衰落较稳,但\textbf{带宽占用较大}。
    \item \textbf{PSK}:用\textbf{相位}表示信息。BPSK 每符号1比特,QPSK 每符号2比特。相干解调下,\textbf{功率效率高于ASK/FSK}(同带宽下误码率更低)。
    \item \textbf{$M$‑QAM}:在I/Q平面上取$M$个离散点(方阵最常见),每符号 $k=\log_2 M$ 比特,\textbf{谱效率高},但对SNR更敏感,阶数越高越易误码。
\end{itemize}

\paragraph{时间域小示意(与图示一比一复刻:ASK/FSK/PSK)}
\begin{center}
\leavevmode
\resizebox{\linewidth}{!}{%
\begin{tikzpicture}[x=1.15cm,y=0.85cm]
    % 全局参数
    \def\N{8}
    \def\bits{0,1,0,0,1,1,1,0}
    % 行的基线y坐标
    \def\ybb{6.8}  % 基带
    \def\yask{4.4} % ASK
    \def\yfsk{2.2} % FSK
    \def\ypsk{0.0} % PSK

    % 为了确保比特边界处严格对齐,载波在每个比特宽度内走“整周期”
    \def\cyclesASK{2} % ASK 每比特整周期数
    \def\cyclesPSK{2} % PSK 每比特整周期数
    \def\cyclesFHigh{2}  % FSK 中“高频”每比特整周期数(1)
    \def\cyclesFLow{1}  % FSK 中“低频”每比特整周期数(0)

    % 竖向比特边界
    \foreach \x in {0,...,8}{
        \draw[gray!60,dashed] (\x,\ybb+1.1) -- (\x,\ypsk-1.0);
    }

    % 左侧标题
    \node[anchor=east,font=\small] at (-0.35,\ybb+0.5) {数字基带信号};
    \node[anchor=east,font=\small] at (-0.35,\yask+0.4) {(a) 调幅};
    \node[anchor=east,font=\small] at (-0.35,\yfsk+0.4) {(b) 调频};
    \node[anchor=east,font=\small] at (-0.35,\ypsk+0.4) {(c) 调相};

    % 顶部:数字基带(0/1高低电平阶梯)
    \draw[gray!70] (0,\ybb) -- (\N,\ybb);
    \foreach [count=\i] \b in \bits {
        % 位值标注
        \node[font=\small] at (\i-0.5,\ybb+0.85) {\b};
        % 1比特矩形(高电平=1,低电平=0)
        \ifnum\b=1
            \draw[thick] (\i-1,\ybb) -- (\i-1,\ybb+0.9) -- (\i,\ybb+0.9) -- (\i,\ybb) -- cycle;
        \else
            % 画低电平基线
            \draw[thick] (\i-1,\ybb) -- (\i,\ybb);
        \fi
    }

    % ===== (a) ASK(OOK:有/无载波,与示意图一致并标注文本) =====
    \draw[gray!70] (0,\yask) -- (\N,\yask);
    \foreach [count=\i] \b in \bits {
        % 顶部文字:有载波/无载波
        \ifnum\b=1
            \node[font=\scriptsize] at (\i-0.5,\yask+1.05) {有载波};
        \else
            \node[font=\scriptsize] at (\i-0.5,\yask+1.05) {无载波};
        \fi
        % 每位的载波/无载波波形(边界已由全局虚线标出)
        \begin{scope}[xshift={(\i-1)}]
            \ifnum\b=1
                % 位=1:绘制连续正弦载波(有载波)
                \pgfmathsetmacro{\phaseStep}{10} % 每位相位滞后角(度)
                \pgfmathsetmacro{\ph}{(\i-1)*\phaseStep}
                \draw[blue!70,thick,domain=0:1,samples=160,smooth,variable=\t]
                    plot (\t, {\yask + 0.75*sin(360*\cyclesASK*\t - \ph)});
            \else
                % 位=0:无载波,画基线
                \draw[blue!70,thick] (0,\yask) -- (1,\yask);
            \fi
        \end{scope}
    }

    % ===== (b) FSK(0/1用不同频率;边界对齐:整周期) =====
    % 约定:0 -> 低频(1周期/比特),1 -> 高频(2周期/比特)
    \draw[gray!70] (0,\yfsk) -- (\N,\yfsk);
    \foreach [count=\i] \b in \bits {
        \begin{scope}[xshift={(\i-1)}]
            \ifnum\b=1
                \def\cf{\cyclesFHigh} % 高频
            \else
                \def\cf{\cyclesFLow} % 低频
            \fi
            \draw[green!60!black,thick,domain=0:1,samples=160,smooth,variable=\t]
                plot (\t, {\yfsk + 0.7*sin(360*\cf*\t)});
        \end{scope}
    }

    % ===== (c) PSK(相位键控:0/1相位0/180°;边界对齐:整周期) =====
    \draw[gray!70] (0,\ypsk) -- (\N,\ypsk);
    \foreach [count=\i] \b in \bits {
        \begin{scope}[xshift={(\i-1)}]
            \ifnum\b=1
                \def\phi{180} % 1 -> 反相
            \else
                \def\phi{0}   % 0 -> 同相
            \fi
            \draw[red!70,thick,domain=0:1,samples=160,smooth,variable=\t]
                plot (\t, {\ypsk + 0.7*sin(360*\cyclesPSK*\t + \phi)});
        \end{scope}
    }
\end{tikzpicture}}%
\end{center}

\paragraph{关系总览(I/Q 与频域)}
\begin{center}
\leavevmode
\resizebox{0.95\linewidth}{!}{%
\begin{tikzpicture}[x=1cm,y=1cm,>=stealth]
    % ========== (d) I/Q 平面:PSK在等幅圆上,QAM为二维网格 ==========
    \begin{scope}[shift={(0,0)}]
        % 轴
        \draw[->] (-2.2,0) -- (2.3,0) node[right]{\small I};
        \draw[->] (0,-2.2) -- (0,2.3) node[above]{\small Q};
        % PSK 等幅圆
        \def\r{1.3}
        \draw[gray!60] (0,0) circle (\r);
        % BPSK 两点(在 I 轴)
        \foreach \x in {-\r,\r} {\fill[red!70] (\x,0) circle (1.8pt);} 
        \node[red!70!black,anchor=west,font=\scriptsize] at (0.2,-1.7) {BPSK:相位二选,等幅圆上两点};
        % QPSK 四点(在等幅圆四象限)
        \pgfmathsetmacro{\rq}{\r/sqrt(2)}
        \foreach \sx in {-1,1} \foreach \sy in {-1,1} {\fill[blue!70] (\sx*\rq,\sy*\rq) circle (1.6pt);} 
        \node[blue!70!black,anchor=west,font=\scriptsize] at (0.2,-1.95) {QPSK:相位四选,等幅圆四点};
        % 16-QAM 网格点(四乘四)
        \foreach \ix in {-1.5,-0.5,0.5,1.5} {
            \foreach \iy in {-1.5,-0.5,0.5,1.5} {
                \fill[green!60!black] (\ix,\iy) circle (1.3pt);
            }
        }
        \node[green!50!black,anchor=west,font=\scriptsize] at (0.2,-2.2) {16-QAM:幅度+相位双维取值,二维星座};
        \node[anchor=west,font=\small] at (-2.1,2.05) {(d) I/Q 平面(PSK 与 QAM)};
    \end{scope}

    % ========== (e) ASK:I 轴不同幅度(OOK 可以包含原点) ==========
    \begin{scope}[shift={(6,0)}]
        \draw[->] (-0.4,0) -- (3.2,0) node[right]{\small I};
        \node[anchor=west,font=\small] at (-0.3,2.05) {(e) ASK(幅度变化)};
        % OOK: 0 -> 0, 1 -> A
        \fill[red!60] (0,0) circle (1.8pt); \node[font=\scriptsize,below] at (0,0) {0};
        \fill[red!60] (2.2,0) circle (1.8pt); \node[font=\scriptsize,below] at (2.2,0) {1};
        \draw[<->,gray!60] (0,0.7) -- node[fill=white,inner sep=1pt,font=\scriptsize]{幅度变化 $\Delta A$} (2.2,0.7);
    \end{scope}

    % ========== (f) FSK 频域:载波两侧频线(正交BFSK 示意) ==========
    \begin{scope}[shift={(11,0)}]
        \draw[->] (-0.5,0) -- (4.6,0) node[right]{\small 频率 $f$};
        \node[anchor=west,font=\small] at (-0.4,2.05) {(f) FSK(频域示意)};
        % fc 与两侧频率
        \def\fc{2.05}
        \def\df{0.9}
        \draw[gray!60] (\fc,-0.08) -- (\fc,0.08); \node[font=\scriptsize,below] at (\fc,-0.08) {$f_c$};
        % 频谱线(抽象脉冲)
        \draw[green!60!black,thick] (\fc-\df,0) -- (\fc-\df,1.25);
        \draw[green!60!black,thick] (\fc+\df,0) -- (\fc+\df,1.25);
        % Δf 标注
        \draw[<->,gray!60] (\fc-\df,1.5) -- node[fill=white,inner sep=1pt,font=\scriptsize]{$2\Delta f$} (\fc+\df,1.5);
        \node[font=\scriptsize,below] at (\fc-\df,0) {$f_0$};
        \node[font=\scriptsize,below] at (\fc+\df,0) {$f_1$};
    \end{scope}
\end{tikzpicture}}%
\end{center}

\paragraph{一眼看懂关系}
\begin{itemize}
    \item \textbf{只改幅度}:$A \\Rightarrow \\text{ASK}$;\textbf{只改相位}:$\varphi \\Rightarrow \\text{PSK}$;\textbf{只改频率}:$f \\Rightarrow \\text{FSK}$;\textbf{同时改I/Q(幅+相位)}:$(I,Q) \\Rightarrow \\text{QAM}$。
    \item \textbf{I/Q 视角}:PSK 星座点在\textbf{等幅圆}上(相位离散);QAM 星座点是\textbf{二维网格}(幅+相位离散)。
    \item \textbf{频域视角}:FSK 体现为\textbf{频线位置}变化(间隔 $\Delta f$);ASK/PSK/QAM 的\textbf{主带宽}主要由符号率与成形滚降系数 $\alpha$ 决定。
\end{itemize}



\paragraph{关键关系式(速率、带宽、谱效率)}
\begin{itemize}
    \item \textbf{每符号比特数}:$k=\log_2 M$,$R_b=k\,R_s$($R_b$比特率,$R_s$符号率)。
    \item \textbf{成形滤波带宽近似}:升余弦成形时 $B\approx (1+\alpha)R_s$($0\le\alpha\le1$)。故\textbf{谱效率} $\eta=\dfrac{R_b}{B}\approx\dfrac{k}{1+\alpha}$。
    \item \textbf{FSK带宽}(概略):$B\approx 2\Delta f + (1+\alpha)R_s$($\Delta f$为两个频率的间隔;正交BFSK常取 $\Delta f\approx R_s/2$)。
    \item \textbf{功率效率(定性)}:相干BPSK/QPSK $>$ FSK(相干) $>$ ASK;$M$ 增大(如16/64‑QAM)谱效率升高,但对SNR更敏感,给定误码率所需 $E_b/N_0$ 更高。
\end{itemize}

\paragraph{常见带通调制对比}
{\small\begin{longtable}{|p{0.16\linewidth}|p{0.16\linewidth}|p{0.24\linewidth}|p{0.20\linewidth}|p{0.20\linewidth}|}
\hline
	\textbf{调制} & \textbf{每符号比特} & \textbf{带宽/谱效率(近似)} & \textbf{接收复杂度} & \textbf{特点/应用} \\
\hline
ASK(OOK) & 1 & $B\approx (1+\alpha)R_s$,$\eta\approx 1/(1+\alpha)$ & 非相干/包络检波可行 & 硬件简单,功率效率差;短距、低功耗链路、RFID等 \\
\hline
2‑FSK & 1 & $B\approx 2\Delta f+(1+\alpha)R_s$;正交取 $\Delta f\!\approx\!R_s/2$ & 非相干或相干 & 抗衰落好、频谱占用较大;低速无线、旧制式调制 \\
\hline
BPSK/QPSK & 1/2 & $B\approx (1+\alpha)R_s$;$\eta\approx k/(1+\alpha)$ & 载波/定时同步+相干判决 & 功率效率高;QPSK常用于\textbf{Wi‑Fi/蜂窝/卫星} \\
\hline
$M$‑QAM(16/64…) & $\log_2 M$ & $B\approx (1+\alpha)R_s$;$\eta\approx k/(1+\alpha)$ & 相干,且需I/Q增益/相位校准 & 谱效率高;对SNR/线性度更敏感,常配合\textbf{Gray编码}与均衡/纠错 \\
\hline
\end{longtable}}

\paragraph{408考试提示}
\begin{itemize}
    \item \textbf{概念辨析}:\textbf{基带(线码)} vs \textbf{带通(调制)};$R_b$、$R_s$、$k$ 与谱效率的关系:$R_b=kR_s$、$\eta\approx k/(1+\alpha)$。
    \item \textbf{优劣排序(定性)}:功率效率 BPSK/QPSK $>$ FSK $>$ ASK;谱效率 QAM $>$ QPSK $>$ BFSK(正交)。
    \item \textbf{易错点}:提高 $M$ 并非“白捡容量”,\textbf{同样误码率}所需 $E_b/N_0$ 会显著升高;FSK\textbf{非相干}比相干通常多消耗约\textasciitilde{}3\,dB。
    \item \textbf{工程常识}:QAM/PSK 实际系统用\textbf{成形滤波}与\textbf{Gray编码}降低ISI/误比特;无线链路需\textbf{同步(载波+定时)}与\textbf{均衡/纠错}协同。
\end{itemize}


\subsection{信道复用技术}

复用(Multiplexing)是在\textbf{一个物理信道}上同时承载多个逻辑通信的技术;按资源维度划分常见有:\textbf{频分(FDM)}、\textbf{时分(TDM)}、\textbf{波分(WDM)}与\textbf{码分(CDM)}。\textbf{多址}(Multiple Access)是复用思想在\textbf{共享介质多用户接入}上的体现,如 FDMA/TDMA/CDMA。二者核心一致但面向的层次/场景不同。

为便于一眼理解,下图以\textbf{频域/时间轴/波长/码域}四个视角并排示意:

\begin{figure}[htbp]
    \centering
    \resizebox{0.95\linewidth}{!}{%
    \begin{tikzpicture}[x=1cm,y=1cm]
        % Panel (a) FDM 频谱
        \begin{scope}
            \node[anchor=west] at (-0.2,1.6) {\small (a) FDM 频谱:互不重叠,设护频带};
            % axis
            \draw[->] (0,0) -- (8,0) node[below] {\small 频率 $f$};
            % channels as rectangles
            \fill[blue!20] (0.5,0) rectangle (2.0,1.0);
            \fill[blue!20] (2.6,0) rectangle (4.1,1.2);
            \fill[blue!20] (4.7,0) rectangle (6.2,0.8);
            % guard bands
            \draw[gray!60,decorate,decoration={brace,amplitude=3pt}] (2.0,0.2) -- (2.6,0.2) node[midway,below=4pt,gray!70]{\scriptsize 护频带};
            \draw[gray!60,decorate,decoration={brace,amplitude=3pt}] (4.1,0.2) -- (4.7,0.2) node[midway,below=4pt,gray!70]{\scriptsize 护频带};
        \end{scope}

        % Panel (b) TDM 时间轴
        \begin{scope}[shift={(0,-2.0)}]
            \node[anchor=west] at (-0.2,1.6) {\small (b) TDM 时间:时隙轮转,帧同步};
            % timeline
            \draw[->] (0,0.4) -- (8,0.4) node[below] {\small 时间 $t$};
            % slots
            \foreach \i/\lab/\col in {0/A/red!25,1/B/green!25,2/C/blue!25,3/A/red!25,4/B/green!25,5/C/blue!25} {
                \fill[\col] (0.5+\i*1.0,0) rectangle (1.4+\i*1.0,0.8);
                \node at (0.95+\i*1.0,0.4) {\small $\lab$};
            }
            % frame mark
            \draw[gray!60,dashed] (0.5,-0.05) -- (0.5,1.0);
            \draw[gray!60,dashed] (3.5,-0.05) -- (3.5,1.0);
            \node[gray!70] at (2.0,1.1) {\scriptsize 一帧};
        \end{scope}

        % Panel (c) WDM 波长
        \begin{scope}[shift={(0,-4.1)}]
            \node[anchor=west] at (-0.2,1.6) {\small (c) WDM 波长:不同 $\lambda$ 复用于光纤};
            \draw[->] (0,0) -- (8,0) node[below] {\small 波长 $\lambda$(nm)};
            % channels as triangles/rectangles
            \fill[purple!20] (1.0,0) rectangle (2.0,1.0);
            \fill[purple!20] (3.0,0) rectangle (4.0,1.2);
            \fill[purple!20] (5.0,0) rectangle (6.0,0.9);
            \node[gray!70] at (1.5,1.35) {\scriptsize 1310};
            \node[gray!70] at (3.5,1.55) {\scriptsize 1550};
            \node[gray!70] at (5.5,1.25) {\scriptsize 1625};
        \end{scope}

        % Panel (d) CDM 码域
        \begin{scope}[shift={(0,-6.2)}]
            \node[anchor=west] at (-0.2,1.6) {\small (d) CDM 码域:正交/低相关码同时占用全时频};
            % chips timeline
            \draw[->] (0,0.4) -- (8,0.4) node[below] {\small 芯片序列(chips)};
            \foreach \i in {0,...,7} {\draw[gray!50] (0.5+\i*0.8,0) -- (0.5+\i*0.8,0.9);}
            % two codes rows
            \node[anchor=east] at (0.3,1.0) {\scriptsize 码1:};
            \node[anchor=east] at (0.3,-0.1) {\scriptsize 码2:};
            \foreach \i/\b in {0/1,1/-1,2/1,3/1,4/-1,5/1,6/-1,7/1} {
                \fill[cyan!25] (0.5+\i*0.8,0.5) rectangle (0.5+\i*0.8+0.8,0.9);
                \node at (0.9+\i*0.8,0.7) {\scriptsize $\b$};
            }
            \foreach \i/\b in {0/1,1/1,2/-1,3/1,4/1,5/-1,6/1,7/-1} {
                \fill[orange!25] (0.5+\i*0.8,0.0) rectangle (0.5+\i*0.8,0.0); % placeholder to keep style
                \node at (0.9+\i*0.8,0.2) {\scriptsize $\b$};
            }
            % correlation comment
            \node[gray!70] at (6.7,1.2) {\scriptsize $\langle\text{码1},\text{码2}\rangle\approx 0$};
        \end{scope}
    \end{tikzpicture}%
    }
    \caption{信道复用四视角示意:FDM(频域)、TDM(时间)、WDM(波长)、CDM(码域)}
    \label{fig:mux-overview}
\end{figure}

\paragraph{术语对齐}
\begin{itemize}
    \item 复用(物理/链路层常见):FDM/TDM/WDM/CDM 对应\textbf{合/分波器、合/分时器、合/分码器}。
    \item 多址(MAC 常见):FDMA/TDMA/CDMA —— 面向\textbf{多用户共享接入}的资源划分。
\end{itemize}

\subsubsection{频分复用FDM}
思想:把可用带宽分成若干\textbf{互不重叠}的子带,各路用户各占一段频谱,中间留\textbf{护频带}以减小串扰。典型设备为\textbf{合路器/分路器};在无线/有线电视/宽带接入中广泛使用。与之在光域的对应是\textbf{WDM}。

要点:
\begin{itemize}
    \item 近似带宽预算:$B_\text{总}\approx\sum_i B_i +$ 护频带总和;调制成形越好,护频带可越小。
    \item FDM 不需严格时钟同步,\textbf{频率规划}与\textbf{滤波器选择性}是关键。
    \item OFDM 是“\textbf{正交子载波}的 FDM”,子载波间隔 $\Delta f=1/T_u$,可\textbf{重叠而不互扰},并常配\textbf{循环前缀}对抗多径。
\end{itemize}

\subsubsection{时分复用TDM}
思想:各路在\textbf{时间上轮流}占用链路。\textbf{同步TDM}按固定时隙分配(即便空载也保留);\textbf{统计复用}按需求动态分配时隙,提高链路利用率。

要点:
\begin{itemize}
    \item 线路速率 $R_\text{线}$ 需覆盖各源汇总与开销:$R_\text{线}\gtrsim \sum_i R_i + R_\text{开销}$。
    \item 帧结构含\textbf{帧同步/时钟恢复}字段;跨域时可能需\textbf{缓冲与整形}。
    \item 典型:E1/T1 复用层级,SDH/SONET 帧复用,交换机端口轮询(内核层面)。
\end{itemize}

\subsubsection{波分复用}
思想:在同一根光纤上以不同\textbf{波长}承载多路光信号,等价于“光域的 FDM”。分为\textbf{粗波分(CWDM)}与\textbf{密集波分(DWDM)};前者通道间隔大、成本低,后者通道密、容量高。

要点:
\begin{itemize}
    \item 典型窗口:\textbf{1310/1550/1625\,nm};DWDM 在 C/L 波段可达\textbf{几十/上百}波长。
    \item 关键器件:\textbf{AWG}、\textbf{光合分波器}、\textbf{OADM/ROADM};工程上关注\textbf{光功率预算、色散、非线性}。
\end{itemize}

\subsubsection{码分复用CDM}
思想:各路同时占用\textbf{全时频},靠\textbf{编码正交(或低相关)}在接收端通过相关解扩区分。常见实现是\textbf{扩频(DSSS)}与\textbf{跳频(FHSS)};通信制式中的\textbf{CDMA}即由此而来。

要点:
\begin{itemize}
    \item \textbf{处理增益} $G_p=\dfrac{W}{R_b}$(扩频带宽 $W$ 对比用户比特率 $R_b$),$G_p$ 越大越抗干扰/多径。
    \item 需要\textbf{功率控制}以缓解\textbf{远近效应};码字选取常用\textbf{Walsh}、\textbf{m序列/Gold}(低互相关)。
    \item 解扩依赖\textbf{精确同步}(码相位/载波/定时)。
\end{itemize}

\paragraph{考试例题(CDMA 解扩计算)}
两用户采用长度 $4$ 的正交扩频码:$\mathbf{c}_1=[+1,+1,-1,-1]$,$\mathbf{c}_2=[+1,-1,+1,-1]$,满足 $\mathbf{c}_1\cdot\mathbf{c}_2=0$。已知比特到幅度映射为 $1\mapsto +1$、$0\mapsto -1$。本比特周期内,用户1发送 $1$,用户2发送 $0$,忽略噪声与载波/码相位偏差。

(1)问:复用后接收端芯片序列 $\mathbf{r}$ 为何?(2)问:分别对 $\mathbf{r}$ 与 $\mathbf{c}_1$、$\mathbf{c}_2$ 做相关,判决两用户比特。

解:映射得到 $b_1=+1$、$b_2=-1$。叠加:$\mathbf{r}=b_1\,\mathbf{c}_1+b_2\,\mathbf{c}_2=[1,1,-1,-1]+[-1,1,-1,1]=[0,2,-2,0]$。

相关判决:
\[
\langle\mathbf{r},\mathbf{c}_1\rangle=0\cdot1+2\cdot1+(-2)\cdot(-1)+0\cdot(-1)=4\Rightarrow \;\hat b_1=\operatorname{sign}\!\left(\dfrac{1}{4}\langle\mathbf{r},\mathbf{c}_1\rangle\right)=+1
\]
\[
\langle\mathbf{r},\mathbf{c}_2\rangle=0\cdot1+2\cdot(-1)+(-2)\cdot1+0\cdot(-1)=-4\Rightarrow \;\hat b_2=\operatorname{sign}\!\left(\dfrac{1}{4}\langle\mathbf{r},\mathbf{c}_2\rangle\right)=-1
\]
故判决比特为用户1: $1$,用户2: $0$。

提示:若存在\textbf{远近效应}(一用户功率显著更大)或码间相关非零/同步误差,多用户间会出现泄漏干扰,需\textbf{功率控制}与\textbf{严格同步}保障判决可靠。

\paragraph{对比一览}
{\small\begin{longtable}{|p{0.17\linewidth}|p{0.20\linewidth}|p{0.25\linewidth}|p{0.18\linewidth}|p{0.20\linewidth}|}
\hline
	\textbf{方案} & \textbf{资源维度} & \textbf{同步/隔离机制} & \textbf{典型系统} & \textbf{优缺点(概述)} \\
\hline
FDM & 频率子带 + 护频带 & 频率规划 + 滤波选择性 & 有线电视、微波、xDSL 上下行分频 & 实现成熟;频谱利用受护频带限制 \\
\hline
TDM(同步/统计) & 时间时隙/帧 & 帧同步、时钟恢复 & E1/T1、SDH/SONET、交换机内部调度 & 统计复用效率高;需严格定时 \\
\hline
WDM(CWDM/DWDM) & 光波长 & 光合分波/AWG/ROADM & 城域/骨干光传输、数据中心互联 & 容量极高;器件成本/色散/非线性挑战 \\
\hline
CDM(DSSS/FHSS) & 码域(正交/低相关) & 相关解扩 + 功率控制 & 蜂窝 CDMA、抗干扰专用链路 & 抗干扰/抗窄带衰落;设计复杂、同步严格 \\
\hline
\end{longtable}}

\paragraph{408 考试提示}
\begin{itemize}
    \item \textbf{辨析:}复用 vs 多址;TDM 的\textbf{同步}与\textbf{统计}两类;WDM 属于\textbf{物理层光域复用}。
    \item \textbf{名词:}护频带/护时带、帧同步、处理增益 $G_p$、远近效应、循环前缀(OFDM)。
    \item \textbf{易错:}把 OFDM 误认为“额外占带宽”——\textbf{子载波正交}可重叠;把 CDM 与\textbf{频/时}划分混淆。
\end{itemize}

% ========== 第3章 数据链路层 ==========


\chapter{数据链路层}

\section{数据链路层的功能}
\subsection{为网络层提供服务}
数据链路层通过\textbf{帧}向网络层提供传输服务。根据是否建立连接、是否逐帧确认与重传,可归纳为三类(与教材一致):
\begin{itemize}
    \item \textbf{无确认的无连接服务}(Best Effort):不建立连接,不逐帧确认,不保证到达与有序,\textbf{开销最小、时延低}。典型:以太网(IEEE 802.3)。
    \item \textbf{带确认的无连接服务}:不建立连接,但逐帧\textbf{链路层确认/重传},\textbf{仅保证单跳可靠}。典型:802.11 无线局域网的 MAC ACK。
    \item \textbf{面向连接的服务}:建立—传输—释放三个阶段,通常配合\textbf{序号、确认、重传、流量控制},对\textbf{每一跳}提供可靠、有序的帧传输。典型:HDLC/LLC Type 2。
\end{itemize}

\begin{figure}[htbp]
    \centering
    \resizebox{0.9\linewidth}{!}{%
    \begin{tikzpicture}[x=1cm,y=1cm]
        % lifelines
        \begin{scope}
            \node[anchor=east] (tx1) at (0,3.0) {\small 发送端};
            \node[anchor=west] (rx1) at (7,3.0) {\small 接收端};
            \draw[gray!50] (0.2,2.2) -- (0.2,3.8);
            \draw[gray!50] (6.8,2.2) -- (6.8,3.8);
            \node[anchor=west] at (-0.2,3.8) {\small (a) 无确认的无连接};
            % frames only
            \draw[->,>=Stealth,blue!70,thick] (0.2,3.5) -- (6.8,3.5) node[midway,above] {\scriptsize 帧1};
            \draw[->,>=Stealth,blue!70,thick] (0.2,3.0) -- (6.8,3.0) node[midway,above] {\scriptsize 帧2};
            \draw[->,>=Stealth,blue!70,thick] (0.2,2.5) -- (6.8,2.5) node[midway,above] {\scriptsize 帧3};
        \end{scope}

        \begin{scope}[shift={(0,-2.6)}]
            \node[anchor=east] at (0,3.0) {\small 发送端};
            \node[anchor=west] at (7,3.0) {\small 接收端};
            \draw[gray!50] (0.2,2.0) -- (0.2,3.8);
            \draw[gray!50] (6.8,2.0) -- (6.8,3.8);
            \node[anchor=west] at (-0.2,3.8) {\small (b) 带确认的无连接};
            % frame + ack
            \draw[->,>=Stealth,blue!70,thick] (0.2,3.5) -- (6.8,3.5) node[midway,above] {\scriptsize 帧1};
            % ACK1:对“帧1”的确认;数字1表示确认第1个数据帧
            % 坐标:(0.2,3.2) -> (6.8,3.2)
            %  - x=0.2 与 x=6.8:分别对应左/右两条生命线的位置
            %  - y=3.2:该条确认消息在时间轴上的垂直位置
            \draw[<-,>=Stealth,green!60!black,thick,dashed] (0.2,3.1) -- (6.8,3.1) node[midway,above] {\scriptsize ACK1};
            \draw[->,>=Stealth,blue!70,thick] (0.2,2.7) -- (6.8,2.7) node[midway,above] {\scriptsize 帧2};
            \draw[<-,>=Stealth,green!60!black,thick,dashed] (0.2,2.2) -- (6.8,2.2) node[midway,above] {\scriptsize ACK2};
        \end{scope}

        \begin{scope}[shift={(0,-5.2)}]
            \node[anchor=east] at (0,3.0) {\small 发送端};
            \node[anchor=west] at (7,3.0) {\small 接收端};
            \draw[gray!50] (0.2,1.0) -- (0.2,4.0);
            \draw[gray!50] (6.8,1.0) -- (6.8,4.0);
            \node[anchor=west] at (-0.2,4.0) {\small (c) 面向连接};
            % setup
            \draw[->,>=Stealth,orange!70!black,thick] (0.3,3.8) -- (6.8,3.8) node[midway,above] {\scriptsize 建立(SETUP)};
            \draw[<-,>=Stealth,orange!70!black,thick] (0.2,3.3) -- (6.8,3.3) node[midway,above] {\scriptsize 确认(CONN)};
            % data with seq
            \draw[->,>=Stealth,blue!70,thick] (0.2,2.8) -- (6.8,2.8) node[midway,above] {\scriptsize 帧(seq=1)};
            \draw[<-,>=Stealth,green!60!black,thick,dashed] (0.2,2.4) -- (6.8,2.4) node[midway,above] {\scriptsize ACK1};
            \draw[->,>=Stealth,blue!70,thick] (0.2,1.8) -- (6.8,1.8) node[midway,above] {\scriptsize 帧(seq=2)};
            % release
            \draw[->,>=Stealth,gray!70,thick] (0.2,1.2) -- (6.8,1.2) node[midway,above] {\scriptsize 释放(RELEASE)};
        \end{scope}
    \end{tikzpicture}%
    }
    \caption{链路层对网络层的三类服务:无确认无连接、带确认无连接、面向连接}
    \label{fig:dl-service-types}
\end{figure}

\paragraph{对比与适用}
{\small\begin{longtable}{|p{0.22\linewidth}|p{0.25\linewidth}|p{0.25\linewidth}|p{0.25\linewidth}|}
\hline
    \textbf{服务类型} & \textbf{连接与确认} & \textbf{可靠性保证} & \textbf{典型场景/特点} \\
\hline
无确认无连接 & 无连接、无确认 & 不保证有序、不重传 & 以太网;\textbf{开销小、时延低},可靠性交由高层 \\
\hline
带确认无连接 & 无连接、有ACK & 单跳可靠、按帧次序 & 802.11;适合\textbf{无线差错高}链路,\textbf{单跳可靠} \\
\hline
面向连接 & 建连、序号确认 & 有序交付、流量控制 & 点到点可靠链路(HDLC/LLC2);\textbf{开销较大} \\
\hline
\end{longtable}}

\paragraph{408 考试提示}
\begin{itemize}
    \item \textbf{分层责任}:链路层的可靠性是\textbf{逐跳}的;端到端可靠性由\textbf{传输层}保证(如 TCP)。
    \item \textbf{易混点}:802.11 有 ACK 但仍是\textbf{无连接};\textbf{建连}不等价于“端到端面向连接”。
    \item \textbf{选择策略}:低差错/低时延链路倾向无连接;高差错链路可用带确认或面向连接以换取可靠性。
\end{itemize}
\subsection{数据链路管理}
\subsection{帧同步}
\subsection{流量控制}
\subsection{差错控制}



\section{组帧}
\subsection{字符计数法}
\subsection{字符填充法}
\subsection{零比特填充法}
\subsection{违法编码法}

\section{差错控制}
\subsection{差错的产生与分类}
\subsection{检错编码}
\subsubsection{奇偶校验码}
\subsubsection{循环冗余码CRC}
\subsection{纠错编码}
\subsubsection{海明码}

\section{流量控制与可靠传输机制}
\subsection{流量控制}
\subsection{停止-等待协议}
\subsection{后退N帧协议}
\subsection{选择重传协议}

\section{介质访问控制}
\subsection{介质访问控制的基本概念}
\subsection{静态划分信道}
\subsubsection{频分多路复用FDM}
\subsubsection{时分多路复用TDM}
\subsubsection{波分多路复用WDM}
\subsubsection{码分多路复用CDM}
\subsection{动态分配信道}
\subsubsection{ALOHA协议}
\subsubsection{CSMA协议}
\subsubsection{CSMA/CD协议}
\subsubsection{CSMA/CA协议}


\section{局域网}
\subsection{局域网的基本概念和体系结构}
\subsection{以太网}
\subsubsection{以太网的发展}
\subsubsection{以太网的MAC帧格式}
\subsubsection{高速以太网}
\subsection{无线局域网}
\subsubsection{IEEE 802.11}
\subsubsection{无线局域网的组成}
\subsubsection{虚拟局域网VLAN}

\section{广域网}
\subsection{广域网的基本概念}
\subsection{PPP协议}
\subsection{PPP协议}
\subsection{PPP的帧格式}
\subsection{PPP协议的工作状态}
\subsection{HDLC协议}


\section{数据链路层设备}
\subsection{网桥}
\subsection{局域网交换机}

% ========== 第4章 网络层 ==========
\chapter{网络层}

\section{网络层的功能}
\subsection{异构网络互连}
\subsection{路由与转发}
\subsection{拥塞控制}

\section{路由算法}
\subsection{静态路由与动态路由}
\subsection{距离向量路由算法}
\subsection{链路状态路由算法}
\subsection{层次路由}

\section{IPv4}
\subsection{IPv4地址}
\subsubsection{IPv4地址格式}
\subsubsection{分类编址}
\subsubsection{无分类编址CIDR}
\subsubsection{特殊IP地址}
\subsection{IPv4数据报格式}
\subsection{IPv4数据报的分片与重组}

\section{IPv6}
\subsection{IPv6的特点}
\subsection{IPv6地址}
\subsection{IPv6数据报格式}
\subsection{从IPv4到IPv6的过渡}

\section{路由协议}
\subsection{自治系统}
\subsection{域内路由协议}
\subsubsection{RIP协议}
\subsubsection{OSPF协议}
\subsection{域间路由协议}
\subsubsection{BGP协议}

\section{网际控制协议ICMP}
\subsection{ICMP的基本概念}
\subsection{ICMP报文格式}
\subsection{ICMP的常见类型}
\subsection{ICMP的应用}

\section{虚拟专用网和网络地址转换NAT}

\section{IP多播}
\subsection{IP多播的基本概念}
\subsection{IP多播地址}
\subsection{IGMP协议}
\subsection{多播路由算法}

\section{移动IP}
\subsection{移动IP的基本概念}
\subsection{移动IP的工作原理}

\section{软件定义网络SDN}

\section{网络层设备}
\subsection{路由器}
\subsection{三层交换机}

% ========== 第5章 传输层 ==========
\chapter{传输层}

\section{传输层提供的服务}
\subsection{传输层的功能}
\subsection{传输层的寻址与端口}
\subsection{无连接服务与面向连接服务}

\section{UDP协议}
\subsection{UDP的特点}
\subsection{UDP报文段格式}
\subsection{UDP校验}

\section{TCP协议}
\subsection{TCP的特点}
\subsection{TCP报文段格式}
\subsection{TCP连接管理}
\subsubsection{TCP连接建立}
\subsubsection{TCP连接释放}
\subsubsection{TCP有限状态机}
\subsection{TCP可靠传输}
\subsubsection{序号与确认号}
\subsubsection{重传机制}
\subsubsection{滑动窗口}
\subsection{TCP流量控制}
\subsection{TCP拥塞控制}
\subsubsection{慢开始}
\subsubsection{拥塞避免}
\subsubsection{快重传}
\subsubsection{快恢复}

% ========== 第6章 应用层 ==========
\chapter{应用层}

\section{网络应用模型}
\subsection{客户/服务器模型}
\subsection{P2P模型}

\section{DNS系统}
\subsection{域名系统的层次结构}
\subsection{域名解析过程}
\subsection{DNS报文格式}

\section{FTP协议}
\subsection{FTP的工作原理}
\subsection{FTP连接}
\subsection{FTP数据传输模式}

\section{电子邮件}
\subsection{电子邮件系统组成}
\subsection{简单邮件传输协议SMTP}
\subsection{邮局协议POP3}
\subsection{网际邮件访问协议IMAP}
\subsection{电子邮件格式与MIME}

\section{万维网WWW}
\subsection{万维网的概念与组成}
\subsection{超文本传输协议HTTP}
\subsubsection{HTTP的特点}
\subsubsection{HTTP报文格式}
\subsubsection{HTTP状态码}
\subsubsection{Cookie与Session}
\subsection{万维网缓存与代理服务器}

% ========== 附录与总结 ==========
\chapter{网络安全基础}

\section{网络安全概述}
\subsection{网络安全的基本概念}
\subsection{网络攻击与威胁}
\subsection{网络安全策略}

\section{加密技术}
\subsection{对称加密}
\subsection{非对称加密}
\subsection{数字签名}
\subsection{报文摘要}

\section{认证技术}
\subsection{身份认证}
\subsection{数字证书}
\subsection{公钥基础设施PKI}

\section{网络安全协议}
\subsection{IPSec}
\subsection{SSL/TLS}
\subsection{HTTPS}

\section{防火墙技术}
\subsection{防火墙的基本概念}
\subsection{防火墙的分类}
\subsection{防火墙的配置策略}

\chapter{408真题考点总结}

\section{历年高频考点}
\subsection{OSI模型与TCP/IP模型}
\subsection{以太网与CSMA/CD}
\subsection{IP地址与子网划分}
\subsection{路由算法与路由协议}
\subsection{TCP协议特性}
\subsection{DNS域名解析}
\subsection{HTTP协议}

\section{重点计算题型}
\subsection{子网划分计算}
\subsection{滑动窗口计算}
\subsection{RTT与超时重传}
\subsection{信道利用率计算}
\subsection{CRC校验计算}

\section{应试技巧与答题策略}
\subsection{选择题技巧}
\subsection{综合应用题策略}
\subsection{时间分配建议}
% \end{center}
\end{document}