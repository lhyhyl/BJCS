\documentclass[lang=cn,newtx,10pt,scheme=chinese]{../../elegantbook}

\title{408计算机网络冲刺背诵手册}
\subtitle{北街学长倾力之作}

\author{北街}
\date{2025/08/17}
\version{1.0}


\setcounter{tocdepth}{3}

\logo{../../figure/logo-blue.png}
\cover{../../figure/cover.jpg}

% 本文档命令
\usepackage{array}
\usepackage{longtable}
\usepackage{booktabs}
\newcommand{\ccr}[1]{\makecell{{\color{#1}\rule{1cm}{1cm}}}}

% 修改标题页的橙色带
\definecolor{customcolor}{RGB}{32,178,170}
\colorlet{coverlinecolor}{customcolor}
\usepackage{cprotect}

\addbibresource[location=local]{reference.bib} % 参考文献,不要删除
\usepackage{listings}         % 导入listings宏包
\usepackage{xcolor}           % 支持颜色
\usepackage{graphicx}        % 支持图形
% 配置C++代码样式
\lstset{
    language=C++,             % 语言设置为C++
    basicstyle=\ttfamily,      % 基本样式
    keywordstyle=\color{blue}, % 关键词颜色
    commentstyle=\color[rgb]{0,0.5,0},% 注释颜色  stringstyle=\color{red},   % 字符串颜色
    numbers=left,              % 显示行号
    numberstyle=\tiny,         % 行号样式
    stepnumber=1,              % 每行显示行号
    breaklines=true,           % 自动换行
    frame=lines                % 代码块边框样式
}

\usepackage{tikz}
\usetikzlibrary{chains, positioning, arrows.meta}
\begin{document}

\maketitle
\frontmatter

\tableofcontents

\mainmatter

% ================== 408计算机网络冲刺背诵手册 ==================
% 根据典型 408 大纲结构生成章节骨架,逐步填充 PDF 原文内容

% ========== 第1章 计算机网络概述 ==========
\chapter{计算机网络概述}

\section{计算机网络的定义、功能和分类}

\textbf{定义:}互连的、自主的计算机集合,通过通信线路连接,实现资源共享和信息传递。

\textbf{主要功能:}
\begin{itemize}
  \item \textbf{数据通信}:点对点数据传输
  \item \textbf{资源共享}:硬件、软件、数据共享
  \item \textbf{分布式处理}:多台计算机协同处理任务
  \item \textbf{提高可靠性}:备用设备和冗余路径
  \item \textbf{负载均衡}:任务分配优化
\end{itemize}

\textbf{分类方式:}
\begin{longtable}{@{}llp{6cm}@{}}
\toprule
\textbf{分类标准} & \textbf{类型} & \textbf{特点} \\\\ \midrule
\endhead

\multirow{4}{*}{覆盖范围} & PAN (个域网) & 10m以内,蓝牙、红外 \\\\
& LAN (局域网) & 几km,以太网 \\\\
& MAN (城域网) & 10-100km,城市范围 \\\\
& WAN (广域网) & 100km以上,Internet \\\\
\hline

\multirow{3}{*}{拓扑结构} & 总线型 & 共享总线,冲突检测 \\\\
& 星型 & 中心节点,易扩展 \\\\
& 环型 & 令牌传递,无冲突 \\\\
\hline

\multirow{2}{*}{传输技术} & 广播式 & 共享信道,竞争访问 \\\\
& 点对点 & 专用链路,存储转发 \\\\

\bottomrule
\end{longtable}

\section{计算机网络的组成与结构}

\textbf{网络组成三要素:}
\begin{enumerate}
  \item \textbf{硬件}:计算机、通信线路、网络设备
  \item \textbf{软件}:网络操作系统、网络管理软件
  \item \textbf{协议}:通信规则和标准
\end{enumerate}

\textbf{网络结构:}
\begin{itemize}
  \item \textbf{网络边缘}:主机(端系统),运行应用程序
  \item \textbf{接入网}:连接端系统到边缘路由器
  \item \textbf{网络核心}:路由器网状网络,分组交换
\end{itemize}

\section{计算机网络的性能指标}

\subsection{速率相关指标}

\textbf{核心概念对比:}
\begin{longtable}{@{}p{2.5cm}p{5cm}p{6cm}@{}}
\toprule
\textbf{指标} & \textbf{定义} & \textbf{区别} \\\\ \midrule
\endhead

速率(Rate) & 数据传输速率,单位时间传输比特数 & 理论最大值,bps \\\\
带宽(Bandwidth) & 信道能通过的最高数据率 & 物理概念,Hz或bps \\\\
吞吐量(Throughput) & 实际单位时间传输的数据量 & 实际测量值 ≤ 带宽 \\\\

\bottomrule
\end{longtable}

\subsection{时延相关指标}

\textbf{四种时延组成:}
\begin{align}
\text{总时延} &= \text{发送时延} + \text{传播时延} + \text{处理时延} + \text{排队时延}
\end{align}

\begin{itemize}
  \item \textbf{发送时延} = $\frac{L}{R}$ (L:分组长度,R:发送速率)
  \item \textbf{传播时延} = $\frac{d}{s}$ (d:物理距离,s:传播速率)
  \item \textbf{处理时延}:路由器处理分组的时间
  \item \textbf{排队时延}:分组在队列中等待的时间
\end{itemize}

\textbf{重要概念:}
\begin{itemize}
  \item \textbf{时延带宽积} = 传播时延 × 带宽("管道"中的比特数)
  \item \textbf{往返时间RTT}:从发送方到接收方再返回的时间
\end{itemize}

\subsection{其他性能指标}

\begin{itemize}
  \item \textbf{信道利用率}:$U = \frac{T_0}{T_0 + RTT}$(停止等待协议)
  \item \textbf{网络利用率}:信道利用率加权平均值
  \item \textbf{丢包率}:丢失分组数/总分组数
  \item \textbf{误码率BER}:传输中出错的比特占总比特数的比率
\end{itemize}

\textbf{记忆要点:}
\begin{enumerate}
  \item 发送时延看发送方,传播时延看距离
  \item 利用率过高导致时延急剧增加
  \item 时延带宽积 = 链路容量
\end{enumerate}

\section{计算机网络体系结构}

\subsection{网络协议与分层}

\textbf{协议三要素:}
\begin{enumerate}
  \item \textbf{语法}:数据格式、编码、信号电平
  \item \textbf{语义}:控制信息的含义,差错处理
  \item \textbf{同步}:事件顺序的详细说明
\end{enumerate}

\textbf{分层的优点:}
\begin{itemize}
  \item 降低复杂性,便于设计和维护
  \item 标准化接口,促进互操作性
  \item 模块化设计,便于更新和扩展
\end{itemize}

\subsection{OSI七层模型}

\begin{longtable}{@{}clp{4cm}p{5cm}@{}}
\toprule
\textbf{层次} & \textbf{名称} & \textbf{主要功能} & \textbf{关键设备/协议} \\\\ \midrule
\endhead

7 & 应用层 & 为应用程序提供网络服务 & HTTP, FTP, SMTP, DNS \\\\
6 & 表示层 & 数据加密、压缩、格式转换 & SSL/TLS, JPEG, MPEG \\\\
5 & 会话层 & 建立、管理、终止会话 & NetBIOS, RPC \\\\
4 & 传输层 & 端到端可靠传输 & TCP, UDP \\\\
3 & 网络层 & 路径选择,逻辑寻址 & IP, ICMP, 路由器 \\\\
2 & 数据链路层 & 成帧,差错检测与纠正 & 以太网, PPP, 交换机 \\\\
1 & 物理层 & 比特流传输 & 集线器, 中继器 \\\\

\bottomrule
\end{longtable}

\subsection{TCP/IP四层模型}

\begin{longtable}{@{}p{3cm}p{4cm}p{6cm}@{}}
\toprule
\textbf{TCP/IP层} & \textbf{对应OSI层} & \textbf{主要协议} \\\\ \midrule
\endhead

应用层 & 应用层+表示层+会话层 & HTTP, FTP, SMTP, DNS, Telnet \\\\
传输层 & 传输层 & TCP, UDP \\\\
网际层 & 网络层 & IP, ICMP, ARP, RARP \\\\
网络接口层 & 数据链路层+物理层 & 以太网, PPP, SLIP \\\\

\bottomrule
\end{longtable}

\textbf{记忆口诀:}
\begin{itemize}
  \item OSI:\textbf{物数网传会表应}
  \item 各层PDU:比特 → 帧 → 分组/包 → 段 → 报文
\end{itemize}

% ========== 第2章 物理层 ==========
\chapter{物理层}

\section{物理层基本概念}

\textbf{物理层任务:}在物理媒体上传输比特流,定义电气、机械、功能、过程特性。

\textbf{四个特性:}
\begin{enumerate}
  \item \textbf{机械特性}:接口形状、尺寸、引脚数
  \item \textbf{电气特性}:电压范围、传输速率、距离限制
  \item \textbf{功能特性}:引脚的功能分配
  \item \textbf{过程特性}:建立连接的工作步骤
\end{enumerate}

\section{数据通信基础}

\subsection{基本术语}

\textbf{通信基础概念:}
\begin{itemize}
  \item \textbf{数据}:传送信息的实体
  \item \textbf{信号}:数据的电气/光学表现
    \begin{itemize}
      \item 模拟信号:连续变化
      \item 数字信号:离散变化
    \end{itemize}
  \item \textbf{信道}:传输信号的通道
    \begin{itemize}
      \item 单工:单向传输(广播)
      \item 半双工:双向交替传输(对讲机)
      \item 全双工:双向同时传输(电话)
    \end{itemize}
  \item \textbf{信源}:产生和发送数据的源点
  \item \textbf{信宿}:接收数据的终点
\end{itemize}

\textbf{重要术语:}
\begin{itemize}
  \item \textbf{码元}:在数字通信中时间轴上的一个信号单元
  \item \textbf{波特率}:每秒传输的码元个数,单位Baud
  \item \textbf{比特率}:每秒传输的比特个数,单位bps
  \item \textbf{关系}:比特率 = 波特率 × $\log_2V$(V为每个码元可能的离散值个数)
\end{itemize}

\textbf{编码与调制:}
\begin{itemize}
  \item \textbf{编码}:数据变换为数字信号
    \begin{itemize}
      \item 不归零编码(NRZ):1用高电平,0用低电平
      \item 归零编码(RZ):信号在码元结束前归零
      \item 曼彻斯特编码:位中间有跳变,自同步
      \item 差分曼彻斯特编码:位开始有无跳变表示数据
    \end{itemize}
  \item \textbf{调制}:数据变换为模拟信号
    \begin{itemize}
      \item 调幅(ASK):改变载波幅度
      \item 调频(FSK):改变载波频率  
      \item 调相(PSK):改变载波相位
      \item 正交调幅(QAM):同时调幅和调相
    \end{itemize}
\end{itemize}

\textbf{三种交换方式:}
\begin{longtable}{@{}p{3cm}p{4cm}p{6cm}@{}}
\toprule
\textbf{交换方式} & \textbf{特点} & \textbf{应用场景} \\\\ \midrule
\endhead

电路交换 & 建立专用物理连接 & 传统电话网络 \\\\
& 三阶段:建立、通信、释放 & 实时通信 \\\\
\hline

报文交换 & 整个报文存储转发 & 早期数据网络 \\\\
& 存储转发延迟大 & 电子邮件 \\\\
\hline

分组交换 & 将报文分割成分组 & 现代数据网络 \\\\
& 存储转发延迟小 & Internet \\\\

\bottomrule
\end{longtable}

\textbf{数据报vs虚电路:}
\begin{itemize}
  \item \textbf{数据报}:每个分组独立路由,无连接
  \item \textbf{虚电路}:预先建立路径,面向连接
\end{itemize}

\subsection{奈奎斯特定理和香农定理}

\textbf{奈奎斯特定理(理想信道):}
\[
C = 2W \log_2 V \text{ (bps)}
\]
其中:W=带宽(Hz),V=信号电平数

\textbf{香农定理(实际信道):}
\[
C = W \log_2(1 + S/N) \text{ (bps)}
\]
其中:S/N=信噪比,单位dB时:$S/N_{dB} = 10\log_{10}(S/N)$

\textbf{记忆要点:}
\begin{itemize}
  \item 奈奎斯特:理想无噪声,主要看电平数
  \item 香农:考虑噪声,实际信道极限
  \item 两者取较小值为实际极限
\end{itemize}

\section{传输介质}

\subsection{有线传输介质}

\begin{longtable}{@{}p{2.5cm}p{4cm}p{6cm}@{}}
\toprule
\textbf{介质类型} & \textbf{特点} & \textbf{应用场景} \\\\ \midrule
\endhead

双绞线 & 抗干扰能力强,成本低 & 局域网,电话线 \\\\
& 5类线:100Mbps & \\\\
& 6类线:1Gbps & \\\\
\hline

同轴电缆 & 带宽大,抗干扰强 & 有线电视,早期以太网 \\\\
& 50Ω:数字传输 & \\\\
& 75Ω:模拟传输 & \\\\
\hline

光纤 & 带宽极大,不受电磁干扰 & 长距离高速传输 \\\\
& 单模:长距离 & 主干网络 \\\\
& 多模:短距离 & 建筑物内部 \\\\

\bottomrule
\end{longtable}

\subsection{无线传输介质}
\begin{itemize}
  \item \textbf{无线电波}:长距离,穿透性强,全向传播
  \item \textbf{微波}:视距传播,点对点通信
  \item \textbf{红外线}:短距离,需要视距,安全性好
  \item \textbf{激光}:高带宽,方向性强,易受干扰
\end{itemize}

\section{信道复用技术}

\textbf{复用目的:}提高信道利用率,多个用户共享信道

\begin{longtable}{@{}p{3cm}p{5cm}p{5cm}@{}}
\toprule
\textbf{复用方式} & \textbf{原理} & \textbf{特点} \\\\ \midrule
\endhead

频分复用FDM & 不同频率载波传输 & 模拟通信,频谱分割 \\\\
时分复用TDM & 不同时间片传输 & 数字通信,时间分割 \\\\
统计时分复用STDM & 按需分配时间片 & 提高效率,复杂控制 \\\\
波分复用WDM & 不同波长光信号 & 光纤通信,容量巨大 \\\\
码分复用CDM & 不同编码序列 & CDMA,抗干扰强 \\\\

\bottomrule
\end{longtable}

\textbf{记忆技巧:}
\begin{itemize}
  \item FDM:频率分开用
  \item TDM:时间轮流用  
  \item WDM:波长分开用
  \item CDM:编码区分用
\end{itemize}

\section{物理层设备}

\textbf{中继器(Repeater):}
\begin{itemize}
  \item \textbf{功能}:放大和重新整形信号,延长传输距离
  \item \textbf{工作层次}:物理层
  \item \textbf{特点}:
    \begin{itemize}
      \item 两端口设备,连接同种传输媒体
      \item 不理解帧、分组等高层概念
      \item 简单的信号放大,不检查数据内容
      \item 延长单段传输距离,但不增加总带宽
    \end{itemize}
  \item \textbf{限制}:5-4-3规则(以太网中最多5段,4个中继器,3段有主机)
\end{itemize}

\textbf{集线器(Hub):}
\begin{itemize}
  \item \textbf{功能}:多端口中继器,星型拓扑中心
  \item \textbf{工作层次}:物理层
  \item \textbf{特点}:
    \begin{itemize}
      \item 所有端口共享总带宽
      \item 半双工工作方式
      \item 单一冲突域和广播域
      \item 使用CSMA/CD协议
    \end{itemize}
  \item \textbf{缺点}:
    \begin{itemize}
      \item 冲突域大,效率低
      \item 不能隔离冲突
      \item 所有端口必须同速率
      \item 已基本被交换机取代
    \end{itemize}
\end{itemize}

\textbf{中继器vs集线器:}
\begin{longtable}{@{}p{3cm}p{5cm}p{5cm}@{}}
\toprule
\textbf{设备} & \textbf{中继器} & \textbf{集线器} \\\\ \midrule
\endhead

端口数量 & 2个 & 多个(4-24) \\\\
拓扑结构 & 总线延长 & 星型中心 \\\\
冲突域 & 1个 & 1个 \\\\
主要用途 & 距离延长 & 网络集中 \\\\

\bottomrule
\end{longtable}



% ========== 第3章 数据链路层 ==========
\chapter{数据链路层}

\section{数据链路层基本概念}

\textbf{数据链路层功能:}
\begin{enumerate}
  \item \textbf{封装成帧}:在网络层分组前后添加帧头和帧尾
  \item \textbf{透明传输}:保证任何比特组合都能传输
  \item \textbf{差错检测}:检测传输中的差错
  \item \textbf{差错纠正}:纠正某些传输差错
  \item \textbf{流量控制}:防止接收方缓存溢出
\end{enumerate}

\textbf{三个基本问题:}
\begin{itemize}
  \item \textbf{封装成帧}:确定帧的边界
  \item \textbf{透明传输}:解决帧定界符在数据中出现的问题
  \item \textbf{差错检测}:接收方检测传输差错
\end{itemize}

\section{封装成帧}

\textbf{帧的结构:}
\begin{center}
\begin{tabular}{|c|c|c|}
\hline
帧头 & 数据部分(IP数据报) & 帧尾 \\
\hline
\end{tabular}
\end{center}

\textbf{帧定界方法:}
\begin{enumerate}
  \item \textbf{字符计数法}:帧头指明帧长度
  \item \textbf{字符填充法}:使用特殊字符作为定界符
  \item \textbf{零比特填充法}:连续5个1后插入0
  \item \textbf{违法编码法}:使用物理层违法信号
\end{enumerate}

\textbf{透明传输:}
确保数据中任意比特组合都不会被误认为控制信息
\begin{itemize}
  \item \textbf{字符填充}:遇到定界符时前面加转义字符
  \item \textbf{比特填充}:遇到连续5个1时插入0
\end{itemize}

\section{差错控制}

\subsection{差错检测}

\textbf{差错产生原因:}
\begin{itemize}
  \item 噪声干扰、设备故障、传输衰减
  \item 误码率BER:出错比特数/传输总比特数
\end{itemize}

\textbf{奇偶校验:}
\begin{itemize}
  \item \textbf{简单奇偶校验}:检测奇数个比特错误
  \item \textbf{二维奇偶校验}:按行列分别校验,可定位错误
\end{itemize}

\textbf{循环冗余校验CRC:}
\begin{enumerate}
  \item 选择生成多项式G(x),k个数据位,r个校验位
  \item 发送方:将数据左移r位,除以G(x)得余数作为CRC
  \item 接收方:收到数据除以G(x),余数为0则无错
\end{enumerate}

\textbf{CRC特点:}
\begin{itemize}
  \item 检错能力强,漏检概率为$2^{-r}$
  \item 能检测出所有奇数位错误
  \item 能检测出长度≤r的突发错误
  \item 只能检错,不能纠错
\end{itemize}

\subsection{差错纠正}

\textbf{海明码:}
\begin{itemize}
  \item \textbf{海明距离}:两个等长字符串对应位不同的个数
  \item \textbf{纠错能力}:纠正t个错误需要海明距离≥2t+1
  \item \textbf{检错能力}:检测t个错误需要海明距离≥t+1
\end{itemize}

\textbf{海明码编码规则:}
\begin{enumerate}
  \item 校验位位置:$2^0, 2^1, 2^2, ...$
  \item 数据位:其余位置
  \item 每个校验位负责特定位置的奇偶性
\end{enumerate}

\section{流量控制与可靠传输}

\subsection{停止等待协议}

\textbf{工作原理:}
\begin{enumerate}
  \item 发送方发送一帧后停止,等待确认
  \item 收到ACK后发送下一帧
  \item 超时则重传当前帧
\end{enumerate}

\textbf{自动重传请求ARQ:}
\begin{itemize}
  \item \textbf{确认机制}:接收方发送ACK确认
  \item \textbf{超时重传}:发送方设置定时器
  \item \textbf{编号机制}:数据帧和ACK帧编号
\end{itemize}

\textbf{效率分析:}
\[
\text{信道利用率} = \frac{T_0}{T_0 + RTT}
\]
其中$T_0$为发送时间,RTT为往返时间

\subsection{滑动窗口协议}

\textbf{基本概念:}
\begin{itemize}
  \item \textbf{发送窗口}:发送方可连续发送的帧数
  \item \textbf{接收窗口}:接收方可接收的帧数
  \item \textbf{窗口滑动}:确认后窗口向前移动
\end{itemize}

\textbf{后退N帧协议(GBN):}
\begin{itemize}
  \item 发送窗口:1~$W_T$,接收窗口:1
  \item 累积确认:确认序号n表示n之前的帧都正确接收
  \item 出错处理:重传出错帧及其后所有帧
  \item 序号范围:$W_T + 1 \leq 2^k$
\end{itemize}

\textbf{选择重传协议(SR):}
\begin{itemize}
  \item 发送窗口:$W_T$,接收窗口:$W_R$
  \item 单独确认:每帧单独确认
  \item 出错处理:只重传出错帧
  \item 序号范围:$W_T + W_R \leq 2^k$
\end{itemize}

\textbf{协议对比:}
\begin{longtable}{@{}p{3cm}p{5cm}p{5cm}@{}}
\toprule
\textbf{特性} & \textbf{GBN} & \textbf{SR} \\\\ \midrule
\endhead

接收窗口 & 1 & $W_R \geq 1$ \\\\
确认方式 & 累积确认 & 单独确认 \\\\
重传策略 & 回退重传 & 选择重传 \\\\
复杂度 & 较简单 & 较复杂 \\\\
效率 & 较低 & 较高 \\\\

\bottomrule
\end{longtable}

\section{介质访问控制}

\textbf{问题:}多个节点共享同一广播信道,如何协调访问?

\subsection{信道划分}

\textbf{频分多址FDMA:}
\begin{itemize}
  \item 将信道按频率划分给不同用户
  \item 无冲突,但频谱利用率低
\end{itemize}

\textbf{时分多址TDMA:}
\begin{itemize}
  \item 将信道按时间划分给不同用户
  \item 需要严格同步,有时隙浪费
\end{itemize}

\textbf{码分多址CDMA:}
\begin{itemize}
  \item 每用户分配唯一的码序列
  \item 抗干扰能力强,频谱利用率高
  \item 应用:3G移动通信
\end{itemize}

\subsection{随机访问}

\textbf{ALOHA协议:}
\begin{itemize}
  \item \textbf{纯ALOHA}:想发就发,冲突后随机延时重发
    \begin{itemize}
      \item 最大吞吐量:18.4\%
      \item 冲突检测:通过确认机制
    \end{itemize}
  \item \textbf{时隙ALOHA}:将时间分成时隙,只能在时隙开始发送
    \begin{itemize}
      \item 最大吞吐量:36.8\%
      \item 减少了冲突概率
    \end{itemize}
\end{itemize}

\textbf{载波监听多路访问CSMA:}
\begin{itemize}
  \item \textbf{基本思想}:发送前先监听信道
  \item \textbf{1-坚持CSMA}:信道空闲立即发送,忙则持续监听
  \item \textbf{非坚持CSMA}:信道忙则等待随机时间再监听
  \item \textbf{p-坚持CSMA}:信道空闲以概率p发送
\end{itemize}

\textbf{CSMA/CD(载波监听/冲突检测):}
\begin{enumerate}
  \item \textbf{发送前}:载波监听,信道空闲才发送
  \item \textbf{发送中}:边发送边监听,检测冲突
  \item \textbf{冲突处理}:停止发送,发送干扰信号
  \item \textbf{退避算法}:二进制指数退避
\end{enumerate}

\textbf{最小帧长要求:}
\[
L_{min} = 2 \times \tau \times R
\]
其中$\tau$为单程传播时延,R为发送速率

\textbf{CSMA/CA(载波监听/冲突避免):}
\begin{itemize}
  \item 用于无线网络(难以检测冲突)
  \item 预约信道:RTS/CTS机制
  \item 确认机制:ACK确认
  \item 应用:WiFi (IEEE 802.11)
\end{itemize}

\subsection{轮流协议}

\textbf{令牌传递:}
\begin{itemize}
  \item 令牌按固定顺序在节点间传递
  \item 持有令牌的节点可发送数据
  \item 应用:令牌环网、FDDI
\end{itemize}

\textbf{轮询:}
\begin{itemize}
  \item 主节点轮询各从节点
  \item 被轮询节点可发送数据
  \item 开销:轮询时延、等待时延
\end{itemize}

\section{局域网}

\subsection{局域网基本概念}

\textbf{局域网特点:}
\begin{itemize}
  \item 覆盖范围小(几km以内)
  \item 数据传输速率高(10Mbps~10Gbps)
  \item 误码率低($10^{-8}$~$10^{-12}$)
  \item 各站点地位平等
  \item 共享传输媒体
\end{itemize}

\textbf{局域网拓扑结构:}
\begin{itemize}
  \item \textbf{总线型}:所有节点连在一条总线上
  \item \textbf{星型}:所有节点连接到中央节点
  \item \textbf{环型}:节点连成环状结构
  \item \textbf{树型}:层次结构,有根节点
\end{itemize}

\subsection{以太网}

\textbf{IEEE 802.3标准特点:}
\begin{itemize}
  \item 使用CSMA/CD访问控制
  \item 无连接、不可靠的服务
  \item 发送前不握手
  \item 不确认接收到的帧
\end{itemize}

\textbf{以太网帧格式:}
\begin{center}
\begin{tabular}{|c|c|c|c|c|c|c|c|}
\hline
前导码 & 帧开始符 & 目的地址 & 源地址 & 类型 & 数据 & 填充 & FCS \\
\hline
7字节 & 1字节 & 6字节 & 6字节 & 2字节 & 46-1500字节 & 0-46字节 & 4字节 \\
\hline
\end{tabular}
\end{center}

\textbf{字段说明:}
\begin{itemize}
  \item \textbf{前导码}:7个字节的10101010,用于同步
  \item \textbf{帧开始符}:10101011,标识帧开始
  \item \textbf{地址字段}:MAC地址,48位
  \item \textbf{类型字段}:上层协议类型(如0x0800表示IP)
  \item \textbf{数据字段}:46-1500字节,不足46字节需填充
  \item \textbf{FCS}:帧校验序列,CRC校验
\end{itemize}

\textbf{MAC地址:}
\begin{itemize}
  \item 48位(6字节),全球唯一
  \item 前24位:厂商标识(OUI)
  \item 后24位:厂商内部编号
  \item 广播地址:FF-FF-FF-FF-FF-FF
  \item 组播地址:第一字节最低位为1
\end{itemize}

\textbf{以太网发展:}
\begin{longtable}{@{}p{3cm}p{4cm}p{6cm}@{}}
\toprule
\textbf{标准} & \textbf{速率} & \textbf{特点} \\\\ \midrule
\endhead

10BASE-T & 10 Mbps & 双绞线,星型拓扑 \\\\
100BASE-TX & 100 Mbps & 快速以太网,保持帧格式 \\\\
1000BASE-T & 1 Gbps & 千兆以太网,4对双绞线 \\\\
10GBASE-T & 10 Gbps & 万兆以太网,光纤为主 \\\\

\bottomrule
\end{longtable}

\subsection{无线局域网}

\textbf{IEEE 802.11标准:}
\begin{itemize}
  \item 使用CSMA/CA访问控制
  \item 隐藏终端问题:A能听到B,C能听到B,但A听不到C
  \item 暴露终端问题:不必要的等待
\end{itemize}

\textbf{RTS/CTS机制:}
\begin{enumerate}
  \item 发送方先发RTS(请求发送)
  \item 接收方回应CTS(允许发送)
  \item 其他节点听到CTS后保持静默
  \item 发送数据,接收方发送ACK
\end{enumerate}

\textbf{WiFi标准演进:}
\begin{longtable}{@{}p{2.5cm}p{4cm}p{6cm}@{}}
\toprule
\textbf{标准} & \textbf{速率} & \textbf{频段} \\\\ \midrule
\endhead

802.11 & 2 Mbps & 2.4 GHz \\\\
802.11b & 11 Mbps & 2.4 GHz \\\\
802.11g & 54 Mbps & 2.4 GHz \\\\
802.11n & 600 Mbps & 2.4/5 GHz \\\\
802.11ac & 6.93 Gbps & 5 GHz \\\\
802.11ax & 9.6 Gbps & 2.4/5 GHz \\\\

\bottomrule
\end{longtable}

\section{广域网}

\textbf{广域网特点:}
\begin{itemize}
  \item 覆盖范围广(几十到几万公里)
  \item 通常使用点对点链路
  \item 主要解决路由和流量控制问题
\end{itemize}

\textbf{点对点协议PPP:}

\textbf{PPP协议特点:}
\begin{itemize}
  \item 面向字节,异步串行
  \item 不可靠传输(无确认机制)
  \item 支持多种网络层协议
  \item 有差错检测功能
\end{itemize}

\textbf{PPP帧格式:}
\begin{center}
\begin{tabular}{|c|c|c|c|c|c|c|}
\hline
标志字段 & 地址字段 & 控制字段 & 协议字段 & 信息字段 & FCS & 标志字段 \\
\hline
7E & FF & 03 & 2字节 & 可变长 & 2字节 & 7E \\
\hline
\end{tabular}
\end{center}

\textbf{PPP工作状态:}
\begin{enumerate}
  \item \textbf{静止状态}:物理层不可用
  \item \textbf{建立状态}:建立链路,配置选项
  \item \textbf{认证状态}:身份认证(可选)
  \item \textbf{网络状态}:进行网络层配置
  \item \textbf{打开状态}:链路可用,传输数据
  \item \textbf{终止状态}:关闭链路
\end{enumerate}

\textbf{高级数据链路控制HDLC:}
\begin{itemize}
  \item 面向比特的协议
  \item 使用零比特填充法
  \item 支持半双工和全双工
  \item 主要用于同步串行线路
\end{itemize}

\section{数据链路层设备}

\subsection{网桥}

\textbf{网桥功能:}
\begin{itemize}
  \item 连接两个或多个局域网
  \item 存储转发工作方式
  \item 根据MAC地址表转发帧
  \item 可以隔离冲突域
\end{itemize}

\textbf{透明网桥工作原理:}
\begin{enumerate}
  \item \textbf{学习}:根据源MAC地址学习站点位置
  \item \textbf{转发}:查表决定转发端口
  \item \textbf{过滤}:同一段内通信不转发
  \item \textbf{老化}:定期删除过期表项
\end{enumerate}


\subsection{以太网交换机}

\textbf{交换机特点:}
\begin{itemize}
  \item 多端口网桥,每端口一个冲突域
  \item 全双工工作,无冲突
  \item 支持不同速率端口
  \item 可构成星型拓扑
\end{itemize}

\textbf{交换方式:}
\begin{itemize}
  \item \textbf{存储转发}:接收完整帧后检查再转发
  \item \textbf{直通交换}:读取目的地址后立即转发
  \item \textbf{无片段交换}:读取64字节后开始转发
\end{itemize}

\textbf{生成树协议STP:}
\begin{itemize}
  \item 目的:消除网络中的环路
  \item 方法:构造生成树,阻塞部分端口
  \item 根网桥:优先级最高的交换机
  \item 端口状态:根端口、指定端口、阻塞端口
\end{itemize}

\textbf{VLAN虚拟局域网:}
\begin{itemize}
  \item 逻辑分割物理网络
  \item 同一VLAN内可通信,不同VLAN间需路由
  \item 实现方式:端口VLAN、MAC VLAN、协议VLAN
  \item 标准:IEEE 802.1Q(VLAN标记)
\end{itemize}

\textbf{记忆要点:}
\begin{itemize}
  \item 数据链路层负责相邻节点间的可靠传输
  \item CRC只能检错不能纠错,海明码可以纠错
  \item CSMA/CD用于有线网络,CSMA/CA用于无线网络
  \item 以太网帧最小64字节,数据部分最少46字节
  \item 交换机学习MAC地址,路由器学习IP地址
\end{itemize}

% ========== 第4章 网络层 ==========
\chapter{网络层}

\section{网络层概述}

\textbf{网络层主要功能:}
\begin{enumerate}
  \item \textbf{路由选择}:确定分组从源到目的的最佳路径
  \item \textbf{分组转发}:根据路由表转发分组到下一跳
  \item \textbf{异构网络互连}:连接不同的网络
  \item \textbf{拥塞控制}:防止网络过载
\end{enumerate}

\textbf{网络层设计思路:}
\begin{itemize}
  \item \textbf{网络层向上提供的服务}:无连接、尽力而为
  \item \textbf{网络层核心功能}:转发和路由
  \item \textbf{路由器功能}:存储转发、路由选择
\end{itemize}

\textbf{SDN基本概念:}
\begin{itemize}
  \item \textbf{软件定义网络}:控制平面与数据平面分离
  \item \textbf{核心思想}:网络可编程、集中控制、开放接口
  \item \textbf{架构组成}:
    \begin{itemize}
      \item 应用层:网络应用程序
      \item 控制层:SDN控制器(大脑)
      \item 基础设施层:SDN交换机(执行)
    \end{itemize}
  \item \textbf{优势}:灵活配置、统一管理、快速创新
  \item \textbf{协议}:OpenFlow协议(控制器与交换机通信)
\end{itemize}

\section{网络层提供的两种服务}

\begin{longtable}{@{}p{3cm}p{5cm}p{5cm}@{}}
\toprule
\textbf{服务类型} & \textbf{虚电路服务} & \textbf{数据报服务} \\\\ \midrule
\endhead

连接方式 & 面向连接 & 无连接 \\\\
可靠性 & 网络保证可靠性 & 主机保证可靠性 \\\\
路径 & 所有分组同一路径 & 每个分组独立路由 \\\\
分组顺序 & 有序到达 & 可能乱序 \\\\
差错处理 & 网络负责重传 & 端系统负责重传 \\\\
拥塞控制 & 网络负责 & 端系统负责 \\\\
典型应用 & X.25, ATM & Internet (IP) \\\\

\bottomrule
\end{longtable}

\textbf{Internet采用数据报服务的原因:}
\begin{itemize}
  \item 网络核心部分简单、可靠性高
  \item 端系统智能化,差错处理能力强
  \item 能够适应多种应用需求
  \item 网络造价大大降低
\end{itemize}

\section{IPv4}

\subsection{IPv4地址}

\textbf{IPv4地址格式:}
\begin{itemize}
  \item 32位二进制,通常用点分十进制表示
  \item 结构:网络号 + 主机号
  \item 分为A、B、C、D、E五类
\end{itemize}

\textbf{IPv4地址分类:}
\begin{longtable}{@{}p{1.5cm}p{4cm}p{3cm}p{4cm}@{}}
\toprule
\textbf{类别} & \textbf{第一字节范围} & \textbf{网络号位数} & \textbf{网络数量} \\\\ \midrule
\endhead

A类 & 1-126 & 8位 & $2^7-2=126$个 \\\\
B类 & 128-191 & 16位 & $2^{14}=16384$个 \\\\
C类 & 192-223 & 24位 & $2^{21}=2097152$个 \\\\
D类 & 224-239 & \multicolumn{2}{c}{组播地址} \\\\
E类 & 240-255 & \multicolumn{2}{c}{保留地址} \\\\

\bottomrule
\end{longtable}

\textbf{特殊IP地址:}
\begin{itemize}
  \item \textbf{0.0.0.0}:本网络本主机
  \item \textbf{255.255.255.255}:受限广播地址
  \item \textbf{127.x.x.x}:环回地址(loopback)
  \item \textbf{169.254.x.x}:链路本地地址
  \item \textbf{私有地址}:
    \begin{itemize}
      \item A类:10.0.0.0/8
      \item B类:172.16.0.0/12
      \item C类:192.168.0.0/16
    \end{itemize}
\end{itemize}

\textbf{子网划分与CIDR:}
\begin{itemize}
  \item \textbf{子网掩码}:区分网络号和主机号
  \item \textbf{VLSM}:可变长子网掩码
  \item \textbf{CIDR}:无类域间路由,消除传统分类限制
  \item \textbf{CIDR表示法}:192.168.1.0/24
\end{itemize}

\textbf{子网划分计算:}
\begin{align}
\text{子网数} &= 2^n \\\\
\text{每个子网主机数} &= 2^m - 2 \\\\
\text{子网掩码长度} &= 32 - m
\end{align}
其中n为借用的主机位数,m为剩余主机位数

\subsection{地址解析协议ARP}

\textbf{ARP功能:}将IP地址解析为MAC地址

\textbf{ARP工作过程:}
\begin{enumerate}
  \item 主机A要发送给主机B,查找ARP缓存表
  \item 若无对应项,广播ARP请求(Who has IP\_B? Tell IP\_A)
  \item 主机B收到请求,单播ARP应答(IP\_B is at MAC\_B)
  \item 主机A收到应答,更新ARP表,发送数据帧
\end{enumerate}

\textbf{ARP报文格式:}
\begin{center}
\begin{tabular}{|l|c|}
\hline
硬件类型 & 2字节 \\
\hline
协议类型 & 2字节 \\
\hline
硬件地址长度 & 1字节 \\
\hline
协议地址长度 & 1字节 \\
\hline
操作码 & 2字节 \\
\hline
发送方硬件地址 & 6字节 \\
\hline
发送方协议地址 & 4字节 \\
\hline
目标硬件地址 & 6字节 \\
\hline
目标协议地址 & 4字节 \\
\hline
\end{tabular}
\end{center}

\textbf{ARP相关协议:}
\begin{itemize}
  \item \textbf{RARP}:反向地址解析,MAC→IP
  \item \textbf{免费ARP}:检测地址冲突、更新ARP表
  \item \textbf{代理ARP}:路由器代为应答ARP请求
\end{itemize}

\subsection{网络地址转换NAT}

\textbf{NAT产生背景:}
\begin{itemize}
  \item IPv4地址空间不足
  \item 私有网络需要访问Internet
  \item 节约公有IP地址
\end{itemize}

\textbf{NAT工作原理:}
\begin{enumerate}
  \item 内部主机使用私有IP地址
  \item 访问外网时,NAT设备替换源IP为公网IP
  \item 维护NAT转换表记录对应关系
  \item 外网回复时,根据转换表还原私有IP
\end{enumerate}

\textbf{NAT类型:}
\begin{itemize}
  \item \textbf{静态NAT}:一对一固定映射
  \item \textbf{动态NAT}:一对多动态分配
  \item \textbf{NAPT}:端口地址转换,多对一映射
\end{itemize}

\textbf{NAT转换表示例:}
\begin{center}
\begin{tabular}{|c|c|c|c|}
\hline
\textbf{内部IP:端口} & \textbf{外部IP:端口} & \textbf{目标IP:端口} & \textbf{状态} \\
\hline
192.168.1.2:1024 & 202.1.1.1:8001 & 8.8.8.8:53 & 活跃 \\
192.168.1.3:1025 & 202.1.1.1:8002 & 163.1.1.1:80 & 活跃 \\
\hline
\end{tabular}
\end{center}

\textbf{NAT的优缺点:}
\begin{itemize}
  \item \textbf{优点}:节约IP地址、提供安全性、灵活组网
  \item \textbf{缺点}:违反端到端原则、影响某些协议、增加延迟
\end{itemize}

\subsection{动态主机配置协议DHCP}

\textbf{DHCP功能:}
\begin{itemize}
  \item 自动为主机分配IP地址
  \item 配置子网掩码、默认网关、DNS服务器
  \item 统一管理网络配置参数
\end{itemize}

\textbf{DHCP工作过程(四次握手):}
\begin{enumerate}
  \item \textbf{DHCP Discover}:客户端广播发现DHCP服务器
  \item \textbf{DHCP Offer}:服务器提供IP地址配置
  \item \textbf{DHCP Request}:客户端请求使用该配置
  \item \textbf{DHCP ACK}:服务器确认分配
\end{enumerate}

\textbf{DHCP地址分配方式:}
\begin{itemize}
  \item \textbf{静态分配}:为特定MAC地址预留IP
  \item \textbf{动态分配}:从地址池临时分配
  \item \textbf{自动分配}:永久分配,类似静态
\end{itemize}

\textbf{DHCP租约管理:}
\begin{itemize}
  \item \textbf{租约时间}:IP地址的使用期限
  \item \textbf{续约}:租约到期前自动续租
  \item \textbf{释放}:主机主动释放IP地址
  \item \textbf{回收}:租约到期后回收地址
\end{itemize}

\textbf{DHCP中继:}
\begin{itemize}
  \item 跨子网DHCP服务
  \item 路由器转发DHCP报文
  \item 避免每个子网部署DHCP服务器
\end{itemize}

\subsection{IP数据报}

\textbf{IP数据报格式:}
\begin{center}
\small
\begin{tabular}{|c|c|c|c|}
\hline
\multicolumn{4}{|c|}{0 \quad\quad\quad\quad\quad\quad\quad\quad 16 \quad\quad\quad\quad\quad\quad\quad\quad 31} \\
\hline
版本 & 首部长度 & 服务类型 & 总长度 \\
\hline
\multicolumn{2}{|c|}{标识} & 标志 & 片偏移 \\
\hline
生存时间TTL & 协议 & \multicolumn{2}{c|}{首部校验和} \\
\hline
\multicolumn{4}{|c|}{源IP地址} \\
\hline
\multicolumn{4}{|c|}{目的IP地址} \\
\hline
\multicolumn{4}{|c|}{可选字段(长度可变)} \\
\hline
\multicolumn{4}{|c|}{数据部分} \\
\hline
\end{tabular}
\end{center}

\textbf{重要字段说明:}
\begin{itemize}
  \item \textbf{版本}:4位,IPv4=4,IPv6=6
  \item \textbf{首部长度}:4位,以4字节为单位,最小值5
  \item \textbf{总长度}:16位,包括首部和数据,最大65535字节
  \item \textbf{标识}:16位,唯一标识数据报的分片
  \item \textbf{标志}:3位,DF(不分片)、MF(更多分片)
  \item \textbf{片偏移}:13位,以8字节为单位
  \item \textbf{TTL}:8位,跳数限制,每经过路由器减1
  \item \textbf{协议}:8位,指示上层协议(TCP=6,UDP=17)
\end{itemize}

\textbf{IP分片与重组:}

\textbf{分片原因:}
\begin{itemize}
  \item 数据报长度超过网络的MTU(最大传输单元)
  \item 以太网MTU=1500字节,其他网络可能更小
\end{itemize}

\textbf{分片过程:}
\begin{enumerate}
  \item 保持标识字段相同
  \item 除最后一片外,MF=1,最后一片MF=0
  \item 片偏移=前面所有分片数据长度÷8
  \item 除最后一片外,数据长度必须是8的倍数
\end{enumerate}

\textbf{重组过程:}
\begin{itemize}
  \item 目的主机根据标识字段识别分片
  \item 根据片偏移确定分片位置
  \item 直到收到MF=0的分片才开始重组
  \item 重组失败则丢弃所有分片
\end{itemize}

\textbf{分片计算示例:}
原数据报3800字节,MTU=1500字节
\begin{itemize}
  \item 第一片:20字节首部+1480字节数据,片偏移=0
  \item 第二片:20字节首部+1480字节数据,片偏移=1480÷8=185
  \item 第三片:20字节首部+840字节数据,片偏移=2960÷8=370
\end{itemize}

\subsection{网际控制报文协议ICMP}

\textbf{ICMP功能:}
\begin{itemize}
  \item 差错报告:向源主机报告差错情况
  \item 网络探测:测试网络连通性
  \item 不纠正差错,只报告差错
\end{itemize}

\textbf{ICMP报文类型:}
\begin{longtable}{@{}p{1cm}p{4cm}p{7cm}@{}}
\toprule
\textbf{类型} & \textbf{名称} & \textbf{用途} \\\\ \midrule
\endhead

0 & 回送应答 & ping响应 \\\\
3 & 目标不可达 & 网络/主机/端口不可达 \\\\
4 & 源点抑制 & 拥塞控制(已废弃) \\\\
5 & 重定向 & 路由重定向 \\\\
8 & 回送请求 & ping请求 \\\\
11 & 时间超过 & TTL=0或重组超时 \\\\
12 & 参数问题 & IP首部字段错误 \\\\

\bottomrule
\end{longtable}

\textbf{ICMP应用:}
\begin{itemize}
  \item \textbf{ping}:使用回送请求和应答测试连通性
  \item \textbf{traceroute}:利用TTL超时逐跳探测路径
    \begin{enumerate}
      \item 发送TTL=1的数据包,第一个路由器返回超时
      \item 发送TTL=2的数据包,第二个路由器返回超时
      \item 重复直到到达目标主机
    \end{enumerate}
\end{itemize}

\section{IPv6}

\textbf{IPv6产生背景:}
\begin{itemize}
  \item IPv4地址空间不足(43亿个地址)
  \item IPv4首部复杂,处理效率低
  \item 缺乏对实时服务的支持
  \item 安全性不足
\end{itemize}

\textbf{IPv6地址特点:}
\begin{itemize}
  \item \textbf{地址长度}:128位(16字节)
  \item \textbf{地址数量}:$2^{128} \approx 3.4 \times 10^{38}$个
  \item \textbf{表示方法}:冒号十六进制,如2001:0db8:85a3::8a2e:0370:7334
  \item \textbf{地址压缩}:连续的0组可用::表示(只能用一次)
\end{itemize}

\textbf{IPv6地址类型:}
\begin{itemize}
  \item \textbf{单播地址}:一对一通信
  \item \textbf{组播地址}:一对多通信(FF00::/8)
  \item \textbf{任播地址}:一对最近一个通信
  \item \textbf{没有广播地址}:用组播代替
\end{itemize}

\textbf{IPv6数据报格式:}
\begin{itemize}
  \item \textbf{固定首部}:40字节,简化处理
  \item \textbf{扩展首部}:可选,提供灵活性
  \item \textbf{取消校验和}:由上层协议负责
  \item \textbf{取消分片}:只能在源端分片
\end{itemize}

\textbf{IPv6与IPv4对比:}
\begin{longtable}{@{}p{3cm}p{5cm}p{5cm}@{}}
\toprule
\textbf{特性} & \textbf{IPv4} & \textbf{IPv6} \\\\ \midrule
\endhead

地址长度 & 32位 & 128位 \\\\
首部长度 & 20-60字节(可变) & 40字节(固定) \\\\
校验和 & 有 & 无 \\\\
分片 & 路由器可分片 & 只能源端分片 \\\\
地址配置 & 手动或DHCP & 自动配置 \\\\
安全性 & 可选IPSec & 强制IPSec \\\\
QoS支持 & 有限 & 内置支持 \\\\

\bottomrule
\end{longtable}

\textbf{IPv4到IPv6过渡技术:}
\begin{itemize}
  \item \textbf{双协议栈}:同时运行IPv4和IPv6
  \item \textbf{隧道技术}:IPv6数据包封装在IPv4中传输
  \item \textbf{地址转换}:NAT64等技术
\end{itemize}

\section{路由算法与路由协议}

\subsection{路由算法}

\textbf{路由算法分类:}
\begin{longtable}{@{}p{3cm}p{5cm}p{5cm}@{}}
\toprule
\textbf{分类标准} & \textbf{类型1} & \textbf{类型2} \\\\ \midrule
\endhead

适应性 & 静态路由(人工配置) & 动态路由(自动计算) \\\\
信息来源 & 全局信息 & 局部信息 \\\\
算法类型 & 距离向量算法 & 链路状态算法 \\\\

\bottomrule
\end{longtable}

\textbf{距离向量算法(DV):}
\begin{itemize}
  \item \textbf{基本思想}:每个节点维护到所有目的地的距离向量
  \item \textbf{更新规则}:$D_x(y) = \min_v\{c(x,v) + D_v(y)\}$
  \item \textbf{信息交换}:定期与邻居交换距离向量
  \item \textbf{优点}:简单、分布式
  \item \textbf{缺点}:收敛慢、无穷计数问题
\end{itemize}

\textbf{链路状态算法(LS):}
\begin{itemize}
  \item \textbf{基本思想}:每个节点维护整个网络拓扑
  \item \textbf{算法核心}:Dijkstra最短路径算法
  \item \textbf{信息传播}:LSA(链路状态通告)洪泛
  \item \textbf{优点}:收敛快、无环路
  \item \textbf{缺点}:内存消耗大、计算复杂
\end{itemize}

\textbf{Dijkstra算法步骤:}
\begin{enumerate}
  \item 初始化:设源节点距离为0,其他为∞
  \item 选择当前未访问节点中距离最小的节点
  \item 更新该节点所有邻居的距离
  \item 标记该节点为已访问
  \item 重复直到所有节点被访问
\end{enumerate}

\subsection{自治系统与路由协议}

\textbf{自治系统(AS):}
\begin{itemize}
  \item 在统一管理下的路由器集合
  \item 使用相同的路由协议
  \item AS号(ASN)全球唯一
\end{itemize}

\textbf{路由协议分类:}
\begin{itemize}
  \item \textbf{内部网关协议(IGP)}:AS内部使用
  \item \textbf{外部网关协议(EGP)}:AS之间使用
\end{itemize}

\textbf{RIP协议(距离向量):}
\begin{itemize}
  \item \textbf{度量}:跳数(最大15跳)
  \item \textbf{更新}:每30秒广播路由表
  \item \textbf{特点}:简单、适合小型网络
  \item \textbf{版本}:RIPv1(有类)、RIPv2(无类)
  \item \textbf{防环机制}:水平分割、毒性逆转
\end{itemize}

\textbf{OSPF协议(链路状态):}
\begin{itemize}
  \item \textbf{度量}:带宽(代价值)
  \item \textbf{更新}:LSA洪泛,增量更新
  \item \textbf{特点}:收敛快、支持VLSM、支持认证
  \item \textbf{区域}:骨干区域(Area 0)和普通区域
  \item \textbf{LSA类型}:路由器LSA、网络LSA、汇总LSA等
\end{itemize}

\textbf{BGP协议(路径向量):}
\begin{itemize}
  \item \textbf{用途}:AS间路由、Internet骨干路由
  \item \textbf{特点}:基于策略、支持CIDR
  \item \textbf{路径信息}:AS路径,防止环路
  \item \textbf{连接类型}:eBGP(外部)、iBGP(内部)
  \item \textbf{属性}:下一跳、AS路径、本地优先等
\end{itemize}

\textbf{路由协议对比:}
\begin{longtable}{@{}p{2cm}p{3cm}p{3cm}p{4cm}@{}}
\toprule
\textbf{协议} & \textbf{算法} & \textbf{度量} & \textbf{适用范围} \\\\ \midrule
\endhead

RIP & 距离向量 & 跳数 & 小型网络 \\\\
OSPF & 链路状态 & 带宽 & 企业网络 \\\\
BGP & 路径向量 & 策略 & Internet骨干 \\\\

\bottomrule
\end{longtable}

\section{IP组播}

\textbf{组播基本概念:}
\begin{itemize}
  \item \textbf{单播}:一对一通信
  \item \textbf{广播}:一对所有通信
  \item \textbf{组播}:一对多通信(一对一组)
  \item \textbf{任播}:一对最近一个通信
\end{itemize}

\textbf{组播优点:}
\begin{itemize}
  \item 节约网络带宽
  \item 减轻服务器负担
  \item 支持多媒体应用
\end{itemize}

\textbf{组播地址:}
\begin{itemize}
  \item \textbf{D类地址}:224.0.0.0-239.255.255.255
  \item \textbf{本地链路组播}:224.0.0.x(不转发)
  \item \textbf{预留组播地址}:
    \begin{itemize}
      \item 224.0.0.1:所有主机
      \item 224.0.0.2:所有路由器
      \item 224.0.0.5:OSPF路由器
    \end{itemize}
\end{itemize}

\textbf{IGMP协议:}
\begin{itemize}
  \item \textbf{功能}:主机向路由器报告组成员关系
  \item \textbf{版本}:IGMPv1、IGMPv2、IGMPv3
  \item \textbf{报文类型}:查询、报告、离开
  \item \textbf{工作原理}:
    \begin{enumerate}
      \item 路由器周期性发送查询
      \item 主机响应报告组成员身份
      \item 离开时发送离开报文
    \end{enumerate}
\end{itemize}

\textbf{组播路由协议:}
\begin{itemize}
  \item \textbf{PIM-DM}:密集模式,洪泛剪枝
  \item \textbf{PIM-SM}:稀疏模式,显式加入
  \item \textbf{DVMRP}:距离向量组播路由
\end{itemize}

\section{移动IP}

\textbf{移动IP基本概念:}
\begin{itemize}
  \item 支持移动节点在不同网络间移动
  \item 保持原有IP地址不变
  \item 维持现有的网络连接
\end{itemize}

\textbf{基本术语:}
\begin{itemize}
  \item \textbf{归属网络}:移动节点的原始网络
  \item \textbf{外地网络}:移动节点当前所在网络
  \item \textbf{归属代理}:归属网络中的路由器
  \item \textbf{外地代理}:外地网络中的路由器
  \item \textbf{转交地址}:外地网络分配的临时地址
\end{itemize}

\textbf{移动IP工作过程:}
\begin{enumerate}
  \item \textbf{代理发现}:移动节点发现可用代理
  \item \textbf{注册}:向归属代理注册转交地址
  \item \textbf{隧道传输}:数据包通过隧道转发
  \item \textbf{注销}:返回归属网络时注销
\end{enumerate}

\section{网络层设备}

\textbf{路由器工作原理:}

\textbf{路由器功能:}
\begin{itemize}
  \item \textbf{路径选择}:运行路由算法,维护路由表
  \item \textbf{分组转发}:根据目的IP地址转发分组
  \item \textbf{协议转换}:连接不同类型的网络
  \item \textbf{流量控制}:防止网络拥塞
\end{itemize}

\textbf{路由器结构:}
\begin{itemize}
  \item \textbf{输入端口}:物理层和数据链路层处理
  \item \textbf{交换结构}:高速转发分组
  \item \textbf{输出端口}:输出链路调度和队列管理
  \item \textbf{路由处理器}:运行路由协议,维护路由表
\end{itemize}

\textbf{转发表查找:}
\begin{itemize}
  \item \textbf{最长前缀匹配}:选择匹配位数最多的表项
  \item \textbf{默认路由}:0.0.0.0/0,匹配所有目的地
  \item \textbf{查找算法}:线性查找、二分查找、树查找
\end{itemize}

\textbf{路由器vs交换机:}
\begin{longtable}{@{}p{3cm}p{5cm}p{5cm}@{}}
\toprule
\textbf{特性} & \textbf{交换机} & \textbf{路由器} \\\\ \midrule
\endhead

工作层次 & 数据链路层 & 网络层 \\\\
转发依据 & MAC地址 & IP地址 \\\\
广播域 & 不分割 & 分割 \\\\
冲突域 & 分割 & 分割 \\\\
主要功能 & 帧交换 & 路由选择 \\\\

\bottomrule
\end{longtable}

\textbf{记忆要点:}
\begin{itemize}
  \item 网络层实现端到端的数据传输
  \item IP地址分为网络号和主机号两部分
  \item 路由算法分为距离向量和链路状态两大类
  \item ICMP用于差错报告,不纠正差错
  \item IPv6地址128位,IPv4地址32位
  \item 路由器工作在网络层,交换机工作在数据链路层
\end{itemize}

% ========== 第5章 传输层 ==========
\chapter{传输层}

\section{传输层概述}

\textbf{传输层功能:}为应用进程提供逻辑通信,实现端到端的数据传输

\textbf{端口号分类:}
\begin{itemize}
  \item \textbf{熟知端口}:0-1023(系统保留)
  \item \textbf{注册端口}:1024-49151(应用程序)
  \item \textbf{动态端口}:49152-65535(临时分配)
\end{itemize}

\textbf{复用与分解:}
\begin{itemize}
  \item \textbf{复用}:发送方多个进程数据通过一个传输层协议发送
  \item \textbf{分解}:接收方根据端口号将数据分发给对应进程
\end{itemize}

\section{UDP协议}

\textbf{UDP特点:}
\begin{itemize}
  \item 无连接、不可靠、面向报文
  \item 开销小、传输效率高
  \item 支持一对一、一对多、多对多通信
\end{itemize}

\textbf{UDP首部格式(8字节):}
\begin{center}
\begin{tabular}{|c|c|}
\hline
源端口号(16bit) & 目的端口号(16bit) \\
\hline
UDP长度(16bit) & UDP校验和(16bit) \\
\hline
\multicolumn{2}{|c|}{数据} \\
\hline
\end{tabular}
\end{center}

\textbf{适用场景:}DNS查询、TFTP、SNMP、音视频流

\section{TCP协议}

\subsection{TCP首部}
\begin{center}
\small
\begin{tabular}{|c|c|c|c|}
\hline
\multicolumn{4}{|c|}{0 \quad\quad\quad\quad\quad\quad\quad\quad 16 \quad\quad\quad\quad\quad\quad\quad\quad 31} \\
\hline
\multicolumn{2}{|c|}{源端口 (16)} & \multicolumn{2}{c|}{目的端口 (16)} \\
\hline
\multicolumn{4}{|c|}{序号 (32)} \\
\hline
\multicolumn{4}{|c|}{确认号 (32)} \\
\hline
数据偏移(4) & 保留(6) & 控制位(6) & 窗口大小(16) \\
\hline
\multicolumn{2}{|c|}{校验和 (16)} & \multicolumn{2}{c|}{紧急指针 (16)} \\
\hline
\multicolumn{4}{|c|}{选项(可变) + 填充} \\
\hline
\multicolumn{4}{|c|}{数据部分} \\
\hline
\end{tabular}
\end{center}

\subsection{TCP特点}
\begin{itemize}
  \item \textbf{面向连接}:通信前需建立连接
  \item \textbf{可靠传输}:无差错、不丢失、不重复、按序到达
  \item \textbf{面向字节流}:把应用层数据看成字节流
  \item \textbf{全双工通信}:双向同时传输
  \item \textbf{点对点}:只能一对一通信
\end{itemize}

\subsection{TCP报文段}

\textbf{重要字段:}
\begin{itemize}
  \item \textbf{序号}:本报文段第一个字节的序号
  \item \textbf{确认号}:期望收到下个报文段的第一个字节序号
  \item \textbf{窗口大小}:接收窗口大小,用于流量控制
  \item \textbf{标志位}:URG、ACK、PSH、RST、SYN、FIN
\end{itemize}

\subsection{TCP连接管理}

\textbf{三次握手建立连接:}
\begin{enumerate}
  \item 客户端:SYN=1, seq=x
  \item 服务器:SYN=1, ACK=1, seq=y, ack=x+1  
  \item 客户端:ACK=1, seq=x+1, ack=y+1
\end{enumerate}

\textbf{四次挥手释放连接:}
\begin{enumerate}
  \item 客户端:FIN=1, seq=u
  \item 服务器:ACK=1, seq=v, ack=u+1
  \item 服务器:FIN=1, ACK=1, seq=w, ack=u+1
  \item 客户端:ACK=1, seq=u+1, ack=w+1
\end{enumerate}

\textbf{记忆要点:}
\begin{itemize}
  \item 三次握手:防止失效连接请求到达服务器
  \item 四次挥手:需要等待服务器数据发送完成
  \item TIME\_WAIT状态:等待2MSL,确保最后ACK到达
\end{itemize}

\subsection{TCP可靠传输}

\textbf{可靠传输机制:}
\begin{enumerate}
  \item \textbf{序号与确认号}:保证按序、无重复
  \item \textbf{校验和}:检测差错
  \item \textbf{定时器}:超时重传
  \item \textbf{滑动窗口}:流量控制和效率
\end{enumerate}

\textbf{超时重传机制:}
\begin{itemize}
  \item \textbf{RTT测量}:$RTT_s = (1-\alpha) \times RTT_s + \alpha \times RTT_m$
  \item \textbf{RTO计算}:$RTO = RTT_s + 4 \times RTT_D$
  \item \textbf{超时翻倍}:重传时RTO翻倍,最大64秒
\end{itemize}

\textbf{快速重传:}
\begin{itemize}
  \item 收到3个重复ACK立即重传
  \item 不等超时,提高效率
\end{itemize}

\subsection{TCP流量控制}

\textbf{目的:}防止发送方发送过快,导致接收方缓存溢出

\textbf{滑动窗口机制:}
\begin{itemize}
  \item \textbf{接收窗口rwnd}:接收方通告的可用缓存空间
  \item \textbf{发送窗口}:min(rwnd, cwnd)
  \item \textbf{零窗口}:rwnd=0时停止发送,定期探测
\end{itemize}

\subsection{TCP拥塞控制}

\textbf{拥塞控制vs流量控制:}
\begin{itemize}
  \item 流量控制:考虑接收方处理能力
  \item 拥塞控制:考虑网络传输能力
\end{itemize}

\textbf{四种算法:}

\begin{enumerate}
  \item \textbf{慢开始}:cwnd指数增长至ssthresh
    \[cwnd_{new} = cwnd_{old} \times 2\]
  
  \item \textbf{拥塞避免}:cwnd线性增长
    \[cwnd_{new} = cwnd_{old} + 1\]
  
  \item \textbf{快重传}:收到3个重复ACK立即重传
  
  \item \textbf{快恢复}:快重传后cwnd减半,进入拥塞避免
\end{enumerate}

\textbf{拥塞控制状态转换:}
\begin{itemize}
  \item \textbf{超时}:ssthresh = cwnd/2, cwnd = 1,慢开始
  \item \textbf{快重传}:ssthresh = cwnd/2, cwnd = ssthresh,拥塞避免
\end{itemize}

\textbf{记忆口诀:}
\begin{itemize}
  \item 慢开始:小心试探,指数增长
  \item 拥塞避免:稳步前进,线性增长  
  \item 快重传:及时发现,立即重传
  \item 快恢复:适度惩罚,减半继续
\end{itemize}

% ========== 第6章 应用层 ==========
\chapter{应用层}

\section{应用层概述}

\textbf{网络应用模型:}
\begin{enumerate}
  \item \textbf{客户/服务器模式(C/S)}
    \begin{itemize}
      \item 服务器:永远运行,固定IP地址,提供服务
      \item 客户端:主动发起通信,可间歇运行
    \end{itemize}
  
  \item \textbf{对等模式(P2P)}
    \begin{itemize}
      \item 对等方:既是客户端又是服务器
      \item 直接通信,无专用服务器
      \item 可扩展性好,但管理复杂
    \end{itemize}
\end{enumerate}

\section{动态主机配置协议DHCP}

\textbf{DHCP功能:}
动态分配IP地址和网络配置参数,实现网络的即插即用。

\textbf{DHCP优点:}
\begin{itemize}
  \item \textbf{自动配置}:减少手工配置错误
  \item \textbf{集中管理}:统一管理IP地址分配
  \item \textbf{动态分配}:提高IP地址利用率
  \item \textbf{移动支持}:便于移动设备接入网络
\end{itemize}

\textbf{DHCP工作原理:}

\textbf{DHCP报文类型:}
\begin{longtable}{@{}p{3cm}p{4cm}p{6cm}@{}}
\toprule
\textbf{报文类型} & \textbf{发送方向} & \textbf{功能} \\\\ \midrule
\endhead

DHCP DISCOVER & 客户端→服务器 & 广播寻找DHCP服务器 \\\\
DHCP OFFER & 服务器→客户端 & 提供IP地址配置 \\\\
DHCP REQUEST & 客户端→服务器 & 请求使用特定IP地址 \\\\
DHCP ACK & 服务器→客户端 & 确认分配IP地址 \\\\
DHCP NAK & 服务器→客户端 & 拒绝IP地址请求 \\\\
DHCP RELEASE & 客户端→服务器 & 释放IP地址 \\\\
DHCP RENEW & 客户端→服务器 & 续租IP地址 \\\\

\bottomrule
\end{longtable}

\textbf{DHCP四步握手过程:}
\begin{enumerate}
  \item \textbf{发现(DISCOVER)}:
    \begin{itemize}
      \item 客户端开机后广播DHCP DISCOVER报文
      \item 源IP:0.0.0.0,目的IP:255.255.255.255
      \item 包含客户端MAC地址
    \end{itemize}
    
  \item \textbf{提供(OFFER)}:
    \begin{itemize}
      \item DHCP服务器收到DISCOVER后回应OFFER报文
      \item 包含建议的IP地址、子网掩码、默认网关、DNS服务器
      \item 包含租约时间
    \end{itemize}
    
  \item \textbf{请求(REQUEST)}:
    \begin{itemize}
      \item 客户端选择一个OFFER,发送REQUEST报文
      \item 广播形式,让其他DHCP服务器知道选择结果
      \item 包含所选择的服务器标识符
    \end{itemize}
    
  \item \textbf{确认(ACK)}:
    \begin{itemize}
      \item 被选中的DHCP服务器发送ACK确认
      \item 包含最终的IP配置参数
      \item 客户端开始使用分配的IP地址
    \end{itemize}
\end{enumerate}

\textbf{DHCP租约管理:}

\textbf{租约时间:}
\begin{itemize}
  \item 默认租约时间:通常为8天
  \item T1时间:租约时间的50\%,开始续租
  \item T2时间:租约时间的87.5\%,向任何DHCP服务器续租
  \item 租约到期:释放IP地址,重新申请
\end{itemize}

\textbf{续租过程:}
\begin{itemize}
  \item 客户端在T1时间向原服务器发送REQUEST
  \item 服务器回应ACK延长租约
  \item 如果原服务器不可达,在T2时间广播REQUEST
  \item 任何DHCP服务器都可以响应续租
\end{itemize}

\textbf{DHCP中继代理:}
\begin{itemize}
  \item 解决DHCP服务器跨网段问题
  \item 中继代理接收客户端广播
  \item 单播转发给DHCP服务器
  \item 转发服务器响应给客户端
\end{itemize}

\textbf{DHCP报文格式:}
\begin{longtable}{@{}p{3cm}p{2cm}p{8cm}@{}}
\toprule
\textbf{字段} & \textbf{长度} & \textbf{说明} \\\\ \midrule
\endhead

OP & 1字节 & 操作类型:1=请求,2=回应 \\\\
HTYPE & 1字节 & 硬件地址类型:1=以太网 \\\\
HLEN & 1字节 & 硬件地址长度:6=MAC地址 \\\\
HOPS & 1字节 & 中继跳数 \\\\
XID & 4字节 & 事务ID,匹配请求和响应 \\\\
SECS & 2字节 & 客户端启动后经过的秒数 \\\\
FLAGS & 2字节 & 标志位,最高位为广播标志 \\\\
CIADDR & 4字节 & 客户端IP地址 \\\\
YIADDR & 4字节 & 你的IP地址(服务器分配) \\\\
SIADDR & 4字节 & 服务器IP地址 \\\\
GIADDR & 4字节 & 网关IP地址(中继代理) \\\\
CHADDR & 16字节 & 客户端硬件地址 \\\\
OPTIONS & 可变 & 选项字段,包含各种配置参数 \\\\

\bottomrule
\end{longtable}

\textbf{DHCP选项:}
\begin{itemize}
  \item \textbf{选项1}:子网掩码
  \item \textbf{选项3}:默认网关
  \item \textbf{选项6}:DNS服务器
  \item \textbf{选项51}:IP地址租约时间
  \item \textbf{选项53}:DHCP消息类型
  \item \textbf{选项54}:DHCP服务器标识符
\end{itemize}

\textbf{DHCP安全考虑:}
\begin{itemize}
  \item \textbf{DHCP欺骗}:恶意DHCP服务器分配错误配置
  \item \textbf{DHCP饥饿攻击}:耗尽IP地址池
  \item \textbf{防护措施}:DHCP侦听、端口安全、MAC地址绑定
\end{itemize}

\textbf{DHCP故障排除:}
\begin{itemize}
  \item 检查DHCP服务器状态
  \item 检查IP地址池是否耗尽
  \item 检查网络连通性
  \item 检查DHCP中继代理配置
\end{itemize}

\section{域名系统DNS}

\textbf{DNS功能:}域名与IP地址的映射

\textbf{域名层次结构:}
\begin{center}
\texttt{www.example.com.}
\end{center}
\begin{itemize}
  \item 根域:. (dot)
  \item 顶级域:com, org, cn, edu
  \item 二级域:example
  \item 主机名:www
\end{itemize}

\textbf{DNS查询过程:}
\begin{enumerate}
  \item 递归查询:客户端 → 本地DNS服务器
  \item 迭代查询:本地DNS → 根DNS → 顶级域DNS → 权威DNS
\end{enumerate}

\section{万维网WWW}

\subsection{HTTP协议}

\textbf{HTTP特点:}
\begin{itemize}
  \item 基于TCP,端口80
  \item 无状态协议
  \item 支持持久连接和非持久连接
\end{itemize}

\textbf{HTTP方法:}
\begin{itemize}
  \item \textbf{GET}:请求资源
  \item \textbf{POST}:提交数据
  \item \textbf{PUT}:上传文件
  \item \textbf{DELETE}:删除资源
  \item \textbf{HEAD}:只要响应头
\end{itemize}

\textbf{常见状态码:}
\begin{longtable}{@{}clp{6cm}@{}}
\toprule
\textbf{代码} & \textbf{含义} & \textbf{说明} \\\\ \midrule
\endhead

200 & OK & 请求成功 \\\\
301 & Moved Permanently & 永久重定向 \\\\
302 & Found & 临时重定向 \\\\
304 & Not Modified & 未修改,使用缓存 \\\\
400 & Bad Request & 请求语法错误 \\\\
401 & Unauthorized & 需要身份验证 \\\\
403 & Forbidden & 服务器拒绝请求 \\\\
404 & Not Found & 资源不存在 \\\\
500 & Internal Server Error & 服务器内部错误 \\\\

\bottomrule
\end{longtable}

\section{文件传输协议FTP}

\textbf{FTP基本概念:}
\begin{itemize}
  \item 基于TCP的可靠文件传输协议
  \item 使用两个并行连接:控制连接和数据连接
  \item 支持ASCII和二进制两种传输模式
\end{itemize}

\textbf{FTP连接模式:}

\textbf{控制连接:}
\begin{itemize}
  \item 端口21,持续整个FTP会话
  \item 传输FTP命令和响应
  \item 面向字符,使用7位ASCII编码
\end{itemize}

\textbf{数据连接:}
\begin{itemize}
  \item 端口20(主动模式)或动态端口(被动模式)
  \item 只在传输文件时建立
  \item 每传输一个文件就建立和关闭一次
\end{itemize}

\textbf{主动模式vs被动模式:}
\begin{longtable}{@{}p{3cm}p{5cm}p{5cm}@{}}
\toprule
\textbf{模式} & \textbf{主动模式(PORT)} & \textbf{被动模式(PASV)} \\\\ \midrule
\endhead

数据连接发起方 & 服务器 & 客户端 \\\\
服务器数据端口 & 20 & 动态分配 \\\\
防火墙友好性 & 较差 & 较好 \\\\
NAT兼容性 & 较差 & 较好 \\\\
适用场景 & 早期应用 & 现代应用 \\\\

\bottomrule
\end{longtable}

\textbf{常用FTP命令:}
\begin{itemize}
  \item \textbf{USER username}:指定用户名
  \item \textbf{PASS password}:指定密码
  \item \textbf{LIST}:列出目录内容
  \item \textbf{RETR filename}:下载文件
  \item \textbf{STOR filename}:上传文件
  \item \textbf{PWD}:显示当前目录
  \item \textbf{CWD path}:改变目录
  \item \textbf{QUIT}:退出FTP会话
\end{itemize}

\textbf{FTP响应码:}
\begin{itemize}
  \item \textbf{1xx}:肯定的初步回答
  \item \textbf{2xx}:肯定的完成回答(如220 Ready)
  \item \textbf{3xx}:肯定的中间回答(如331 Password required)
  \item \textbf{4xx}:瞬时否定的完成回答(如425 Can't open connection)
  \item \textbf{5xx}:永久性否定的完成回答(如500 Command not recognized)
\end{itemize}

\section{电子邮件系统}

\textbf{电子邮件系统组成:}
\begin{itemize}
  \item \textbf{用户代理(UA)}:用户与邮件系统的接口
  \item \textbf{邮件服务器}:存储和转发邮件
  \item \textbf{邮件传输协议}:SMTP、POP3、IMAP
\end{itemize}

\textbf{邮件地址格式:}
\begin{center}
\texttt{username@domain.com}
\end{center}

\textbf{电子邮件格式与MIME:}

\textbf{RFC 822邮件格式:}
\begin{itemize}
  \item \textbf{信封}:传输时使用的地址信息
  \item \textbf{首部}:邮件头字段
    \begin{itemize}
      \item From:发件人
      \item To:收件人
      \item Subject:主题
      \item Date:发送日期
      \item Cc:抄送
      \item Bcc:密件抄送
    \end{itemize}
  \item \textbf{主体}:邮件正文内容
\end{itemize}

\textbf{MIME扩展:}
\begin{itemize}
  \item \textbf{多用途互联网邮件扩展}:支持多媒体邮件
  \item \textbf{解决问题}:RFC 822只支持7位ASCII字符
  \item \textbf{MIME首部字段}:
    \begin{itemize}
      \item MIME-Version:版本信息
      \item Content-Type:内容类型
      \item Content-Transfer-Encoding:传输编码
      \item Content-Disposition:内容处置
    \end{itemize}
\end{itemize}

\textbf{主要MIME类型:}
\begin{longtable}{@{}p{3cm}p{9cm}@{}}
\toprule
\textbf{内容类型} & \textbf{说明} \\\\ \midrule
\endhead

text/plain & 纯文本 \\\\
text/html & HTML文档 \\\\
image/jpeg & JPEG图像 \\\\
image/gif & GIF图像 \\\\
audio/mpeg & MP3音频 \\\\
video/mp4 & MP4视频 \\\\
application/pdf & PDF文档 \\\\
multipart/mixed & 多部分混合内容 \\\\
multipart/alternative & 多部分可选内容 \\\\

\bottomrule
\end{longtable}

\textbf{MIME编码方式:}
\begin{itemize}
  \item \textbf{Base64}:将二进制数据转换为64个可打印字符
  \item \textbf{Quoted-Printable}:可打印字符引用编码
  \item \textbf{7bit}:7位ASCII字符
  \item \textbf{8bit}:8位字符
\end{itemize}

\subsection{SMTP协议}

\textbf{SMTP特点:}
\begin{itemize}
  \item 简单邮件传输协议,端口25
  \item 基于TCP的可靠传输
  \item 只能传输7位ASCII字符
  \item 推送协议(push)
\end{itemize}

\textbf{SMTP工作过程:}
\begin{enumerate}
  \item 建立TCP连接(端口25)
  \item 握手:服务器发送220 Ready
  \item 邮件传输:
    \begin{itemize}
      \item HELO/EHLO:客户端标识
      \item MAIL FROM:发件人地址
      \item RCPT TO:收件人地址
      \item DATA:邮件内容
    \end{itemize}
  \item 结束:QUIT命令关闭连接
\end{enumerate}

\textbf{SMTP命令示例:}
\begin{lstlisting}[language=bash,caption={SMTP会话示例}]
S: 220 mail.example.com Ready
C: HELO client.example.com
S: 250 Hello client.example.com
C: MAIL FROM:<sender@example.com>
S: 250 OK
C: RCPT TO:<recipient@example.com>
S: 250 OK
C: DATA
S: 354 Start mail input
C: Subject: Test Email
C: 
C: This is a test message.
C: .
S: 250 Message accepted
C: QUIT
S: 221 Goodbye
\end{lstlisting}

\textbf{扩展SMTP (ESMTP):}
\begin{itemize}
  \item 支持认证:AUTH命令
  \item 支持加密:STARTTLS
  \item 支持二进制传输:8BITMIME
  \item 支持较大邮件:CHUNKING
\end{itemize}

\subsection{POP3协议}

\textbf{POP3特点:}
\begin{itemize}
  \item 邮局协议版本3,端口110
  \item 拉取协议(pull)
  \item 下载并删除模式
  \item 无状态协议
\end{itemize}

\textbf{POP3工作阶段:}
\begin{enumerate}
  \item \textbf{认证阶段}:
    \begin{itemize}
      \item USER username:指定用户名
      \item PASS password:指定密码
    \end{itemize}
  
  \item \textbf{事务阶段}:
    \begin{itemize}
      \item LIST:列出邮件
      \item RETR n:检索第n封邮件
      \item DELE n:删除第n封邮件
      \item STAT:获取邮箱统计信息
    \end{itemize}
  
  \item \textbf{更新阶段}:
    \begin{itemize}
      \item QUIT:退出,删除标记的邮件
    \end{itemize}
\end{enumerate}

\subsection{IMAP协议}

\textbf{IMAP特点:}
\begin{itemize}
  \item 互联网消息访问协议,端口143
  \item 邮件保留在服务器上
  \item 支持多设备同步
  \item 有状态协议
\end{itemize}

\textbf{IMAP优势:}
\begin{itemize}
  \item 支持在线和离线工作
  \item 支持文件夹管理
  \item 支持部分邮件下载
  \item 支持搜索功能
  \item 支持多用户访问同一邮箱
\end{itemize}

\textbf{邮件协议对比:}
\begin{longtable}{@{}p{2.5cm}p{4cm}p{3cm}p{3cm}@{}}
\toprule
\textbf{特性} & \textbf{SMTP} & \textbf{POP3} & \textbf{IMAP} \\\\ \midrule
\endhead

用途 & 发送邮件 & 接收邮件 & 接收邮件 \\\\
端口 & 25 & 110 & 143 \\\\
邮件存储 & 转发 & 下载删除 & 服务器保存 \\\\
多设备支持 & N/A & 不支持 & 支持 \\\\
离线工作 & N/A & 支持 & 支持 \\\\
文件夹管理 & N/A & 不支持 & 支持 \\\\

\bottomrule
\end{longtable}

\textbf{邮件安全:}
\begin{itemize}
  \item \textbf{SMTPS}:SMTP over SSL/TLS,端口465/587
  \item \textbf{POP3S}:POP3 over SSL/TLS,端口995
  \item \textbf{IMAPS}:IMAP over SSL/TLS,端口993
  \item \textbf{SPF}:发送方策略框架,防止邮件伪造
  \item \textbf{DKIM}:域名密钥识别邮件,数字签名
  \item \textbf{DMARC}:域名消息认证报告一致性
\end{itemize}



% ========== 附录:术语速查 ==========
\appendix

\chapter{重要术语速查表}

\section{中英文术语对照}
\begin{longtable}{@{}p{4cm}p{4cm}p{6cm}@{}}
\toprule
\textbf{中文术语} & \textbf{英文术语} & \textbf{简要说明} \\\\ \midrule
\endhead

计算机网络 & Computer Network & 互连的自主计算机集合 \\\\
带宽 & Bandwidth & 信道能通过的最高数据率 \\\\
时延 & Delay/Latency & 数据从源到目的地所需时间 \\\\
吞吐量 & Throughput & 单位时间内实际传输的数据量 \\\\
协议 & Protocol & 网络中数据交换的规则集合 \\\\
分组交换 & Packet Switching & 将数据分割成分组传输 \\\\
电路交换 & Circuit Switching & 建立专用物理链路 \\\\
以太网 & Ethernet & 最常用的局域网技术 \\\\
CSMA/CD & Carrier Sense Multiple Access with Collision Detection & 载波监听多路访问/冲突检测 \\\\
交换机 & Switch & 数据链路层设备,学习MAC地址 \\\\
路由器 & Router & 网络层设备,转发IP分组 \\\\
网关 & Gateway & 连接不同网络的设备 \\\\
子网掩码 & Subnet Mask & 用于划分网络和主机部分 \\\\
VLSM & Variable Length Subnet Mask & 可变长子网掩码 \\\\
CIDR & Classless Inter-Domain Routing & 无类域间路由 \\\\
NAT & Network Address Translation & 网络地址转换 \\\\
DHCP & Dynamic Host Configuration Protocol & 动态主机配置协议 \\\\
ARP & Address Resolution Protocol & 地址解析协议 \\\\
ICMP & Internet Control Message Protocol & 网际控制报文协议 \\\\
RIP & Routing Information Protocol & 路由信息协议 \\\\
OSPF & Open Shortest Path First & 开放最短路径优先 \\\\
BGP & Border Gateway Protocol & 边界网关协议 \\\\
TCP & Transmission Control Protocol & 传输控制协议 \\\\
UDP & User Datagram Protocol & 用户数据报协议 \\\\
HTTP & HyperText Transfer Protocol & 超文本传输协议 \\\\
HTTPS & HTTP Secure & 安全的HTTP \\\\
FTP & File Transfer Protocol & 文件传输协议 \\\\
SMTP & Simple Mail Transfer Protocol & 简单邮件传输协议 \\\\
POP3 & Post Office Protocol 3 & 邮局协议版本3 \\\\
IMAP & Internet Message Access Protocol & 互联网消息访问协议 \\\\
DNS & Domain Name System & 域名系统 \\\\
SNMP & Simple Network Management Protocol & 简单网络管理协议 \\\\
SSL/TLS & Secure Socket Layer/Transport Layer Security & 安全套接字层/传输层安全 \\\\
VPN & Virtual Private Network & 虚拟专用网络 \\\\
IPSec & IP Security & IP安全协议 \\\\
DES & Data Encryption Standard & 数据加密标准 \\\\
AES & Advanced Encryption Standard & 高级加密标准 \\\\
RSA & Rivest-Shamir-Adleman & RSA公钥算法 \\\\
MD5 & Message Digest 5 & 消息摘要算法5 \\\\
SHA & Secure Hash Algorithm & 安全散列算法 \\\\

\bottomrule
\end{longtable}

\section{重要公式汇总}
\subsection{性能计算公式}
\begin{align}
\text{发送时延} &= \frac{L}{R} \quad (L:\text{分组长度},R:\text{发送速率}) \\\\
\text{传播时延} &= \frac{d}{s} \quad (d:\text{距离},s:\text{传播速率}) \\\\
\text{总时延} &= \text{发送时延} + \text{传播时延} + \text{处理时延} + \text{排队时延} \\\\
\text{时延带宽积} &= \text{传播时延} \times \text{带宽} \\\\
\text{信道利用率} &= \frac{T_0}{T_0 + RTT} \quad (\text{停止等待协议}) \\\\
\text{香农公式} &= W \log_2(1 + S/N) \quad (\text{信道容量}) \\\\
\text{奈奎斯特公式} &= 2W \log_2 V \quad (\text{理想信道})
\end{align}

\subsection{子网划分公式}
\begin{align}
\text{子网数} &= 2^n \quad (n:\text{借用的主机位数}) \\\\
\text{每个子网主机数} &= 2^m - 2 \quad (m:\text{剩余主机位数}) \\\\
\text{子网地址} &= \text{网络地址} \text{ AND } \text{子网掩码}
\end{align}

\subsection{TCP窗口计算}
\begin{align}
\text{发送窗口} &= \min(\text{拥塞窗口}, \text{接收窗口}) \\\\
\text{慢开始阈值} &= \max(\text{飞行中字节数}/2, 2 \times MSS) \\\\
\text{超时间隔} &= \text{估计RTT} + 4 \times \text{RTT偏差}
\end{align}

\section{常用端口号}
\begin{longtable}{@{}llp{8cm}@{}}
\toprule
\textbf{端口号} & \textbf{协议} & \textbf{说明} \\\\ \midrule
\endhead

20/21 & FTP & 文件传输协议(20数据,21控制) \\\\
22 & SSH & 安全外壳协议 \\\\
23 & Telnet & 远程登录协议 \\\\
25 & SMTP & 简单邮件传输协议 \\\\
53 & DNS & 域名系统 \\\\
80 & HTTP & 超文本传输协议 \\\\
110 & POP3 & 邮局协议版本3 \\\\
143 & IMAP & 互联网消息访问协议 \\\\
443 & HTTPS & 安全的HTTP \\\\
993 & IMAPS & 安全的IMAP \\\\
995 & POP3S & 安全的POP3 \\\\

\bottomrule
\end{longtable}

% ================== 转录说明 ==================
% 上述章节骨架已建立完成,接下来请:
% 1. 逐章对照PDF原文,将具体内容填入对应章节
% 2. 补充图表:在适当位置插入 \includegraphics 和 table 环境
% 3. 完善公式:使用 align、equation 等环境
% 4. 添加交叉引用:用 \label 和 \ref 建立章节、图表间联系
% 5. 调整术语表:根据实际PDF内容补充重要术语
% ================== 转录区域结束 ==================

\end{document}
